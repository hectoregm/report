\section{Biblioteca Bezel}

Para lograr la integración entre el sistema \texttt{PEAT} y el servicio web C3,
el API proporcionado por el servidor C3, ha sido impementado una biblioteca
con el nombre de \textbf{\texttt{Bezel}}.

\texttt{PEAT} hace uso del marco de trabajo Rails, el cual a su vez hace
uso del patrón de arquitectura MVC. En Rails los modelos por lo general hacen uso
de la biblioteca \texttt{ActiveRecord} la cual permite implementar modelos basados
en tablas de bases de datos relacionales. En \texttt{PEAT} los datos
son proporcionados indirectamente por una base de datos relacional en su lar son
proporcionados por el servicio web C3 proporcionado por el servidor C3.

Dentro del marco de trabajo Rails también se tiene la biblioteca
\texttt{ActiveResource} la cual implementa el caso de uso en que los datos son
provistos por un servicio web tipo RESTful.

En una primera versión se trató de realizar la integración de \texttt{PEAT} con el
servicio web C3 haciendo uso de esta biblioteca, sin embargo se tenían las siguientes
dificultades:

\begin{itemize}
\item Presupone que el servicio web hace uso de todos los métodos HTTP (GET, POST,
  UPDATE, DELETE) para definir las acciones básicas sobre un recurso.
\item No permite la definición de asociaciones entre recursos.
\end{itemize}

Aunque el servicio web C3 es de tipo RESTful, éste no hace uso de los métodos
HTTP para realizar acciones sobre los recursos del servicio. El servicio web
C3 realiza todas sus acciones usando el método HTTP POST.

Otra dificultad era que la biblioteca \texttt{ActiveResource} no da ninguna facilidad
para definir asociaciones entre los modelos lo que implicaba el tener que implementar
una gran cantidad de código dado el gran numero de asociaciones presentes entre los
tipos definidos por el servicio web C3.

Dada las dificultades descritas en los párrafos anteriores el hacer uso de
\texttt{ActiveResource} no era una solución viable, sin embargo sirvió como una
primera aproximación para implementar la biblioteca \texttt{Bezel} y así
proporcionar acceso al servicio web C3.

Cabe recordar que el servidor C3 está implementado en dos lenguajes de programación,
la gran mayoría implementado con Javascript, mientras que el núcleo y funciones
criticas están implementadas en Java. El servidor C3 puede enviar tanto XML como
JSON como respuesta a las peticiones realizadas al servidor\footnote{La representación en JSON
  es la más utilizada dado que el sistema de tipos que se usa en el sistema esta
  definido en Javascript.}.

\texttt{Bezel} permite la reconstrucción automática de los tipos definidos por el
servidor C3 en Javascript a modelos compatibles con el marco de trabajo Rails
y el lenguaje de programación Ruby. Esta reconstrucción automática permite
evitar desfases entre la definición de los tipos entre los dos sistemas y
agiliza el desarrollo del sistema \texttt{PEAT}. Esta reconstrucción automática
se hace por medio de un lenguaje de dominio especifico que permite implementar
modelos basados en tipos del servicio web C3 aprovechando que este servicio
es tipo RESTful y las capacidades de metaprogramación de Ruby.

\texttt{Bezel} tiene dos modos principales de uso:
\begin{itemize}
\item Como un cliente REST con ciertas adecuaciones para una mejor integración
  con el API del servidor C3.
\item Como un lenguaje de dominio especifico interno para Ruby para lograr
  la implementación automatizada de tipos y sus asociaciones del servicio web C3.
\end{itemize}

\subsection{Servicio web C3}

El servicio web C3 es de tipo RESTful, es decir, cada recurso o tipo tiene una única
dirección URL asociada. Las direcciones URL tienen la siguiente forma:

\vspace{2.5mm}
\texttt{POST} /api/$<$\textit{version}$>$/$<$\textit{tenant}$>$/$<$\textit{module}$>$;$<$\textit{tag}$>$/$<$\textit{type}$>$?action=$<$\textit{action-name}$>$

\begin{itemize}
\item \textit{version}: indica la versión del servicio a la cual se está accediendo,
  el sistema \texttt{PEAT} hace uso de la versión 1.0.
\item \textit{tenant}: se refiera al producto o cliente, esto es porque cada
  producto y/o cliente tiene cierta funcionalidad específica. Para \texttt{PEAT}
  se tiene \textquote{peat} y \textquote{c3}.
\item \textit{module}: indica el módulo al que se está accediendo, un conjunto de
  tipos se definen en un módulo en particular. Para el sistema \texttt{PEAT} el
  módulo más usado es el módulo \textquote{peat}, haciendo uso de los módulos
  \textquote{billing} y \textquote{structure}.
\item \textit{tag}: indica el tipo de ambiente, es decir, si el servicio web C3 es
  para ambiente de pruebas, control de calidad, producción, etcétera.
  \textit{peatprod} y \textit{peatqa} son algunas de las etiquetas que se usan
  en \texttt{PEAT}.
\item \textit{type}: se refiere al tipo al que se quiere acceder, por ejemplo
  \textquote{Building}, \textquote{Location} y
  \textquote{BuildingEnergyConservationOption} son algunos de los tipos utilizados
  en \texttt{PEAT}.
\item \textit{action-name}: indica la acción que se quiere realizar sobre
  el tipo indicado anteriormente. En este caso \textquote{fetch}, \textquote{upsert}
  y \textquote{remove} son algunas de las acciones que se pueden realizar.
\end{itemize}

Como se comentó anteriormente el servicio web C3 no es totalmente de tipo RESTful,
principalmente porque utiliza del mismo método HTTP POST para todas las acciones,
indicando por medio del \textquote{action-name} la acción a realizar.
Para la implementación de \texttt{Bezel} solamente se espera que el servicio web
tenga direcciones únicas para cada tipo definido.

Muchas acciones reciben ciertos parámetros los cuales son almacenados dentro del
cuerpo de la petición POST, por lo que el único parámetro que se envía por medio
de la URL es el nombre la acción a realizar.

\subsection{Arquitectura general}

La biblioteca \texttt{Bezel} está conformada por cinco clases y cinco módulos,
siendo las clases principales \texttt{Bezel::Client} y \texttt{Bezel::Base}.

En el lenguaje de programación Ruby se tiene el concepto de módulo (\texttt{module}),
el cual es mas que una agrupación de objetos bajo un único nombre. Los objetos
pueden ser constantes, métodos, clases y otros módulos.

Los módulos tienen dos tipos de usos (1) se pueden utilizar un módulo como una
manera conveniente de agrupar objetos o (2) se pueden incorporar los objetos del
módulo a una clase haciendo uso de \texttt{include} o \texttt{extend}.

\lstinputlisting[language=Ruby]{code/module-namespace.rb}

El ejemplo anterior muestra el uso de los módulos como un espacio de nombres,
en la línea 2 se define la constante \texttt{PI} y en las líneas 3 y 7 se definen
los métodos \texttt{sin} y \texttt{cos}, estos métodos son usados fuera del
módulo por medio del nombre del módulo como ocurre en la línea 12, en esta
línea se hace referencia tanto a la constante \texttt{PI} como al método
\texttt{sin}.

\lstinputlisting[language=Ruby]{code/module-include.rb}

Ahora en este ejemplo se muestra el uso de los módulos para incorporar métodos
de clase o de instancia. En las líneas 1-11 se define el módulo \texttt{Boolean}
el cual define dos métodos \texttt{boolean?} y \texttt{to\_bool}, mientras que
en las líneas 13-17 se define el módulo \texttt{RandomString} el cual a su vez
define un método \texttt{random}. En las líneas 19-22 se permite modificar cualquier
clase en tiempo de ejecución aunque esta clase sea parte de la biblioteca estándar,
así en la línea 20 se hace uso de \texttt{include} para agregar los métodos del
módulo dado como parámetro como métodos de instancia, el uso de estos métodos se
se puede ver en las líneas 24-25. Finalmente en la línea 21 se hace uso de la
declaración \texttt{extend} para agregar los métodos del módulo dado como métodos
de clase como se puede ver en la línea 26.

Por lo anterior se tiene que el uso de módulos es una parte importante para
organizar y compartir funcionalidad en el lenguaje de programación Ruby.

Para la biblioteca \texttt{Bezel} tenemos el módulo \texttt{Bezel} que contiene
a todas las clases y módulos de la biblioteca. Además se tienen cinco submódulos
que permiten organizar de forma coherente la funcionalidad de la biblioteca.

Las clases y módulos que conforman la biblioteca son
(Ver Figura \ref{fig:diagrama-clase}):
\begin{itemize}
\item \texttt{Bezel}: módulo principal que contiene a los demás submódulos y
  clases de la biblioteca.
\item \texttt{Bezel::Config}: módulo que define los métodos y valores validos
  para configurar el cliente con el servicio web C3.
\item \texttt{Bezel::Client}: clase que define un cliente para usar el servicio
  web C3.
\item \texttt{Bezel::Target}: clase que representa un tipo dado un contexto
  específico (\textit{tenant}, \textit{tag}, \textit{module}).
\item \texttt{Bezel::Action}: clase que representa una acción sobre
  un tipo representado por un \texttt{Bezel::Target}.
\item \texttt{Bezel::Base}: clase abstracta que permite definir un modelo en Ruby
  con base en un tipo del servicio web C3.
\item \texttt{Bezel::Connections}: módulo que define las acciones básicas y de
  búsqueda sobre un modelo.
\item \texttt{Bezel::Associations}: módulo que define las asociaciones entre
  modelos.
\item \texttt{Bezel::CacheBase}: módulo que redefine las acciones básicas y
  de búsqueda sobre un modelo haciendo uso de un cache \texttt{Bezel::Cache}.
\item \texttt{Bezel::Cache}: clase que define un API para realizar operaciones
  de lectura/escritura sobre un cache.
\end{itemize}

\rjcimage{1.0}{imagenes/Diagrama-Clase-Bezel.png}{Diagrama de clase de la biblioteca
  Bezel}{diagrama-clase}

\subsubsection{Dependencias}

En la implementación de \texttt{Bezel} se hace uso de las siguiente bibliotecas
de apoyo:

\begin{itemize}
\item moneta: proporciona un API uniforme para realizar
  operaciones de lectura y escritura en un repositorio de cache.
\item faraday: proporciona un API uniforme para realizar
  peticiones HTTP.
\item hashie: proporciona un conjunto de metodos que extienden la funcionalidad
  de la clase \texttt{Hash} en Ruby.
\item activemodel: proporciona interfaces para modelos que deben operar sobre
  Rails.
\end{itemize}

\subsection{\texttt{Bezel}}

El módulo \texttt{Bezel} no solo sirve como el espacio de nombres principal para la
biblioteca, pues a su vez implementa varios métodos auxiliares para facilitar la
configuración y creación de clientes para acceder al servicio web C3.

\lstinputlisting[language=Ruby]{code/bezel.rb}

En el código anterior se tiene la implementación abreviada de \texttt{Bezel},
en la línea 2 por medio de \texttt{extend} se añaden varios métodos de clase
para configurar la biblioteca. En las líneas 4-20 se definen los
métodos auxiliares mas importantes:

\begin{itemize}
\item \texttt{new} (líneas 5-7): regresa una nueva instancia \texttt{Bezel::Client}.
\item \texttt{invoke} (líneas 9-11): realiza una petición al servicio web C3
  por medio del cliente asociado al módulo \texttt{Bezel}.
\item \texttt{client} (líneas 13-15): usando una variable de clase se inicializa o
  reusa una instancia \texttt{Bezel::Client}. Este método permite reutilizar un
  mismo cliente y conexión al servidor C3 evitando el costo de establecer un
  nuevo cliente y en establecer una conexión HTTP con el servidor C3.
\item \texttt{context} (líneas 17-19): permite cambiar el contexto de las
  peticiones que se realizan dentro de un bloque, se regresa al contexto original
  al finalizar de ejecutar el código dentro del bloque.
\end{itemize}

\subsubsection{\texttt{Bezel::Config}}

El módulo \texttt{Bezel::Config} define los atributos que permiten configurar
el comportamiento de la biblioteca y de los clientes que se obtienen por medio
de ella. Este módulo se usa tanto en \texttt{Bezel} para definir una configuración
global de la biblioteca como en \texttt{Bezel::Client} para tener también una
configuración local por cliente.

\lstinputlisting[language=Ruby]{code/config.rb}

En el módulo \texttt{Config} se definen dos constantes:

\begin{itemize}
\item \texttt{VALID\_OPTIONS\_KEYS} (líneas 4-13): un arreglo de símbolos que indica
  que atributos son válidos.
\item \texttt{DEFAULT\_VALUES} (líneas 15-24): una tabla \textit{hash} donde se
  indican los valores por defecto de los atributos válidos.
\end{itemize}

Haciendo uso del método \texttt{attr\_accessor} en la línea 26 se definen por cada
símbolo en \texttt{VALID\_OPTIONS\_KEYS} un atributo de instancia con sus
correspondientes métodos de acceso y escritura. El método \texttt{attr\_accessor} es
un ejemplo del uso de metaprogramación en Ruby, dado que con base en el símbolo
dado, se genera un atributo y sus métodos de acceso.

En la línea 30 se tiene el método \texttt{extended} el cual es un \textit{callback}
que es llamado cuando un módulo es extendido, es decir, es utilizado como parámetro
al método \texttt{extend}. Para el módulo \texttt{Config} se tiene que cuando el
módulo es extendido, como es en el caso de \texttt{Config} dentro del módulo
\texttt{Bezel}, se llama al método \texttt{reset} para inicializar todos los
atributos definidos en el módulo con sus valores por defecto.
El método \texttt{extended} es solo uno de varios métodos \textit{callback} que
proporciona Ruby para la metaprogramación.

Otro método muy importante para la metaprogramación es \texttt{send} el cual es usado
en la línea 35 para inicializar los atributos definidos por \texttt{Config}, el
método \texttt{send} invoca el método identificado con un símbolo o cadena dada,
pasando los argumentos posteriores como argumentos a la invocación del método.
Haciendo uso de \texttt{send} y de las dos constantes definidas en el módulo
se puede inicializar cada uno de los atributos.

\subsection{\texttt{Bezel::Client}}

La clase \texttt{Client} representa un cliente que permite realizar peticiones al
servicio web C3, por medio del módulo \texttt{Config} se permite configurar
cada cliente instanciado.

\lstinputlisting[language=Ruby]{code/client.rb}

En la línea 3 se tiene un \texttt{include} que añade las constantes
y atributos para configurar al cliente. En las líneas 7-13 se define el método
\texttt{conn} que inicializa una conexión HTTP según la configuración definida
para el cliente y pidiendo que las respuestas mandadas por el servidor
sean de tipo JSON.

En las líneas 13-49 se tiene la implementación del método \texttt{invoke}, en
la línea 14 se define un \textit{target} de tipo \texttt{Bezel::Target} usando
tanto el contexto del cliente, es decir el \textit{tenant} y \textit{tag} mas
la información del \textit{module} y \textit{type} proporcionados en los argumentos.
En la línea 15 se define una acción de tipo \texttt{Bezel::Action} que representa
una acción sobre un tipo con sus parámetros.

En la línea 17 se define la URL a la cual se hará la petición al servicio web C3,
haciendo uso de la información de los objetos previos, se convierte los parámetros
de la acción al formato JSON para ser enviados en el cuerpo de la petición POST
al servicio web.

% FIXME: explicar línea 32 de addCredentials

En las líneas 22 a 30 se definen los encabezados y cuerpo de la petición POST al
servicio web. Finalmente en la línea 37 se realiza la petición POST por medio
de la conexión creada por el método \texttt{conn} y con los para metros definidos
anteriormente.

Posteriormente el resultado de la petición es almacenado en el objeto \texttt{action}
y este objeto es regresado como resultado de la invocación del método
\texttt{invoke}.

\subsection{\texttt{Bezel::Base}}

\texttt{Bezel::Base} es una clase abstracta la cual define la mayor parte de su
funcionalidad en base al nombre de la clase que hereda de ella. Es decir no se hace
uso directo de \texttt{Bezel::Base} si no que al heredar de esta clase abstracta
se obtiene funcionalidad para crear modelos basados en tipos provenientes
del servicio web C3.

\lstinputlisting[language=Ruby]{code/base-example.rb}

El código anterior ejemplifica el uso de \texttt{Bezel::Base} para definir el modelo
por medio de herencia, en la línea 1 se define \texttt{Recommendation}. En la línea 2
se tiene el método de clase \texttt{set\_c3\_type} que permite al modelo definir
a que tipo del sistema web C3 está ligado, en este caso el modelo esta acotado al
tipo de nombre \texttt{BuildingEnergyConservationOption}. En la línea 4 se define
que el modelo tiene una asociación con el modelo \texttt{EnergyConservationOption}
en forma única por medio del método \texttt{has\_one}.

Ya definido el modelo en la línea 7 se hace la búsqueda de una recomendación por
medio de su identificador único por medio del método \texttt{find}. En la línea 8 se
obtiene el costo total de la recomendación buscada anteriormente por medio de
\texttt{totalCost}. Los métodos \texttt{set\_c3\_type}, \texttt{has\_one} y
\texttt{find} son definidos por \texttt{Bezel::Base} pero el método
\texttt{totalCost} es definido en tiempo de ejecución por medio de la
metaprogramación.

\subsubsection{Métodos de clase}

Los métodos de clase que se definen en \texttt{Bezel::Base} son necesarios para
definir métodos de configuración del modelo y para realizar operaciones
de búsqueda sobre el modelo.

\lstinputlisting[language=Ruby]{code/base-class-methods.rb}

Los métodos principales son:
\begin{itemize}
\item \texttt{config\_option} (líneas 6-19): este método permite la definición
  de nuevos atributos sobre el modelo en tiempo de ejecución por medio del uso
  de método \texttt{define\_method} que permite crear métodos en tiempo
  de ejecución, con este método y de \texttt{attr\_writer} se definen tanto
  atributos de clase como de instancia. El método toma dos parámetros
  el primero es el nombre de la configuración a crear y el segundo
  es el valor por defecto de la configuración.
\item \texttt{find} (línea 21): este método permite realizar búsquedas sobre
  el modelo por medio de un identificador, también permite realizar búsquedas
  más avanzadas por medio de parámetros como \texttt{filter}.
\item \texttt{invoke} (líneas 25-27): este método permite realizar
  un acción sobre el modelo, como se puede ver en la línea 26 se hace uso del
  método \texttt{invoke} del módulo \texttt{Bezel} haciendo
  uso del contexto del modelo, es decir, su módulo y tipo correspondiente.
\item \texttt{inherited} (líneas 29-34): este método es parte de API de
  metaprogramación de Ruby, el método es invocado cuando una subclase de
  \texttt{Bezel::Base} es creada. La subclase es pasada como el parámetro
  \texttt{klass} y por medio del método \texttt{config\_option} se definen
  tres configuraciones:
  \begin{itemize}
  \item \texttt{c3\_type} (línea 30): define el tipo asociado al modelo
    siendo el nombre de la subclase su valor por defecto.
  \item \texttt{c3\_module} (línea 31): define el módulo asociado al modelo
    siendo su valor por defecto el módulo \textquote{peat}.
  \item \texttt{c3\_include} (línea 32): define si se hace una carga adelantada
    de información de las asociaciones del modelo. El valor por defecto es que
    no se haga ninguna carga adelantada.
  \end{itemize}
  Finalmente en la línea 33 se configura que el modelo por defecto no hace uso
  de un cache.
\item \texttt{c3\_cached} (líneas 36 a 42): este método permite indicar
  que el modelo hace uso de un cache y las opciones de configuración para éste.
  Para activar el cache es necesario tener una instancia de cache definida en
  \texttt{Bezel.cache}, si esto ocurre entonces se hace un \texttt{include}
  del módulo \texttt{Bezel::CacheBase} (línea 38) que agrega el uso y
  mantenimiento del cache a las acciones básicas y de búsqueda del modelo.
\end{itemize}

\subsubsection{Métodos de instancia}

Los métodos de clase que se definen en \texttt{Bezel::Base} son para
inicializar un nuevo modelo, obtener la información sobre un atributo
del modelo y las asociaciones con otros modelos

Cabe recordar que el servicio web C3 hace uso del formato JSON en sus
respuestas a las peticiones al servicio. El formato JSON realiza una serialización
de objetos, arreglos, números, cadenas, booleanos, y \texttt{null} con base en la
sintaxis de Javascript\footnot{La principal divergencia es en el manejo de la
codificación de las cadenas.}.

En JSON los objetos son colecciones de llave-valor en donde la llave siempre
es una cadena y el valor es alguno de los tipos básicos permitidos en JSON, esta
estructura permite una biyección directa entre el formato JSON y tablas \textit{hash}
en Ruby.

\lstinputlisting[language=Ruby]{code/base-json.rb}

En el código anterior tenemos una cadena con un objeto definido según el formato
JSON en la línea 1. Haciendo uso de la biblioteca JSON en la línea 2 se parsea
esta cadena y nos regresa una tabla \textit{hash} la cual contiene la información
definida en la cadena JSON original.

En Bezel se tiene que los atributos de los modelos obtenidos por medio del servicio
web C3 son almacenados en una tabla \textit{hash} interna, permitiendo el acceso
a la información de una forma más orientada a objetos por medio de atributos
y métodos de acceso.

\lstinputlisting[language=Ruby]{code/base-instance-methods.rb}

En la mayoría de los lenguajes de programación orientado a objetos se lanza
una excepción cuando se llama un método que no esta implementado por la clase
del objeto o en la jerarquía de clases del objeto. En el caso de Ruby se lanza
la excepción \texttt{NoMethodError} para indicar que no se encontró la
implementación del método invocado sobre un objeto, sin embargo Ruby permite al
programador manejar esta excepción por medio del método \texttt{method\_missing}
permitiendo que un objeto pueda responder a una conjunto de metodos en forma
dinámica.
El método \texttt{method\_missing} es un método \texttt{callback} que recibe
como argumentos el nombre del método sin implementar en el objeto y los
argumentos originales con los que se llamo el método sin implementar.

Implementando el método \texttt{method\_missing} se permite
que los modelos definidos usando \texttt{Bezel::Base} puedan responder
de forma dinámicamente a mensajes para acceder a los atributos del modelo.



Para dar acceso a los atributos del modelo se hace uso del método
\texttt{method\_missing}, este método es llamado como ultimo paso cuando no se
encontro el metodo cuando se ejecuta un
método sobre un objeto el cual no sabe manejar. El método recibe dos parámetros
un símbolo del método invocado y los argumentos con los que se invoco al método.
Así usando \texttt{method\_missing} se permite el manejo dinámico de métodos
para acceder a los atributos del modelo.

Los métodos principales son:
\begin{itemize}
\item \texttt{initialize} (líneas 8-12): este método es el constructor por defecto
  en Ruby es invocado cuando se crea objeto nuevo. Los datos del modelo son obtenidos
  por medio del parámetro \texttt{attributes} que es una tabla hash.
\item \texttt{load} (línea 14): este método permite el cargar en un objeto ya
  existente nuevos valores de sus atributos.
\item \texttt{method\_missing} (líneas 18-27): como se explico se hace uso de este
  \textit{callback} para permitir el acceso de los atributos del modelo en forma dinámica sin
  tener que definir explícitamente métodos de acceso para cada atributo del modelo.
  En la línea 28 se trata de ver si el nombre del método es una llave en la tabla
  hash de los atributos si es así se regresa el valor contenido en la tabla.
  En la línea 30 se hace el mismo tipo de búsqueda pero se transforma el
  nombre a tu estilo de capitalización medial (\textit{camelCase}).
  Si ninguna llave corresponde se regresa \texttt{nil}.
\item \texttt{id} (líneas 36-38): como se puede observar el método regresa
  el valor asociado a la llave \texttt{id} en la tabla de atributos del modelo
  este método de acceso sale sobrando dada la implementación utilizada
  en \texttt{method\_missing} pero esta solución tiene sus costos en rendimiento
  puesto que la invocación a llamada es mucho mas eficiente que pasar por toda
  la búsqueda de un método inexistente hasta llegar a la invocación del método
  \texttt{method\_missing}. En la implementación total de \texttt{method\_missing}
  se hace uso de \texttt{define\_method} para crear un método de acceso
  de forma dinámica para hacer mas eficiente el acceso posterior al atributo.
  Así para varios atributos básicos como \texttt{id} se define una implementación
  estática.
\end{itemize}

\subsubsection{\texttt{Bezel::Associations}}

En el módulo \texttt{Bezel::Associations} se definen métodos para definir
asociaciones entre modelos basados en \texttt{Bezel::Base}. Este módulo
es agregado por medio \texttt{extend} por lo que son métodos de clase

\lstinputlisting[language=Ruby]{code/base-associations-example.rb}

En el código anterior se tiene la definición del modelo
\texttt{EnergyConservationOption} el cual representa una característica del perfil
de un edificio que esta ligada a una recomendación. Para este modelo se definen
dos asociaciones por medio del uso de los métodos \texttt{has\_one} y
\texttt{has\_many}. Por medio de \texttt{has\_one} (línea 5) se define una relación
uno a uno con el modelo \texttt{ProductCategory} que representa una categoría de un
catalogo de productos. Mientras que con el metodo \texttt{has\_many} (línea 6) se
define una relación uno a muchos con el modelo \texttt{RecommendationFAQ}.

En las líneas 10 y 11 se tiene el uso de los métodos \texttt{productCategory} y
\texttt{faqs}, los cuales regresan los modelos asociados al modelo, son creados
dinámicamente por los métodos \texttt{has\_one} y \texttt{has\_many}

\lstinputlisting[language=Ruby]{code/associations.rb}

En la implementación de ambos métodos se hace uso de dos métodos auxiliares
\texttt{get\_model\_class} y \texttt{define\_assoc\_method}.
En las líneas 7 y 11 se hace uso de \texttt{get\_model\_class} para obtener
la clase del modelo con la que se esta haciendo la asociación. Luego en las
líneas 8 y 13 se hace uso del método \texttt{define\_assoc\_method} para
definir dinámicamente un método de instancia que realiza la recuperación
y búsqueda de los modelos asociados.

En las líneas 18-22 se tiene la implementación del método \texttt{get\_model\_class}
donde se hace uso del método \texttt{constantize} que realiza la recuperación
de una constante a partir de una cadena dada, en este caso para recuperar de
una cadena una clase.

En las líneas 24-49 se tiene la implementación del método
\texttt{define\_assoc\_method} el cual hace uso del método \texttt{define\_method}
para construir un método de instancia para realizar la búsqueda de los objetos
asociados al modelo.

\subsubsection{\texttt{Bezel::Connections}}

En el módulo \texttt{Bezel::Connections} se definen los métodos que permiten
realizar las acciones básicas sobre el modelo. La implementación de este módulo
sigue un patrón bastante común en el cual se definen métodos tanto de clase
como de instancia en un mismo módulo.

\lstinputlisting[language=Ruby]{code/connections.rb}

En la línea 2 se tiene la implementación del método \texttt{included} que es un
\textit{callback} que es invocado cuando el módulo \texttt{Connections} es
utilizado en un \texttt{include}. En la línea 3 al incluirse el módulo en una
clase o módulo \texttt{base} se agregan como métodos de clase los métodos
definidos en el submódulo \texttt{ClassMethods}.

Los metodos de clase mas importante son:
\begin{itemize}
\item \texttt{all} (líneas 7-16): este método realiza una búsqueda dentro del modelo.
  En la línea 8 se toma en cuenta la configuración \texttt{c3\_include} para indicar
  que modelos asociados se deben precargar en la búsqueda. En la línea 9 se realiza
  la petición usando la acción \texttt{fetch} la cual permite hacer búsquedas
  en el servicio web C3. En las líneas 11 a 15 se tiene la lógica necesaria para
  manejar el resultado de la petición del servicio web C3, cabe señalar que al
  obtenerse un arreglo de objetos de tipo hash como resultado de la petición
  realizada al servicio se crea un nuevo modelo por objeto (línea 14) para regresar
  un arreglo de modelos como espera el programador.
\item \texttt{one} (líneas 18-22): este método realiza una búsqueda la cual
  regresa solamente el primer objeto que cumpla el criterio de búsqueda. En la
  peticion al servicio web se hace uso de la acción \texttt{fetch} (línea 20).
\item \texttt{upsert} (líneas 24-28): este método realiza la creación o actualización
  de un modelo dado los atributos del modelo en un hash. La acción
  \texttt{upsert} es usada para realizar la petición al servicio web (línea 26).
\end{itemize}

Aunque el método de clase \texttt{upsert} permite la creación y actualización
de modelos su uso se definen dos métodos de instancia para crear y actualizar
de forma mas ortodoxa un modelo.

\begin{itemize}
\item \texttt{save}: este método de instancia hace uso de \texttt{upsert}
  y de la representación hash del modelo (\texttt{to\_hash}) para
  guardar el modelo en el servicio web C3.
\item \texttt{update}: este método realiza la actualización del modelo
  siendo mas eficiente en tiempo de ejecución puesto que solo se actualizan
  los atributos que serán cambiados.
\item \texttt{destroy}: este método elimina el modelo por medio de la acción
  \texttt{remove} (línea 44).
\end{itemize}

\subsubsection{\texttt{Bezel::CacheBase}}

El módulo \texttt{Bezel::CacheBase} mejora los métodos definidos en
\texttt{Connections} para que las acciones básicas y/o búsqueda hagan uso de un
cache \texttt{Bezel::Cache}.

El módulo sigue el mismo patrón de agrupar los métodos de clase en el submódulo
\texttt{ClassMethods} y los métodos de instancia en el módulo principal, haciendo
uso del método \textit{callback} \texttt{include} para agregar los métodos de clase.

\lstinputlisting[language=Ruby]{code/cache_base.rb}
