\section{Biblioteca Bezel}

Para realizar la integración entre el sistema \texttt{PEAT} y el servicio web C3,
el API proporcionado por el servidor C3, se implemento una biblioteca con el nombre
de \texttt{Bezel}.

\texttt{PEAT} hace uso del marco de trabajo Rails que a su vez hace
uso del patrón de arquitectura MVC. En Rails los modelos por lo general hacen uso
de la biblioteca \texttt{ActiveRecord} la cual permite implementar modelos basados
en tablas de bases de datos relacionales. En \texttt{PEAT} los datos
son proporcionados no directamente por una base de datos relacional sino por
el servicio web C3 proporcionado por el servidor C3.

Dentro del marco de trabajo Rails también se tiene la biblioteca
\texttt{ActiveResource} la cual implementa el caso de uso en que los datos son
proporcionados por un servicio web tipo RESTful. En un primer acercamiento se trato
de realizar la integración de \texttt{PEAT} con el servicio web C3 haciendo uso de
esta biblioteca pero se tenían las siguientes dificultades:

\begin{itemize}
\item Presupone que el servicio web hace uso de todos los métodos HTTP (GET, POST,
  UPDATE, DELETE) para definir las acciones básicas sobre un recurso.
\item No permite la definición de asociaciones entre recursos.
\end{itemize}

Aunque el servicio web C3 es de tipo RESTful este no hace uso de los métodos
HTTP para realizar acciones sobre los recursos del servicio, en el servicio web
C3 todas las acciones que se quieren realizar se realizan usando el método HTTP
POST y se manda la acción a realizar como un parámetro dentro de la petición HTTP.

La otra dificultad es que teníamos un gran numero de asociaciones entre los
recursos definidos por el servicio web C3 pero la biblioteca \texttt{ActiveResource}
no proporcionaba ninguna ayuda para implementar estas asociaciones.

Dada las dificultades descritas hacer uso de \texttt{ActiveResource} no era una
solución viable pero sirvió como inspiración para implementar la biblioteca
\texttt{Bezel} para proporcionar acceso al servicio web C3.

El servidor C3 cabe recordar esta implementado en dos lenguajes de programación, la
gran mayoría implementado con Javascript mientras que el núcleo y funciones criticas
están implementadas en Java. El servidor C3 puede mandar tanto XML como JSON como
respuesta a peticiones realizadas al servidor, pero la representación en JSON es la
mas usada dado que el sistema de tipos usando por el sistema esta definido en
Javascript.

\texttt{Bezel} permite la reconstrucción automática de los tipos definidos por el
servidor C3 en Javascript a modelos compatibles con el marco de trabajo Rails
y el lenguaje de programación Ruby. Esta reconstrucción automática permite
evitar desfases entre la definición de los tipos entre los dos sistemas y
agilizo el desarrollo del sistema \texttt{PEAT}. Esta reconstrucción automática
se hace por medio de un lenguaje de dominio especifico que permite implementar
modelos basados en tipos del servicio web C3 aprovechando que este servicio
es tipo RESTful y las capacidades de metaprogramación de Ruby.

\texttt{Bezel} tiene dos modos principales de uso:
\begin{itemize}
\item Como un cliente REST con ciertas adecuaciones para una mejor
  integración con el API del servidor C3.
\item Como un lenguaje de dominio especifico interno para Ruby para la
  implementación automatizada de tipos y sus asociaciones del servicio web C3.
\end{itemize}

\subsection{Servicio web C3}

El servicio web C3 es de tipo RESTful es decir cada recurso o tipo tiene una única
dirección URL asociada. Las direcciones URL tienen la siguiente forma:

\vspace{2.5mm}
\texttt{POST} /api/$<$\textit{version}$>$/$<$\textit{tenant}$>$/$<$\textit{module}$>$;$<$\textit{tag}$>$/$<$\textit{type}$>$?action=$<$\textit{action-name}$>$

\begin{itemize}
\item \textit{version}: indica la versión del servicio a la cual se esta accediendo,
  el sistema \texttt{PEAT} hace uso de la versión 1.0.
\item \textit{tenant}: indica el producto o cliente, esto es porque cada
  producto y/o cliente cierta funcionalidad especifica. Para \texttt{PEAT} se tiene
  \textquote{peat} y \textquote{c3}.
\item \textit{module}: indica el módulo al que se esta accediendo, un conjunto
  de tipos se definen en un módulo en particular, para el sistema \texttt{PEAT} el
  módulo mas usado es el módulo \textquote{peat} pero también se hace uso
  de los módulos \textquote{billing} y \textquote{structure}.
\item \textit{tag}: indica el tipo de ambiente, es decir si el servicio web es para
  ambiente de pruebas, control de calidad, producción, etcétera.
  \textquote{peatprod} y \textquote{peatqa} son algunas de las \textit{tags}
  que se usaron para \texttt{PEAT}.
\item \textit{type}: indica el tipo al que se quiere acceder, \textquote{Building},
  \textquote{Location} y \textquote{BuildingEnergyConservationOption} son algunos
  de los tipos que se usaron para \texttt{PEAT}.
\item \textit{action-name}: indica la acción que se quiere realizar sobre
  el tipo indicado anteriormente. \textquote{fetch}, \textquote{upsert} y
  \textquote{remove} son unas de las acciones que se pueden realizar.
\end{itemize}

Como se comento anteriormente el servicio web C3 no es totalmente de tipo RESTful
principalmente porque se hace uso del mismo método HTTP POST para todas las acciones,
haciendo uso del \textquote{action-name} para ejecutar la acción correspondiente.
Para la implementación de \texttt{Bezel} solamente se espera que el servicio web
tenga direcciones únicas para cada tipo definido.

Muchas acciones aceptan parámetros los cuales son almacenados dentro del cuerpo
de la petición POST, por lo que el único parámetro que se manda por medio
de la URL es el nombre la acción a realizar.

\subsection{Arquitectura General}

La biblioteca \texttt{Bezel} esta conformada por un cinco clases y cinco módulos
siendo las clases principales \texttt{Bezel::Client} y \texttt{Bezel::Base}.

En el lenguaje de programación Ruby se tiene el concepto de módulo (\texttt{module}),
el cual no es mas que una agrupación de objetos bajo un único nombre. Los objetos
pueden ser constantes, métodos, clases y otros módulos.

Los módulos tienen dos usos: se pueden utilizar un módulo como una manera conveniente
de agrupar objetos o se puede incorporar los objetos del módulo a una clase haciendo
uso de \texttt{include} o \texttt{extend}.

\lstinputlisting[language=Ruby]{code/module-namespace.rb}

El ejemplo anterior muestra el uso de los módulos como un espacio de nombre,
en la linea 2 se define la constante \texttt{PI} y en las lineas 3 y 7 se definen
los métodos \texttt{sin} y \texttt{cos}, estos métodos son usados fuera del
módulo por medio del nombre del módulo como ocurre en la linea 12, en esta
linea se hace referencia tanto a la constante \texttt{PI} como al método
\texttt{sin}.

\lstinputlisting[language=Ruby]{code/module-include.rb}

Ahora en este ejemplo se muestra el uso de los módulos para incorporar métodos
de clase o de instancia. En las lineas 1-11 se define el módulo \texttt{Boolean}
el cual define dos métodos \texttt{boolean?} y \texttt{to\_bool}, mientras que
en las lineas 13-17 se define el módulo \texttt{RandomString} que define un método
\texttt{random}. En las lineas 19-22 no se esta definiendo una nueva clase
\texttt{String} si no que Ruby permite modificar cualquier clase en tiempo de
ejecución aunque esta clase sea parte de la biblioteca estándar, así en la linea 20
se hace uso de \texttt{include} para agregar los métodos del módulo dado como
parámetro como métodos de instancia, el uso de estos métodos se como se ver en las
lineas 24-25. Finalmente en la linea 21 se hace uso de la declaración \texttt{extend}
para agregar los métodos del módulo dado como métodos de clase como se puede
ver en la linea 26.

Por lo anterior se tiene que el uso de módulos es una parte importante para
organizar y compartir funcionalidad en el lenguaje de programación Ruby.
Para la biblioteca \texttt{Bezel} tenemos el módulo \texttt{Bezel} que contiene
a todas las clases y módulos de la biblioteca. Además se tienen cinco submódulos
que permiten organizar de forma coherente la funcionalidad de la biblioteca.

Las clases y módulos que conforman la biblioteca son
(Ver Figura \ref{fig:diagrama-clase}):
\begin{itemize}
\item \texttt{Bezel}: módulo principal que contiene a los demás submódulos y
  clases de la biblioteca.
\item \texttt{Bezel::Config}: módulo que define los métodos y valores validos
  para configurar el cliente con el servicio web C3.
\item \texttt{Bezel::Client}: clase que define un cliente para usar el servicio
  web C3.
\item \texttt{Bezel::Target}: clase que representa un tipo dado un contexto
  especifico (\textit{tenant}, \textit{tag}, \textit{module}).
\item \texttt{Bezel::Action}: clase que representa una acción sobre
  un tipo representado por un \texttt{Bezel::Target}.
\item \texttt{Bezel::Base}: clase abstracta que permite definir un modelo en Ruby
  en base a un tipo del servido web C3.
\item \texttt{Bezel::Connections}: módulo que define las acciones básicas y de
  búsqueda sobre un modelo.
\item \texttt{Bezel::Associations}: módulo que define las asociaciones entre
  modelos.
\item \texttt{Bezel::CacheBase}: clase que redefine las acciones básicas y
  de búsqueda sobre un modelo haciendo uso de un cache \texttt{Bezel::Cache}.
\item \texttt{Bezel::Cache}: clase que define un API para realizar operaciones
  de lectura/escritura sobre un cache.
\end{itemize}

\rjcimage{1.0}{imagenes/Diagrama-Clase-Bezel.png}{Diagrama de clase de la biblioteca
  Bezel}{diagrama-clase}

\subsubsection{Dependencias}

En la implementación de \texttt{Bezel} se hace uso de las siguiente bibliotecas
de apoyo:

\begin{itemize}
\item moneta: proporciona un API uniforme para realizar
  operaciones de lectura y escritura en un repositorio de cache.
\item faraday: proporciona un API uniforme para realizar
  peticiones HTTP.
\item hashie: proporciona un conjunto de metodos que extienden la funcionalidad
  de la clase \texttt{Hash} en Ruby.
\item activemodel: proporciona interfaces para modelos que deben operar sobre
  Rails.
\end{itemize}

\subsection{\texttt{Bezel}}

El módulo \texttt{Bezel} no solo sirve como el espacio de nombres principal para la
biblioteca si no que además implementa varios métodos auxiliares para facilitar la
configuración y creación de clientes para acceder al servicio web C3.

\lstinputlisting[language=Ruby]{code/bezel.rb}

En el código anterior se tiene la implementación abreviada de \texttt{Bezel},
en la linea 2 por medio de \texttt{extend} se añaden varios métodos de clase
para configurar la biblioteca. En las lineas 4-20 se definen los
métodos auxiliares mas importantes:

\begin{itemize}
\item \texttt{new} (lineas 5-7): regresa una nueva instancia \texttt{Bezel::Client}.
\item \texttt{invoke} (lineas 9-11): realiza una petición al servicio web C3
  por medio del cliente asociado al módulo \texttt{Bezel}.
\item \texttt{client} (lineas 13-15): usando una variable de clase se inicializa o
  reusa una instancia \texttt{Bezel::Client}. Este método permite reutilizar un
  mismo cliente y conexión al servidor C3 evitando el costo de establecer un
  nuevo cliente y en establecer una conexión HTTP con el servidor C3.
\item \texttt{context} (lineas 17-19): permite cambiar el contexto de las
  peticiones que se realizan dentro de un bloque, se regresa al contexto original
  al finalizar de ejecutar el código dentro del bloque.
\end{itemize}

\subsubsection{\texttt{Bezel::Config}}

El módulo \texttt{Bezel::Config} define los atributos que permiten configurar
el comportamiento de la biblioteca y de los clientes que se obtienen por medio
de ella. Este módulo se usa en \texttt{Bezel} para definir una configuración global
de la biblioteca como en \texttt{Bezel::Client} para tener también una configuración
local por cliente.

\lstinputlisting[language=Ruby]{code/config.rb}

En el módulo \texttt{Config} se definen dos constantes:

\begin{itemize}
\item \texttt{VALID\_OPTIONS\_KEYS} (lineas 4-13): un arreglo de símbolos que indica
  que atributos son validos.
\item \texttt{DEFAULT\_VALUES} (lineas 15-24): una tabla hash donde se indican los
  valores por defecto de los atributos validos.
\end{itemize}

Haciendo uso del método \texttt{attr\_accessor} en la linea 26 se definen por cada
símbolo en \texttt{VALID\_OPTIONS\_KEYS} una atributo de instancia con sus
correspondientes métodos de acceso y escritura. El método \texttt{attr\_accessor} es
un ejemplo del uso de metaprogramación en Ruby, dado que en base al símbolo dado se
genera un atributo y sus métodos de acceso.

% FIXME: Buscar traducción para callback

En la linea 30 se tiene el método \texttt{extended} el cual es un callback que es
llamado cuando un módulo es extendido, es decir es utilizado como parámetro al
método \texttt{extend}. Para el módulo \texttt{Config} se tiene que cuando el módulo
es extendido, como es en el caso de \texttt{Config} dentro del módulo \texttt{Bezel},
se llama al método \texttt{reset} para inicializar todos los atributos definidos en
el módulo con sus valores por defecto. El método \texttt{extended} es solo uno
de varios métodos callback que proporciona Ruby para la metaprogramación.

Otro método muy importante para la metaprogramación es \texttt{send} el cual es usado
en la linea 35 para inicializar los atributos definidos por \texttt{Config}, el
método \texttt{send} invoca el método identificado con un símbolo o cadena dada,
pasando los argumentos posteriores como argumentos a la invocación del método.
Haciendo uso de \texttt{send} y de las dos constantes definidas en el modulo
se puede inicializar cada uno de los atributos.

\subsection{\texttt{Bezel::Client}}

La clase \texttt{Client} representa un cliente que permite realizar peticiones al 
servicio web C3, por medio del módulo \texttt{Config} se permite configurar
cada cliente instanciado.

\lstinputlisting[language=Ruby]{code/client.rb}

En la linea 3 se tiene un \texttt{include} que añade las constantes
y atributos para configurar al cliente. En las lineas 7-13 se define el método
\texttt{conn} que inicializa una conexión HTTP según la configuración definida
para el cliente y pidiendo que las respuestas mandadas por el servidor
sean de tipo JSON. 

En las lineas 13-49 se tiene la implementación del metodo \texttt{invoke}, en 
la linea 14 se define un \textit{target} de tipo \texttt{Bezel::Target} usando 
tanto el contexto del cliente, es decir el \textit{tenant} y \textit{tag} mas 
la información del \textit{module} y \textit{type} dados en los argumentos.
En la linea 15 se define una acción de tipo \texttt{Bezel::Action} que representa
una acción sobre un tipo con sus parámetros.

En la linea 17 se define la URL a la cual se hará la petición al servicio web C3,
haciendo uso de la información de los objetos previos, se convierte los parámetros
de la acción al formato JSON para ser mandados en el cuerpo de la petición POST
al servicio web.

% FIXME: explicar linea 32 de addCredentials

En las lineas 22 a 30 se definen los encabezados y cuerpo de la petición POST al
servicio web. Finalmente en la linea 37 se realiza la petición POST por medio
de la conexión creada por el método \texttt{conn} y con los para metros definidos
anteriormente.

Posteriormente el resultado de la petición es almacenado en el objeto \texttt{action}
y este objeto es regresado como resultado de la invocación del método
\texttt{invoke}.

\subsection{\texttt{Bezel::Base}}

\texttt{Bezel::Base} es una clase abstracta la cual define la mayor parte de su
funcionalidad en base al nombre de la clase que hereda de ella. Es decir no se hace
uso directo de \texttt{Bezel::Base} si no que al heredar de esta clase abstracta
se obtiene funcionalidad para crear modelos basados en tipos provenientes
del servicio web C3.

\lstinputlisting[language=Ruby]{code/base-example.rb}

El código anterior ejemplifica el uso de \texttt{Bezel::Base} para definir el modelo
por medio de herencia, en la linea 1 se define \texttt{Recommendation}. En la linea 2
se tiene el método de clase \texttt{set\_c3\_type} que permite al modelo definir
a que tipo del sistema web C3 esta ligado,  en este caso el modelo esta ligado al
tipo \texttt{BuildingEnergyConservationOption}. En la linea 4 se define que el modelo
tiene una asociación con el modelo \texttt{EnergyConservationOption} en forma única
por medio del metodo \texttt{has\_one}. Ya definido el modelo en la linea 7 se 
hace la búsqueda de una recomendación por medio de su identificador único por medio
del método \texttt{find}. En la linea 8 se obtiene el costo total de la recomendación
buscada anteriormente por medio de \texttt{totalCost}. Los métodos
\texttt{set\_c3\_type}, \texttt{has\_one} y \texttt{find} son definidos por
\texttt{Bezel::Base} pero el método \texttt{totalCost} igual es definido en tiempo
de ejecución por medio de la metaprogramación.

% \lstinputlisting[language=Ruby]{code/base-extend-include.rb}

\subsubsection{Métodos de clase}

Los métodos de clase que se definen en \texttt{Bezel::Base} son para
definir métodos de configuración del modelo y para realizar operaciones
de búsqueda sobre el modelo.

\lstinputlisting[language=Ruby]{code/base-class-methods.rb}

Los métodos principales son:
\begin{itemize}
  \item \texttt{config\_option} (lineas 6-19): este método permite la definición
    de nuevos atributos sobre el modelo en tiempo de ejecución por medio del uso
    de método \texttt{define\_method} que permite crear métodos en tiempo
    de ejecución, con este método y de \texttt{attr\_writer} se definen tanto
    atributos de clase como de instancia. El método toma dos parámetros
    el primero es el nombre de la configuración a crear y el segundo
    es el valor por defecto de la configuración.
  \item \texttt{find} (linea 21): este método permite realizar búsquedas sobre
    el modelo por medio de un identificador, también permite realizar búsquedas
    mas avanzadas por medio de parámetros como \texttt{filter}.
  \item \texttt{invoke} (lineas 25-27): este método permite realizar
    un acción sobre el modelo, como se puede ver en la linea 26 se hace uso del
    metodo \texttt{invoke} del modulo \texttt{Bezel} haciendo
    uso del contexto del modelo, es decir su modulo y tipo correspondiente.
  \item \texttt{inherited} (lineas 29-34): este método es parte de API de
    metaprogramacion de Ruby, el metodo es invocado cuando una subclase de
    \texttt{Bezel::Base} es creada. La subclase es pasada como el parametro
    \texttt{klass} y por medio del metodo \texttt{config\_option} se definen
    tres configuraciones:
    \begin{itemize}
    \item \texttt{c3\_type} (linea 30): define el tipo asociado al modelo
      siendo el nombre de la subclase su valor por defecto.
    \item \texttt{c3\_module} (linea 31): define el modulo asociado al modelo
      siendo su valor por defecto el modulo \textquote{peat}.
    \item \texttt{c3\_include} (linea 32): define si se hace una carga adelantada
      de información de las asociaciones del modelo. El valor por defecto es que
      no se haga ninguna carga adelantada.
    \end{itemize}
    Finalmente en la linea 33 se configura que el modelo por defecto no hace uso
    de un cache.
    % FIXME: Hablar sobre los tipos de cache
  \item \texttt{c3\_cached} (lineas 36 a 42): este método permite indicar
    que el modelo hace uso de un cache y las opciones de configuración para este.
    Para activar el cache es necesario tener una instancia de cache definida en
    \texttt{Bezel.cache}, si esto ocurre entonces se hace un \texttt{include}
    del modulo \texttt{Bezel::CacheBase} (linea 38) que agrega el uso y
    mantenimiento del cache a las acciones básicas y de búsqueda del modelo.
\end{itemize}

\subsubsection{Métodos de instancia}

Los métodos de clase que se definen en \texttt{Bezel::Base} son para
inicializar un nuevo modelo, obtener la información sobre un atributo
del modelo y las asociaciones con otros modelos

Cabe recordar recordar que el servicio web C3 hace uso del formato JSON en sus
respuestas a las peticiones al servicio. El formato JSON es una serializar objetos,
arreglos, números, cadenas, booleanos, y \texttt{null}. Se basa en la sintaxis de
JavaScript, pero es distinta de ella en el manejo de Unicode en las cadenas.
Los objetos en JSON son colecciones de llave-valor en donde la llave siempre
es una cadena y el valor es alguno de los tipos básicos permitidos en JSON, dado
esta estructura una biyección entre el formato JSON y las tablas hash en Ruby
es directa.

\lstinputlisting[language=Ruby]{code/base-json.rb}

En el código anterior tenemos una cadena con un objeto definido según el formato
JSON en la linea 1. Haciendo uso de la biblioteca JSON en la linea 2 se parsea
esta cadena y nos regresa una tabla hash la cual contiene la información definida
en la cadena JSON original.

En Bezel se tiene que los atributos de los modelos obtenidos por medio del servicio
web C3 son almacenados en una tabla hash interna pero permitiendo el acceso a la
información de una forma mas orientada a objetos por medio de atributos y métodos
de acceso.

\lstinputlisting[language=Ruby]{code/base-instance-methods.rb}

Los métodos principales son:
\begin{itemize}
\item \texttt{initialize}: este método es el constructor por defecto en Ruby
  es invocado cuando se crea objeto nuevo. Los datos del modelo son obtenidos
  por medio del parametro \texttt{attributes} que es una tabla hash.
\item \texttt{load}: este método permite el cargar en un objeto ya existente
  nuevos valores de sus atributos.
\item \texttt{method\_missing}: este metodo es parte del API de metaprogramacion
  de Ruby, el cual es invocado cuando se invoca un metodo que un objeto no sabe
  manejar. El metodo recibe dos parametros \texttt{name} que es simbolo del
  metodo invocado, y \texttt{args} que es un arreglo con todos los parametros
  con los que se invoco al metodo.
\end{itemize}

%\subsubsection{\texttt{Bezel::Associations}}

%\lstinputlisting[language=Ruby]{code/associations.rb}

%\subsubsection{\texttt{Bezel::Connections}}

%\lstinputlisting[language=Ruby]{code/connections.rb}

%\subsubsection{\texttt{Bezel::CacheBase}}

%\lstinputlisting[language=Ruby]{code/cache_base.rb}
