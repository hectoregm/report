\chapter{Situación actual}

\section{Contexto}
De 2000 a 2002 se tuvo una crisis energética en el estado de California en los
Estados Unidos, esta crisis provoco que el gobierno del estado empezara a
tomar decisiones de largo plazo en su estrategia energética. En los años
subsecuentes se aprobó nuevas leyes para incentivar el uso de energía renovable,
la producción de energía, etcétera.

Para el 2010 se aprobó una nueva legislación que permitía un nuevo esquema
de cobro a las empresas generadoras de electricidad. Este nuevo esquema
llamado \textit{hora de consumo} (\textit{time-of-use} TOU) define que las
tasas de cobro varían en función de la temporada y la hora del día en que
se usa la energía eléctrica, en contraste con las tasas fijas tradicionales.

En un esquema \texttt{TOU} se definen tres tarifas y sus zonas de tiempo asociadas:

\begin{itemize}
\item  Demanda baja: es la tarifa mas baja, se aplica durante las horas de la
  mañana, en la noche y los fines de semana.
\item Demanda mediana: es la tarifa intermedia, se aplica durante las horas de
  10 am a 1pm y de 7 pm a 9 pm.
\item Demanda Alta: es la tarifa mas cara, se aplica durante las horas de
  1 pm a 7 pm.
\end{itemize}

El objetivo de un esquema \texttt{TOU} es el desplazar la carga del sistema
eléctrico, alentando a los usuarios a cambiar su demanda durante periodos
de alta demanda a periodos de menor demanda, el bajar la demanda en los periodos
de mas alta demanda provoca que se reduzcan los costos de generación de energía.

Un esquema \texttt{TOU} es solo posible con la llegada de los medidores
inteligentes, a diferencia de los antiguos medidores analógicos los medidores
inteligentes son capaces de medir el consumo de electricidad de forma instantánea,
de distinguir y facturar el consumo de electricidad según el momento en que se
esta realizando el consumo. Desde inicios del 2010 PG\&E empezó el despliegue
de medidores inteligentes, por lo que para inicios del 2012 ya casi había
completado la transición de los medidores.

Como fecha de arranque del nuevo esquema se puso el mes de noviembre del 2012,
las empresas como PG\&E podían iniciar la transición a un esquema \texttt{TOU}
de sus usuarios PyMES. Una parte vital para permitir esta transición es que
PG\&E brindara herramientas para que el usuario final pudiera analizar sus
gastos de energía y tomar decisiones sobre su consumo de energía según las
nuevas tarifas y su historial de consumo.

En este contexto es que nace el sistema \textit{Progressive
  Energy Audit Tool} \ (PEAT) para solventar la necesidad de información
del usuario final sobre sus gastos de energía en una forma detallada para
que este pueda identificar y monitorear sus gastos de energía eléctrica.

\section{Objetivo general}

El objetivo general del sistema es dar la mayor utilidad posible con la menor
información disponible, fomentando en el usuario final el compartir dicha
información sobre su empresa, obteniendo un mejor control
acerca de su consumo energético.
