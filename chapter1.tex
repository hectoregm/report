\chapter{Situación actual}

\section{Contexto}
De 2000 a 2001 se tuvo una crisis energética en el estado de California en los Estados Unidos, esta crisis provoco que el gobierno del estado tuviera una mirada de mas largo plazo en su estrategia energetica. En los años subsecuentes se impulso legislacion para incentivar el uso de energia renovable, la producion de energia, etctera.

Para el 2010 se aprobo una nueva legislacion que permitia un nuevo esquema de cobro a las empresas generadoras de electricidas. Este nuevo esquema llamado \textit{tiempo de uso} (\textit{time-of-use} TOU) define que las tasas de cobro varian en funcion de la temporada y la hora del dia en que se usa la energia electrica. Como fecha de arranque de este nuevo esquema se puso el mes de noviembre del 2012, las utilidades como PG\&E podian transicionar a sus usuarios de pequeñas y medianas empresas (PyMES) a esquemas TOU, aunque una parte vital para permitir esta transicion es que PG\&E brindara herramientas para que el usuario final pudiera analizar sus gastos de energía.

En este contexto es que nace el sistema \textit{Progressive Energy Audit Tool}\ (PEAT) para solventar la necesidad de información del usuario final sobre sus gastos de energia en una forma detallada para que este pueda identificar y monitorear sus gastos de energia electrica.

Desde inicios del 2010 PG\&E había realizado el despliegue de medidores inteligentes en la mayor parte de sus clientes, logrando que para inicios del 2012 ya tuviera un grupo de usuarios con un historial de consumo de por lo menos un año.

\section{Objetivo general}

El objetivo general del sistema es dar la mayor utilidad posible con la menor
información disponible, fomentando en el usuario final el compartir dicha
información sobre su empresa, obteniendo un mejor control
acerca de su consumo energético.
