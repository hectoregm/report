\chapter{Situación actual}

\section{Contexto}
De 2000 a 2002 se tuvo una crisis energética en el estado de California en los
Estados Unidos de América, esta crisis provocó que el gobierno del estado empezara a
tomar decisiones de largo plazo en su estrategia energética. En los años
subsecuentes se aprobaron nuevas leyes para incentivar el uso de energía renovable,
la producción de energía, etcétera.

Para el 2010 se aprobó una nueva legislación que permitía un nuevo esquema
de cobro a las empresas generadoras de electricidad. Este nuevo esquema
llamado \textit{hora de consumo} (\textit{time-of-use} TOU) define que las
tasas de cobro varíen en función de la temporada y la hora del día en que
se usa la energía eléctrica, en contraste con las tasas fijas tradicionales.

En un esquema \texttt{TOU} se definen tres tarifas y sus zonas de tiempo asociadas:

\begin{itemize}
\item  Demanda baja: es la tarifa mas baja, se aplica durante las horas de la
  mañana, en la noche y los fines de semana.
\item Demanda mediana: es la tarifa intermedia, se aplica durante las horas de
  10 am a 1pm y de 7 pm a 9 pm.
\item Demanda Alta: es la tarifa mas cara, se aplica durante las horas de
  1 pm a 7 pm.
\end{itemize}

El objetivo de un esquema \texttt{TOU} es el distribuir la carga del sistema
eléctrico, alentando a los usuarios a cambiar su consumo durante periodos
de demanda alta a periodos de demanda baja, al bajar la demanda en los periodos
de demanda alta provoca que se reduzcan los costos de generación de energía de
forma general.

Un esquema \texttt{TOU} es solo posible con la llegada de los medidores
inteligentes, a diferencia de los antiguos medidores analógicos los medidores
inteligentes son capaces de medir el consumo de electricidad de forma instantánea,
de distinguir y facturar el consumo de electricidad según el momento en que se
esta realizando el consumo. Desde inicios del 2010 Pacific Gas and
Electric Company (PG\&E) empezó la instalación
de medidores inteligentes, por lo que para inicios del 2012 ya casi había
completado la transición de los medidores en su territorio de servicio.

Como fecha de arranque de la nueva legislación se puso el mes de noviembre del 2012,
las empresas como PG\&E podían iniciar la transición a un esquema \texttt{TOU}
de sus usuarios PyMES. Una parte vital para permitir esta transición es que
PG\&E brindara herramientas para que el usuario final pudiera analizar sus
gastos de energía y tomar decisiones sobre su consumo de energía según las
nuevas tarifas y su historial de consumo.

En esta situación es que nace el sistema \textit{Progressive
  Energy Audit Tool} \ (PEAT) para solventar la necesidad de información
del usuario final sobre sus gastos de energía en una forma detallada para
que este pueda identificar y monitorear sus gastos de energía eléctrica.

\section{Objetivo general}

El objetivo general del sistema es dar la mayor utilidad posible con la menor
información disponible, fomentando en el usuario final el compartir dicha
información sobre su empresa, obteniendo un mejor control
acerca de su consumo energético.

\section{Objetivos secundarios}

Los objetivos secundarios que apoyan al objetivo general de este trabajo son:
\begin{itemize}
\item Implementación de una interfaz que permita la obtención
  de información del usuario de una forma eficaz y sencilla.
\item Autentificar a los usuarios mediante el uso de credenciales de acceso
  obtenidas en el portal web de PG\&E.
\item Dar información útil aunque el usuario solo proporcione el
  mínimo de información sobre su empresa.
\item Proporcionar recomendaciones para disminuir sus
  gastos en energía con base en el consumo e información proporcionada
  hasta el momento.
\item Soportar por lo menos a mil usuarios concurrentes.
\item Implementar la infraestructura para el despliegue continuo de la
  aplicación.
\end{itemize}

\section{Propuesta de desarrollo}
