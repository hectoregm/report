\chapter*{Introducción}
\addcontentsline{toc}{chapter}{Introducción}
Este trabajo aborda el diseño e implementación del sistema \textit{Progressive
  Energy Audit Tool}\ (\texttt{PEAT}), el cual es un sistema web que permite
a los usuarios de pequeñas y medianas empresas (PyMES) identificar y monitorear
sus gastos de energía eléctrica, con el fin de obtener un mayor control
en los costos de esta.

El sistema \texttt{PEAT} es resultado del trabajo en conjunto de tres compañías:
Pacific Gas and Electric Company (PG\&E), C3 Energy y Software Next Door.

PG\&E es una compañía proveedora de gas natural y electricidad, una de las
más grandes compañías de Estados Unidos con sede en San Francisco, California.
El sistema \texttt{PEAT} es el resultado de una licitación iniciada por PG\&E,
siendo ganadora de dicha licitación la compañía C3 Energy.

La compañía C3 Energy, en la que trabajé por medio de Software Next Door, ganó
la licitación para el desarrollo de \texttt{PEAT} ya que contaba con la
infraestructura necesaria para el procesamiento de una gran cantidad de datos
de consumo de energía, tiene un sistema de monitoreo de consumo de energía
enfocada a empresas de nivel multinacional. En este trabajo se abordan las
decisiones de diseño e implementación para pasar de un sistema y procesos
diseñados para una docena de clientes, a un sistema que diera servicio
a cientos de miles de clientes PyMES.

\section*{Objetivo general}

El objetivo general del sistema \texttt{PEAT} es dar la mayor utilidad posible
a usuarios PyMES con la menor información disponible, fomentando en el usuario
el compartir más información sobre su empresa, obteniendo un mejor control
acerca de su consumo energético.

El sistema debe además permitir el ingreso progresivo de información, por medio
de una serie de preguntas especificas al usuario, dando mejores recomendaciones
para bajar su consumo energético conforme el sistema obtiene mas información.
A partir de la información obtenida el sistema debe permitir el monitoreo
y revisión del consumo energético de forma detallada, ya sea en horas,
días, meses y años.

\section*{Objetivos secundarios}

Los objetivos secundarios que apoyan al objetivo general de este trabajo son:
\begin{itemize}
\item Implementación de una interfaz que permita la obtención
  de información del usuario de una forma eficaz y sencilla.
\item Dar información útil aunque el usuario solo proporcione el
  mínimo de información sobre su empresa.
\item Proporcionar recomendaciones para disminuir sus
  gastos en energía con base en el consumo e información proporcionada
  hasta el momento.
\item Autentificar a los usuarios mediante el uso de credenciales de acceso
  obtenidas en el portal web de PG\&E.
\item Diseñar e implementar una suite de pruebas unitarias, funcionales
  y de integración para los módulos críticos del sistema.
\item Soportar por lo menos a mil usuarios concurrentes.
\item Implementar la infraestructura para el despliegue continuo de la
  aplicación, permitiendo una retroalimentación continua sobre el
  funcionamiento del sistema.
\end{itemize}
