\chapter{Diseño e implementación del sistema PEAT}
\section{Interfase de usuario}
Dentro de los requerimientos del proyecto PEAT se tenia la implementación de una
interfaz de usuario que fuera lo suficientemente eficaz y sencilla, para así obtener
la mayor cantidad de información del usuario.

\subsection{Configuración inicial}
Para dar el mayor valor al usuario es necesario obtener al menos la siguiente información
para cada cuenta:

\begin{enumerate}
\item La industria que mejor describe el negocio del usuario.
\item El numero de edificios.
\end{enumerate}

Ademas por cada edificio era necesario obtener al menos la siguiente información:

\begin{enumerate}
\item El tamaño del edificio (área).
\item El tipo de edificio.
\item La dirección de servicio asociada al edificio.
\end{enumerate}

Obteniendo al menos esta información era factible el dar un valor al usuario desde el
principio, entonces se tenia un enfasis en la interfase de usuario en el primer ingreso
del usuario.

Se tenia un énfasis en la interfase de usuario en el primer ingreso del usuario puesto
que era critico obtener al menos la siguiente información:

\begin{usecase}
  \addtitle{Caso de Uso I}{Ingresar informacion basica inicial}
  \addfield{Actor Principal:}{Cliente de PG\&E}
  \addfield{Precondiciones:}{}
  \addfield{Postcondiciones:}{}
\end{usecase}
