\chapter{Diseño e implementación del sistema PEAT}

\section{Modelo de información}

El objetivo principal del modelo de informacion de PEAT es definir el concepto de edificio
y su relacion con el consumo energetico del usuario. Asi en PEAT tenemos el concepto de
edifico el cual es una estructura independiente 

\subsection{Estructura de facturación de PG\&E}
Un usuario puede tener una o varias cuentas, a su vez cada cuenta puede tener uno o varios
contratos de servicio. Un contrato de servicio representa los servicios que el usuario
requiere ya sea de electricidad o gas. Una cuenta también tiene uno o varios edificios/locales
Cada uno de esos locales tiene uno o varios puntos de servicio. Finalmente un punto de servicio
puede tener uno varios medidores. Un medidor es un dispositivo instalado en un punto de servicio
que registra el consumo del servicio, ya sea eléctrico o gas, por el local.

\subsection{Relación Edificio-Facturación}

\section{Interfase de usuario}
Dentro de los requerimientos del proyecto PEAT se tenia la implementación de una
interfaz de usuario que fuera lo suficientemente eficaz y sencilla, para así obtener
la mayor cantidad de información del usuario.

\subsection{Configuración inicial}

Los datos de facturación del consumo eléctrico de un cliente no puede ser analizados
para obtener el numero de edificios (y sus características) a los cuales se les
proporciona el servicio eléctrico. No se tienen métodos fiables que permitan
usar solamente los datos de facturación para obtener esta información, así es
necesario que el usuario responda una serie de preguntas básicas sobre sus edificios.

Para dar el mayor valor al usuario es necesario obtener al menos la siguiente información
para cada cuenta:

\begin{enumerate}
\item La industria que mejor describe el negocio del usuario.
\item El numero de edificios.
\end{enumerate}

Ademas por cada edificio era necesario obtener al menos la siguiente información:

\begin{enumerate}
\item El tamaño del edificio (área).
\item El tipo de edificio.
\item La dirección de servicio asociada al edificio.
\end{enumerate}

Obteniendo al menos esta información era factible el dar un valor al usuario desde el
principio, entonces se tenia un énfasis en la interfase de usuario en el primer ingreso
del usuario.

Se tenia un énfasis en la interfase de usuario en el primer ingreso del usuario puesto
que era critico obtener al menos la siguiente información:

\begin{usecase}
  \addtitle{Caso de Uso I:}{Crear perfil inicial de un edificio.}
  \addfield{Descripción:}{Un cliente PyME entra al sistema  para crear un perfil inicial
  de un edificio, respondiendo ciertas preguntas básicas sobre el edificio.}
  \addfield{Actor:}{Cliente PyME}
  \additemizedfield{Precondiciones:}{
  \item El usuario a sido autentificado por medio de un token SSO (Single Sign On)
    asignado por el sistema de PG\&E.
  \item Los datos de intervalo y de facturación del cliente esta en el sistema.
  }
  \addscenario{Flujo normal:}{
  \item El sistema checa si el cliente tiene mas de una cuenta.
  \item El sistema despliega el numero de cuenta.
  \item El usuario puede definir un alias a la cuenta.
  \item El sistema despliega las direcciones de servicio asociadas a la cuenta.
  \item El cliente debe ingresar la siguiente información sobre el edificio:
    \begin{enumerate}
    \item Tipo.
    \item Rango de tamaño.
    \item Antigüedad.
    \end{enumerate}
  \item El usuario puede ingresar los datos o cancelar.
  }
  \addfield{Postcondiciones:}{
    Se genera un identificador único para el perfil del edificio generado, el cual
    esta asociado a los datos de facturación de la cuenta del cliente PyME.
  }
\end{usecase}
