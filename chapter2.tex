\chapter{Fundamentos teóricos}

\section{Patrón Modelo-Vista-Controlador (MVC)}
El patrón Modelo Vista Controlador (\texttt{MVC}) es probablemente el patrón
mas utilizado y citado para el desarrollo de interfaces de usuario y sistemas web.
\texttt{MVC} consiste de tres tipos de objetos:

\begin{itemize}
\item Modelo: Representa la información de dominio del sistema.
\item Vista: Es una representación visual del modelo.
\item Controlador: Define la forma en que la interfaz reacciona a la entrada
  del usuario.
\end{itemize}

\jcimage{1.0}{imagenes/MVC-Rails.png}{Patrón MVC en Rails.}

Su uso extensivo se da porque permite una clara división de responsabilidades
entre la presentacion (vista y controlador) y el dominio (modelo).
\texttt{MVC} desacopla vistas y modelos mediante el establecimiento de un
protocolo de suscribirse / notificar entre ellos. La vista debe asegurarse
de que su aspecto visual refleje el estado del modelo. Cada vez que cambian
los datos del modelo, el modelo notifica a las vistas que depende de ella.
Este enfoque permite conectar múltiples vistas a un modelo para proporcionar
diferentes presentaciones. También puede crear nuevas vistas para un modelo
sin reescribir este ultimo.

En el framework Rails se hace uso de \texttt{MVC} como patrón base para
construir sistemas. El modelo sigue siendo el responsable de almacenar
el estado del sistema, en el caso general esto se almacena fuera del sistema
casi siempre en una base de datos, en \texttt{PEAT} se hace uso de
\texttt{PostgreSQLQ} temporalmente mientras se hacia la implementacion
de los servicios web faltantes, pero en la integracion final los modelos
son objetos obtenidos por medio de los servicios web. Al hacer uso de \texttt{MVC}
hacer la sustitucion de modelos 
Las vistas siguen siendo representaciones del 


Aunque temporalmente se hace uso de una base de datos para almacenar
el estado de los modelos del sistema, en la integracion final los modelos
son objetos obtenidos por los servicios web del \textit{backend} por
medio de la biblioteca \texttt{Bezel}.
