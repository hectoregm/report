\chapter{Fundamentos teóricos}

\section{Patrón Modelo-Vista-Controlador (MVC)}
El patrón Modelo Vista Controlador (\texttt{MVC}) es probablemente el patrón
mas utilizado y citado para el desarrollo de interfaces de usuario y sistemas web.
\texttt{MVC} consiste de tres tipos de objetos:

\begin{itemize}
\item Modelo: Representación de la información de dominio del sistema.
\item Vista: Representación visual del modelo.
\item Controlador: Define la forma en que la interfaz reacciona a la entrada
  del usuario.
\end{itemize}

\jcimage{1.0}{imagenes/MVC-Rails.png}{Patrón MVC en Rails.}

Su uso extensivo se da porque permite una clara división de responsabilidades
entre la presentación (vista y controlador) y el dominio (modelo).
\texttt{MVC} desacopla vistas y modelos mediante el establecimiento de un
protocolo de suscribirse / notificar entre ellos. La vista debe asegurarse
de que su aspecto visual refleje el estado del modelo. Cada vez que cambian
los datos del modelo, el modelo notifica a las vistas que depende de ella.
Este enfoque permite conectar múltiples vistas a un modelo para proporcionar
diferentes presentaciones. También puede crear nuevas vistas para un modelo
sin reescribir este ultimo \cite[pag.~4]{14_gamma_1995}.

\subsection{MVC y Rails}
\texttt{MVC} fue ideado originalmente para aplicaciones gráficas convencionales,
donde los desarrolladores encontraron que la separación de responsabilidades
fomentadas por el patrón llevan a un menor acoplamiento lo que hacia al código
mucho mas fácil de escribir y mantener.

En el marco de trabajo Rails se hace uso de \texttt{MVC} como patrón de arquitectura
para construir sistemas web. En Rails se los modelos se definen haciendo
uso de la biblioteca \texttt{ActiveRecord}, esta biblioteca implementa una capa
de \textit{Object-relational mapping} (\texttt{ORM}) para facilitar el acceso
de información contenida en bases de datos relacionales, dado que es el caso
típico en sistemas web convencionales.

En Rails, la vista es responsable de la creación de la totalidad o parte de la
respuesta dada para ser mostrada en un navegador. En su forma mas simple, una
vista es un trozo de código \texttt{HTML} que muestra un texto fijo. Mas típicamente
se requiere mostrar contenido dinámico creado por una acción en un controlador.
El contenido dinámico es generado por medio de plantillas, el esquema
de plantillas más común es llamada \textit{Embedded Ruby} (\texttt{ERB}),
el cual inserta pedazos de código Ruby dentro de una vista, similar a la forma
como se hace en otros marcos de trabajo como PHP o JSP. También se pude hacer uso de
\texttt{ERB} para construir pedazos de código Javascript en el servidor
para ser ejecutados en el servidor, lo cual permite crear interfases
dinámicas haciendo uso \texttt{AJAX}.

Finalmente en Rails los controladores son el centro lógico del sistema. Coordina
la interacción entre el usuario, las vistas y el modelo \cite[pag.~29]{15_agile_hansson}.

\subsection{MVC y PEAT}
\texttt{PEAT} saca provecho de \texttt{MVC} de las siguientes maneras:

\begin{enumerate}
\item Modelo: Dado que en un principio los servicios web de recomendaciones y
  disagregacion estaban en construcción se hizo uso de \texttt{ActiveRecord}
  para tener datos reales pero estáticos para permitir la implementación
  de la interfaz de usuario. Posteriormente se reemplazaron estos modelos
  por nuevos modelos que hacían uso de los servicios web del \textit{backend},
  por medio de la biblioteca \texttt{Bezel}. Dado que hay un desacoplamiento
  entre los modelos y las vistas esto no implico grandes cambios al hacer el
  reemplazo.
\item Vista: Haciendo uso de plantillas se generan representaciones \texttt{HTML}
  y \texttt{JSON} de los principales modelos del sistema. Para ciertos modelos se
  tenia una tercera representación en forma de \texttt{PDF} del modelo.
\end{enumerate}

\section{Servicios web RESTful}
Un servicio web es un método de comunicación entre dos dispositivos sobre una red,
su creación es por la necesidad de que diferentes sistemas necesitan intercambiar
datos entre ellos, y el servicio web es el método de comunicación que permite a
los sistemas el intercambiar datos a través del Internet.

La Transferencia de Estado Representacional (\texttt{REST}) es una arquitectura
de software para desarrollar servicios web. En \texttt{REST} se define la
existencia de recursos (elementos de información), donde cada recurso tiene un
conjunto de representaciones posibles. Por ejemplo una lista de bugs por arreglar
(recurso) puede ser presentado en forma de un documento XML, una pagina HTML
o un archivo CSV (representaciones). Además se tienen cuatro características
clave \cite[pag.~79]{1_richardson_2007}:

\begin{itemize}
\item Protocolo cliente/servidor sin estado (\texttt{stateless}): cada mensaje
  \texttt{HTTP} contiene toda la información necesaria para comprender la
  petición. Esto hace que ni el cliente ni el servidor necesitan recordar ningún
  estado de las comunicaciones entre mensajes.
\item Conectividad (\texttt{connectedness}): Las representaciones son un hipermedio
  en el cual se tienen ligas a otros recursos. Como resultado de esto, es posible
  navegar de un recurso \texttt{REST} a muchos otros, sin necesidad de una
  infraestructura adicional.
\item Direccionabilidad (\texttt{addressability}): Siendo la capacidad para
  identificar los recursos del sistema. Cada recurso es direccionable únicamente
  a través de su \texttt{URI}.
\item Interfaz uniforme (\texttt{uniform interface}): Se tiene un conjunto de
  operaciones bien definidas que se aplican a todos los recursos del sistema.
  Se usan los verbos de \texttt{HTTP} pare definir las operaciones mas importantes
  que son GET, POST, PUT y DELETE.
\end{itemize}

Los servicios web que implementan una arquitectura \texttt{REST} se hacen llamar
RESTful.

\subsection{REST y Rails}

La arquitectura \texttt{REST} es parte vital de Rails, todo el enrutamiento y
manejo de peticiones se basa en esta arquitectura.

En \texttt{REST} se hace uso de un conjunto finito de verbos para operar sobre otro
conjunto de objetos. Dado que estamos usando HTTP como capa de transporte, los
verbos corresponden a los métodos HTTP (GET, PUT, PATCH, POST y DELETE).
Los objetos corresponden a los recursos del sistema, los cuales son etiquetados
usando URLs.

Cuando un navegador solicita una pagina a Rails por medio de una petición de una
dirección URL utilizando un método HTTP especifico. Cada método es una petición
para realizar una operación sobre un recurso. En Rails se define una ruta de
recursos la cual asigna un numero de peticiones asociadas a las acciones de
un controlador.

Tomando como ejemplo como recurso el concepto de edificio, cuando Rails recibe una
solicitud entrante para:

\begin{verbatim}
DELETE /buildings/17
\end{verbatim}

Entonces se pide al router que asigne una acción del controlador. Si se tiene
que la primera ruta coincidente es:

\begin{verbatim}
resources :buildings
\end{verbatim}

Entonces Rails tendría que enviar la solicitud al método destroy en el controlador
buildings con {id: 17}.

También se tiene que Rails saca todo el provecho del hecho de que REST define
una interfaz uniforme para generar rápidamente las rutas a un recurso dado

Usando todavía el ejemplo anterior se ve por la Figura 2.2 como se definen
las URLs y sus verbos asociados para manejar el recurso.

\jcimage{1.0}{imagenes/REST-Rails.png}{REST en Rails.}

\subsection{REST y PEAT}
