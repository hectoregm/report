\chapter{Fundamentos teóricos}

\section{Patrón Modelo-Vista-Controlador (MVC)}
El patrón Modelo Vista Controlador (\texttt{MVC}) es probablemente el patrón
mas utilizado y citado para el desarrollo de interfaces de usuario.
\texttt{MVC} consiste de tres tipos de objetos:

\begin{itemize}
\item Modelo: Representa la información de dominio del sistema.
\item Vista: Es una representación visual del modelo.
\item Controlador: Define la forma en que la interfaz reacciona a la entrada
  del usuario.
\end{itemize}

Su uso extensivo se da porque permite una clara división de responsabilidades
entre la presentacion (vista y controlador) y el dominio (modelo).
\texttt{MVC} desacopla vistas y modelos mediante el establecimiento de un
protocolo de suscribirse / notificar entre ellos. La vista debe asegurarse
de que su aspecto visual refleje el estado del modelo. Cada vez que cambian
los datos del modelo, el modelo notifica a las vistas que depende de ella.
Este enfoque permite conectar múltiples vistas a un modelo para proporcionar
diferentes presentaciones. También puede crear nuevas vistas para un modelo
sin reescribir este ultimo.

