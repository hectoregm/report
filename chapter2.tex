\chapter{Fundamentos teóricos}

\section{Patrón Modelo-Vista-Controlador (MVC)}
El patrón Modelo Vista Controlador (MVC) es probablemente el patrón
mas utilizado y citado para el desarrollo de interfaces de usuario y sistemas web.
MVC consiste de tres tipos de objetos \cite{22_martin_fowler_mvc}:

\begin{itemize}
\item Modelo: representación de la información de dominio del sistema.
\item Vista: representación visual del modelo.
\item Controlador: define la forma en que la interfaz reacciona a la entrada
  del usuario.
\end{itemize}

\jcimage{1.0}{imagenes/MVC-Web.png}{Patrón MVC para sistemas web \cite{15_agile_hansson}.}

MVC fue ideado originalmente para aplicaciones gráficas convencionales,
donde los desarrolladores encontraron que la separación de responsabilidades,
entre la presentación (vista y controlador) y el dominio (modelo), fomentadas
por el patrón llevan a un menor acoplamiento lo que hacia al código
mucho mas fácil de escribir y mantener (Ver Figura 2.1).

MVC desacopla vistas y modelos mediante el establecimiento de un
protocolo de suscripción / notificación. La vista debe asegurarse
de que su aspecto visual refleje el estado del modelo. Cada vez que cambian
los datos del modelo, el modelo notifica a las vistas que depende de ella.
Este enfoque permite conectar múltiples vistas a un modelo para proporcionar
diferentes presentaciones. También puede crear nuevas vistas para un modelo
sin reescribir este último \cite{14_gamma_1995}.

\subsection{MVC y Rails}
En el marco de trabajo Rails se hace uso de MVC como patrón de arquitectura
para implementar sistemas web. En Rails los modelos se definen haciendo
uso de la biblioteca \textit{ActiveRecord}, la cual implementa el
patrón de mapeo objeto-relacional (\textit{Object-relational mapping},
ORM) esto facilita el acceso de información contenida en
bases de datos relacionales, dado que es el caso típico en sistemas web
convencionales.

En Rails, la vista es responsable de la creación de la respuesta dada para
ser mostrada en un navegador. En su forma mas simple, una
vista es un trozo de código HTML que muestra un texto fijo. Frecuentemente
se requiere mostrar contenido dinámico creado por una acción en un controlador,
por lo que el contenido dinámico es generado por medio de plantillas, el esquema
de plantillas más común es llamado Ruby Embebido (\textit{Embedded Ruby}, ERB),
el cual inserta pedazos de código de Ruby dentro de una vista, similar a la forma
como se hace en otros marcos de trabajo como PHP o JSP. También se pude hacer uso de
Ruby Embebido para incrustar pedazos de código Javascript en el servidor,
que serán ejecutados en el navegador, lo cual permite crear interfases
dinámicas haciendo uso de \textit{Asynchronous JavaScript and XML} (AJAX).

Finalmente en Rails los controladores son el centro lógico del sistema, ya que
coordinan la interacción entre el usuario, las vistas y el modelo
\cite{15_agile_hansson}.

\subsection{MVC y PEAT}
\texttt{PEAT} saca provecho del patrón MVC de las siguientes maneras \cite{15_agile_hansson}:

\begin{enumerate}
\item Modelo: dado que en un principio los servicios web de recomendaciones y
  desagregación estaban en construcción se hizo uso de la biblioteca
  \textit{ActiveRecord} para tener datos reales estáticos que permitieran
  la implementación de la interfaz de usuario.
  Posteriormente se reemplazaron estos modelos por nuevos modelos que hacían
  uso de los servicios web del \textit{backend}, por medio de la biblioteca
  \texttt{Bezel}.
  El realizar este reemplazo no implicó grandes cambios, dado el desacoplamiento
  existente entre los modelos y las vistas.
\item Vista: haciendo uso de plantillas se generan representaciones de los
  principales modelos del sistema en HTML y JSON.

  Sin embargo, para ciertos modelos, como las \textit{recomendaciones} y los
  \textit{reportes de consumo}, se tenia una tercera representación en forma
  de PDF del modelo.
\end{enumerate}

\section{Servicios web RESTful}
La \textit{World Wide Web Consortium}, W3C, define que un servicio web
en general es un sistema de software diseñado para dar soporte a interacciones
máquina-máquina a través de una red informática \cite{17_w3c_webservice}.
Su implementación nació de tener diferentes sistemas que puedan intercambiar
datos entre ellos.

Transferencia de Estado Representacional (\textit{Representational State Transfer}, REST) es una arquitectura de software para la implementación de servicios web.
En REST se define la existencia de recursos (elementos de información), donde
cada recurso tiene un conjunto de representaciones posibles.
Por ejemplo una lista de errores por arreglar (recurso) puede ser presentado en
forma de un documento XML, una pagina HTML o un archivo CSV (representaciones).

\vspace{2.5mm}

Se tienen cuatro características principales \cite{1_richardson_2007}:

\begin{itemize}
\item Protocolo cliente/servidor sin estado (\textit{stateless}): cada mensaje
  HTTP contiene toda la información necesaria para comprender la petición.
  Esto implica que ni el cliente ni el servidor necesitan recordar ningún
  tipo de estado.
\item Conectividad (\textit{connectedness}): las representaciones son un hipermedio
  en el cual se tienen ligas a otros recursos, como resultado de esto, es posible
  navegar de un recurso REST a muchos otros, sin necesidad de una
  infraestructura adicional.
\item Direccionabilidad (\textit{addressability}): la capacidad para
  identificar los recursos del sistema. Cada recurso es direccionable únicamente
  a través de su Identificador de Recursos Uniforme (\textit{Uniform Resource Identifier}, URI).
\item Interfaz uniforme (\textit{uniform interface}): se tiene un conjunto de
  operaciones bien definidas que se aplican a todos los recursos del sistema.
  Se usan los métodos de HTTP para definir las operaciones más importantes
  como son GET, POST, PUT, PATCH y DELETE.
\end{itemize}

Los servicios web que implementan una arquitectura REST se suelen llamar
servicios web RESTful.

\subsection{REST y Rails}

La arquitectura REST es parte vital de Rails, todo el enrutamiento y
manejo de peticiones se basa en esta arquitectura.

En REST se hace uso de un conjunto finito de verbos para operar sobre otro
conjunto de objetos. Dado que estamos usando HTTP como capa de transporte, los
verbos corresponden a los métodos HTTP (GET, POST, PUT, PATCH, y DELETE).
Los objetos corresponden a los recursos del sistema, los cuales son etiquetados
usando un (\textit{Uniform Resource Locator} URL).

Un navegador solicita páginas de Rails al hacer una petición para una dirección URI
haciendo uso de un método HTTP específico, como GET o POST. Cada método es una
petición para realizar una operación sobre el recurso.

Haciendo uso de la interfaz uniforme, Rails define todo un conjunto de rutas
para un recurso, tomando como ejemplo el concepto de edificio (\textit{Building}),
entonces, se define el recurso y sus rutas asociadas con lo siguiente:

\begin{verbatim}
resources :buildings
\end{verbatim}

En la Tabla \ref{table:rest} se pueden ver las rutas y métodos asociados para
las principales operaciones sobre el recurso \textit{buildings}, esto acelera
en gran medida el desarrollo de servicios web.

\begin{table}[h!]
  \centering
  \begin{tabular}{|m{1.2cm}|m{2.5cm}|m{3cm}|m{3.5cm}|}
    \hline
    \rowcolor{lightgray} \textbf{Método HTTP} & \textbf{Ruta} & \textbf{Controlador\#Acción} & \textbf{Usado para}\\
    \hline\hline
    GET & /buildings & buildings\#index & Listar todos los edificios\\
    \hline
    GET & /buildings/new & buildings\#new & Regresar una forma para crear un edificio\\ [2ex]
    \hline
    POST & /buildings & buildings\#create & Crear un nuevo edificio\\
    \hline
    GET & /buildings/:id & buildings\#show & Mostrar un edificio especifico\\
    \hline
    GET & /buildings/:id/edit & buildings\#edit & Regresar una forma para editar un edificio\\
    \hline
    PATCH & /buildings/:id & buildings\#update & Actualizar un edificio\\
    \hline
    DELETE & /buildings/:id & buildings\#destroy & Borrar un edificio\\
    \hline
  \end{tabular}
  \caption{Rutas y métodos para el recurso \textit{buildings} \cite{18_rails_rest}.}
  \label{table:rest}
\end{table}

\subsection{REST y PEAT}
La arquitectura REST influye en las tres partes principales de \texttt{PEAT}.

\begin{itemize}
\item \textit{Frontend}: por medio de Rails se implementa un servicio web RESTful
  para proveer de información a la interfaz gráfica del sistema.
\item \textit{Backend}: los servicios web que componen el \textit{backend}, como el
  sistema de recomendaciones, son del tipo RESTful. El recurso principal del sistema
  es el de edificio (\textit{Building}) del cual se tiene alrededor de los demás
  recursos del sistema.
\item \textit{Middleware}: la biblioteca \texttt{Bezel} se beneficia gracias al hecho
  de que el \textit{backend} sea de tipo RESTful, pues facilita la integración del
  sistema, dado que hace un mapeo casi directo de los recursos a clases y objetos.

\end{itemize}


\section{Desarrollo guiado por pruebas y comportamiento}
El tener código fuente que sea limpio y que además tenga la funcionalidad deseada
es uno de los principales objetivos en cualquier sistema. Cabe señalar que
por código limpio se entenderá en este trabajo, aquel código fuente que sea fácil
de entender y de modificar \cite{4_beck_2003}.

El desarrollo guiado por pruebas (\textit{Test Driven Development}, TDD) y el
desarrollo guiado por comportamiento (\textit{Behavior Driven Development}, BDD)
son prácticas de ingeniería de software que tiene por objetivo el obtener código
limpio y funcional.

\subsection{Desarrollo guiado por pruebas (TDD)}
El desarrollo guiado por pruebas (\textit{Test Driven Development}, TDD) es una
practica de software que involucra el escribir las pruebas antes de
realizar la implementación, esto mejora el diseño y eficacia de la implementación.
Al escribir las pruebas antes que el código fuente, contrario al proceso habitual,
permite que las pruebas ayuden a guiar el diseño del código fuente en pequeños
pasos. En el largo plazo este proceso implementa un código fuente bien
estructurado que es fácil de mantener y de modificar\cite{4_beck_2003}.

\vspace{2.5mm}

El proceso clásico en TDD es el siguiente \cite{12_rappin_2010}:
\begin{enumerate}
\item Implementar una prueba: esta prueba debe ser breve y debe probar
  solamente una unidad de código (función, clase o módulo).
\item Asegurar que la prueba falle: se verifica que la prueba falle antes
  de escribir cualquier código. Esto es para asegurar que la prueba realmente
  hace lo que se espera de ella.
\item Implementar la nueva funcionalidad: se implementa el mínimo código de
  forma que la prueba pase satisfactoriamente.
\item Mejorar el código fuente: se elimina toda repetición que se originó al
  implementar la nueva funcionalidad para lograr que la prueba pase
  satisfactoriamente, dentro de esta fase también se realiza cualquier
  otra optimización y/o abstracción necesaria. Esta proceso recibe el nombre de
  refactorización, es parte vital del proceso.
\end{enumerate}

Este proceso se repite hasta que se termina de implementar toda la funcionalidad
del sistema. Al seguir este proceso, en teoría, se asegura que el código fuente
siempre es lo mas simple posible y que esta completamente verificado.

\subsubsection{Ventajas}
TDD va mas allá de la mera verificación al hacer uso de las pruebas para
mejorar la estructura del código fuente, por esto es que TDD es una práctica
de software y no solamente una herramienta de verificación.

Al continuamente alinear el código fuente a las pruebas se obtiene código fuente
conformado por pequeños métodos, cada uno de los cuales tiene solamente tiene
una sola responsabilidad. Estos métodos tienden a tener bajo acoplamiento y
con pocos efectos laterales, lo que facilita su comprensión y mantenimiento.

\subsubsection{Desventajas}
TDD no es un substituto para pruebas de aceptación, es decir, pruebas que confirman
que el software funciona según los requerimientos del cliente.

TDD asume que se conoce el resultado esperado que se quiere verificar,
cuando los requerimientos no son totalmente claros entonces TDD no es tan benéfico
puesto que es difícil escribir pruebas de un proceso que no se conocen en detalle.

\subsubsection{RSpec}
\textit{RSpec} es un marco de trabajo del proceso TDD/BDD para el lenguaje de
programación Ruby. \textit{RSpec} define un lenguaje de dominio especifico (DSL)
para implementar pruebas unitarias y de aceptación \cite{23_chelimsky_2010}.

\vspace{2.5mm}

Un ejemplo de una prueba unitaria haciendo uso de \textit{RSpec}:
% Ejemplo de prueba en RSpec
\lstinputlisting[language=Ruby]{code/rspec-example.rb}

% FIXME: Add better description of the example from rspec
El método \textit{it} (en la línea 6) crea un ejemplo del comportamiento de
\textit{Bezel::Client} dentro de un contexto, el cual es que el servidor C3
esta corriendo la versión 2.

\subsection{Desarrollo guiado por comportamiento (BDD)}
En desarrollo guiado por comportamiento (\textit{Behavior Driven Development}, BDD)
es una practica de software surgida de TDD, al igual que esta ultima
involucra escribir las pruebas antes de escribir el código fuente y la
refactorización contínua del código fuente. Pero mientras que TDD se enfoca
en describir el comportamiento de unidades de código en BDD el enfoque es
describir el comportamiento en la interacción de varios módulos del sistema.

En BDD la unidad de prueba son las pruebas de aceptación las cuales definen
una serie de escenarios, los cuales surgen de los casos de uso del sistema.
Estos escenarios casi siempre se especifican en un lenguaje DSL que hace uso
de lenguaje natural (casi siempre inglés) para definir el comportamiento
esperado del sistema \cite{23_chelimsky_2010}.

\subsubsection{Ventajas}
En BDD se guía el diseño del código fuente del sistema a través de las pruebas
de aceptación que se especifican en colaboración con el cliente, por lo que
se obtiene software que funciona según los requerimientos del cliente.

\subsubsection{Desventajas}
BDD no es un substituto para pruebas unitarias, dado que el enfoque de
las pruebas de aceptación es sobre el comportamiento de grandes bloques
del sistema y no en los detalles del mismo.

Para obtener el mayor beneficio con BDD es necesario especificar las pruebas
de aceptación con el cliente y gente relacionada con el negocio, lo cual no
siempre es posible \cite{12_rappin_2010}.

\subsubsection{Cucumber}
\textit{Cucumber} es un marco de trabajo que permite el especificar y ejecutar
pruebas de aceptación siguiendo el proceso BDD. Define un lenguaje llamado
Gherkin que permite escribir escenarios en lenguaje natural. De esta manera
facilita que personas no técnicas, como el cliente o gente de negocio,
puedan ayudar a escribir o dar el visto bueno a las pruebas de aceptación
\cite{23_chelimsky_2010}.

\vspace{2.5mm}

Un ejemplo de prueba de aceptación es el siguiente:
\begin{lstlisting}
Característica:
 Para disminuir mis costos de energía eléctrica
 Como un usuario del sistema PEAT
 Quiero saber sobre las posibles mejoras a mi edificio

  Esquema del escenario: Desplegar la lista de recomendaciones
    Dado que ingreso al sistema como el usuario @juan por primera vez
    Y que tengo una cuenta y un edificio
    Cuando me entro a la pagina de recomendaciones
    Entonces Yo debo ver un botón de "Añadir al plan" en cada recomendación
\end{lstlisting}

Gherkin es un lenguaje que usa el indentado para definir la estructura, de manera
que los saltos de linea dividen las diferentes declaraciones,
la mayoría de las lineas empiezan con palabras clave.

% FIXME: , por ejemplo (cuales usas en el ejemplo) hay que explicarlo mas.

\vspace{2.5mm}

En los escenarios se tiene tres principales palabras clave:
\begin{itemize}
\item Dado (\textit{Given}): el propósito de esta declaración es el poner
  el sistema en el estado deseado para iniciar la prueba correspondiente,
  antes de que el usuario (o un sistema externo) interactué con el sistema.
\item Cuando (\textit{When}): el propósito de esta declaración es el describir
  la acción que se realizar por el usuario, la cual vamos a probar.
\item Entonces (\textit{Then}): el propósito de esta declaración es observar
  los resultado de la acción realizada, estas observaciones debe ser visibles
  para el usuario o sistema externo.
\end{itemize}

\subsection{TDD/BDD y PEAT}
Dado que TDD y BDD se enfocan en el comportamiento a diferentes niveles
de un sistema éstos son complementarios y permiten subsanar sus respectivas
desventajas. Así que en la implementación del sistema PEAT está basado en
TDD y BDD para guiar el diseño del sistema, haciendo uso de los marcos
de trabajo \textit{RSpec} y \textit{Cucumber} para realizar la especificación
de las pruebas unitarias y de aceptación del sistema.

Cabe mencionar que el uso de pruebas de aceptación también permitió que la
transición de uso de datos estáticos al uso del API final fuera lo mas fluída
posible al indicar los puntos críticos en los que se tenían que realizar cambios
para obtener nuevamente el comportamiento esperado por el cliente.

\section{Ruby, metaprogramación y lenguajes DSL}
El lenguaje de programación Ruby es un lenguaje dinámico con una gramática compleja
pero expresiva y con una biblioteca de clases base con un API extenso y poderoso.
Ruby toma inspiración de otros lenguajes de programación como Lisp, Smalltalk y Perl.

Ruby es un lenguaje orientado a objetos puro, pero también es adecuado para los
paradigmas funcionales y imperativos, y incluye poderosas capacidades para
la metaprogramación las cuales son usadas para crear fácilmente lenguajes
de dominio especifico (\textit{Domain Specific Language} DSL)
\cite{24_matsumoto_2008}.

\subsection{Metaprogramación}
Metaprogramacion es escribir código fuente que manipule las estructuras del lenguaje
en tiempo de ejecución.

\subsection{Lenguajes de dominio especifico}
