\documentclass{article}
\usepackage{enumitem}
\usepackage[left=4cm,right=4cm,top=4cm,bottom=4cm,letterpaper]{geometry}
\usepackage[spanish]{babel}
\usepackage[utf8]{inputenc}

\renewcommand{\refname}{whatever}
\addto{\captionsspanish}{\renewcommand{\refname}{}}

\author{Héctor E. Gómez Morales}
\title{Proyecto para la Opción de Titulación\\
  por Trabajo Profesional:\\
  Diseño e implementación Sistema PEAT\\
  (Progressive Energy Audit Tool)}
\begin{document}
\author{ Alumno: Héctor Enrique Gómez Morales - No. de Cuenta: 401048742 \and
  Tutor: Karla Ramírez Pulido}
\maketitle
\section{Introducción}
Este proyecto aborda el diseño e implementación del sistema
PEAT (Progressive Energy Audit Tool) que es una aplicación web que
permite a los usuarios de pequeñas y medianas empresas (PyMES)
identificar y monitorear sus gastos en energía, con el fin de obtener
un mayor control en sus costos de energía.

Este sistema fue desarrollado siendo parte de la compañía
\textbf{Software Next Door} en el cargo de \textbf{Senior Software Engineer}.

Las actividades que desarrollé durante este proyecto fueron:
\begin{itemize}
\item Implementación de la interfase de usuario.
\item Implementación de una biblioteca que no solo permite realizar consultas
  al \textit{backend} existente sino que permite una serialización/deserialización
  de objetos de Javascript a Ruby.
\item Diseño e implementación de pruebas unitarias, funcionales e integración.
\item Implementación de un sistema de despliegue continuo de la aplicación, por
  medio de una automatización máxima de todos los pasos para lograr el despliegue
  del sistema.
\item Realización de optimizaciones tanto del lado de la de arquitectura como
  del código desarrollado, para asegurar el soporte de por lo menos mil usuarios concurrentes.
\end{itemize}

\section{Justificación}
En una gran cantidad de proyectos es suficiente
el hacer uso de un solo lenguaje de programación para llevar a buen
término un proyecto, es decir, terminarlo exitosamente.
Sin embargo conforme pasa el tiempo, las necesidades
de las compañías y usuarios exigen que hagamos uso de una serie de
tecnologías que representan casi siempre el uso de más de un lenguaje
de programación. Lo anterior, es debido a que ningún lenguaje de programación
actualmente es universalmente el mejor en todos los contextos posibles de uso.

En este proyecto se tiene que la mayor parte del \textit{backend} esta implementado
en Javascript, con un núcleo escrito en Java. Mientras que PEAT fue escrito
en Ruby, dada sus ventajas para realizar aplicaciones web.

Durante este proyecto se implementó una biblioteca no solo para hacer uso de
los servicios web ya existentes, sino que la biblioteca facilitaba la creación
de clases y objetos en Ruby basados en la jerarquía de clases definida
en Javascript. Usando otra de las fortalezas de Ruby que es la metaprogramación
para así crear ``al vuelo'' las clases y sus atributos según las especificaciones
enviadas por el servidor, manteniendo así la jerarquía de clases de un lenguaje
a otro.

Otro punto de creciente importancia es el despliegue de una aplicación, por un
lado hace algunos años las aplicaciones web se desarrollaban bajo el modelo de
tres capas: la base de datos, el servidor de la aplicación y el servidor web.
Ahora se tienen mas servicios
que deben estar en línea sobre todo para garantizar un servicio concurrente:
caches, equilibradores de carga, servidores de cola, etc. Por lo que tratar
de automatizar el despliegue de una aplicación en sus diferentes contextos,
desarrollo, producción y pruebas, es de vital importancia para el desarrollo
de software en tiempo y en forma.

Durante este proyecto se hacen uso extensivo de técnicas como la metaprogramación
y el despliegue continuo, para permitir un desarrollo acelerado, con el fin de
obtener rápidamente retroalimentación del usuario final.

\section{Descripción general del trabajo}
Pacific Gas and Electric Company (PG\&E) es una compañía proveedora de gas natural
y electricidad, una de las grandes en Estados Unidos con sede en
San Francisco, California. El sistema PEAT es el resultado de una licitación
iniciada por PG\&E para el desarrollo de un sistema enfocado a usuarios PyMES.

Su desarrollo es de importancia crítica para PG\&E y al gobierno de
California puesto que su funcionamiento es un requisito en una nueva
ley de facturación de energía eléctrica desde el 2013. La nueva ley buscaba
lograr distribuir la carga de la red eléctrica y una de las formas
principales para lograr esto era desincentivar el uso de la red
eléctrica en horas pico al darle la facultad a las compañías como PG\&E
de cobrar tasas mucho mas altas en estas horas.

PG\&E había realizado el despliegue de medidores inteligentes en la
mayor parte de sus clientes, por lo que se tenía acceso a una información
muy detallada del consumo de energía de sus clientes.

Para darle el mayor valor e información a las empresas era
vital obtener mayor información de su entorno de operación: numero
de edificios asociados a la cuenta, rubro de la empresa, numero de
empleados, etcétera. Entre mayor información se pudiera captar sobre la
empresa el sistema tendría mas facilidad en obtener un desglose
mas correcto y útil de sus consumos de energía.

El objetivo es dar la mayor utilidad posible con la menor
información disponible fomentando en el usuario el dar mas
información para así darle un mejor control acerca de su consumo energético.

Cabe mencionar que la compañía C3 Energy, en la que trabajé
por medio de Software Next Door, compañía quien ganó esta licitación puesto
que ya contaba con la infraestructura necesaria para el procesamiento
de una gran cantidad de datos de consumo de energía, ya que tiene un sistema
de monitoreo de consumo de energía enfocada a empresas de nivel
multinacional. El reto era pasar de un sistema y procesos diseñados
para una docena de clientes de gran tamaño, a un sistema que diera
servicio a cientos de miles de clientes PyMES.

A grandes rasgos el sistema cuenta con los siguientes requerimientos:
\begin{itemize}
\item Autentificar a los usuarios por uso de credenciales de acceso
  obtenidas en portal web de PG\&E.
\item Dar información útil aunque el usuario solo de
  el mínimo de información sobre su empresa.
\item Proporcionar recomendaciones para disminuir sus
  gastos en energía con base en el consumo e información proporcionada
  hasta el momento.
\item Soportar por lo menos a mil usuarios concurrentes.
\end{itemize}

\section{Índice}
Introducción
\begin{enumerate}
\item Situación Actual
  \begin{enumerate}[label*=\arabic*.]
  \item{Contexto}
  \item{Objetivo primario}
  \item{Objetivos secundarios}
  \item{Propuesta}
  \end{enumerate}
\item Fundamentos teóricos para el desarrollo del sistema PEAT
  \begin{enumerate}[label*=\arabic*.]
  \item{Patrón Modelo-Vista-Controlador (MVC)}
  \item{Servicios web RESTful}
  \item{Desarrollo guiado por pruebas}
  \item{Ruby, metaprogramación y lenguajes DSL}
  \end{enumerate}
\item Diseño e implementación del sistema PEAT
  \begin{enumerate}[label*=\arabic*.]
  \item Arquitectura previa
  \item Interfase de usuario
  \item Biblioteca Bezel
  \item Pruebas de integración
  \item Optimizaciones y cache
  \item Despliegue continuo
  \end{enumerate}
\item Conclusiones
\item Bibliografía
\end{enumerate}

\section{Bibliografía Propuesta}
\nocite{*}
\bibliographystyle{acm}
\bibliography{biblio}

\vspace*{5cm}
\noindent\begin{tabular}{ll}
\makebox[2.5in]{\hrulefill} & \makebox[3in]{\hrulefill}\\
Héctor Enrique Gómez Morales& M. en I. y M. en A.O. Karla Ramírez Pulido\\
%Alumno&     Tutor\\% adds space between the two sets of signatures
\end{tabular}
\end{document}
