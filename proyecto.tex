\documentclass{article}
\usepackage{enumitem}
\usepackage[left=4cm,right=4cm,top=4cm,bottom=4cm,letterpaper]{geometry}
\usepackage[spanish]{babel}
\usepackage[utf8]{inputenc}
\author{Héctor E. Gómez Morales}
\title{Proyecto para la Opción de Titulación por Trabajo Profesional: \\
  Sistema PEAT (Progressive Energy Audit Tool)}
\begin{document}
\author{ Alumno: Héctor Enrique Gómez Morales \and
  Tutor: Karla Ramírez Pulido}
\maketitle
\section{Introducción}
Este proyecto aborda el diseño e implementación de la aplicación
PEAT (Progressive Energy Audit Tool) que es una aplicación web que
permite a los usuarios de pequeñas y medianas empresas (PyMES)
identificar y monitorear sus gastos en energía, para de esta forma
darle un mayor control en sus costos de energía.

Esta aplicación fue desarrollada siendo parte de la compañía
\textbf{Software Next Door} en el cargo de \textbf{Senior Software Engineer}.

Las actividades que desarrolle durante este proyecto fueron:
\begin{itemize}
\item Implementación de la interfase de usuario.
\item Implementación de una biblioteca que no solo permite realizar consultas
  al backend existente si no que mantiene la jerarquía de clases.
\item Diseño e implementación de pruebas unitarias, funcionales e integración.
\item Implementación de un sistema de despliegue continuo de la aplicación.
\item Realización de optimizaciones, de arquitectura y de código,
  para asegurar el soporte de por lo menos mil usuarios concurrentes.
\end{itemize}

\section{Justificación}
En una gran cantidad de proyectos es suficiente
con hacer uso de un solo lenguaje de programación para llevar a buen
termino un proyecto. Pero conforme pasa el tiempo, las necesidades
de las compañías y usuarios exigen que hagamos uso de una serie de
tecnologías que representan casi siempre el uso de mas de un lenguaje
de programación. Esto se da porque ningún lenguaje es universalmente
el mejor en todos los contextos posibles de uso.

En este proyecto se tiene que la mayor parte del backend esta implementado
en Javascript, con un núcleo escrito en Java. Mientras que PEAT fue escrito
en Ruby, dada sus ventajas para realizar aplicaciones web.

Durante este proyecto se implemento una biblioteca no solo para hacer uso de
los web services ya existentes, si no que la biblioteca facilitaba la creación
de clases y objetos en Ruby basados en la jerarquía de clases definida
en Javascript. Usando otra de las fortalezas de Ruby que es la metaprogramación
para así crear al vuelo las clases y sus atributos según las especificaciones
mandadas por el servidor, por lo que de este modo se mantenía la jerarquía
de clases de un lenguaje a otro.

Otro punto de creciente importancia es el despliegue de una aplicación, en tiempos
pasados para las aplicaciones web se tenían tres capas: la base de datos,
el servidor de la aplicación y el servidor web. Ahora se tienen mas servicios
que deben estar en linea sobre todo para garantizar un servicio concurrente:
caches, equilibradores de carga, servidores de cola, etc. Por lo que tratar
de automatizar el despliegue de una aplicación en sus diferentes contextos,
desarrollo, producción y pruebas, es de vital importancia para el desarrollo
en tiempo y en forma.

Durante este proyecto se hacen uso extensivo de estas técnicas: la metaprogramación
y el despliegue continuo, para permitir un desarrollo acelerado el aun permite
obtener rápidamente retroalimentación del usuario final.

\section{Descripción General del Trabajo}
PEAT es producto de una licitación auspiciada por Pacific Gas and
Electric Company (PG\&E), proveedora de gas natural y electricidad
para casi dos tercios del norte de California, USA, para
el desarrollo de una aplicación web enfocada a usuarios PyMES.

Su desarrollo era de importancia critica para PG\&E y al gobierno de
California puesto que su funcionamiento era un requisito en una nueva
ley de facturación de energía eléctrica en 2013. La nueva ley buscaba
lograr distribuir la carga de la red eléctrica, una de las formas
principales para lograr esto era desincentivar el uso de la red
eléctrica en horas picos al darle la facultad a las utilidades como PG\&E
de cobrar tasas mucho mas altas en estas horas.

PG\&E había realizado el despliegue de medidores inteligentes en la
mayor parte de sus clientes con lo que se tenia acceso a una información
muy detallada del consumo de energía de sus clientes.

Para darle el mayor valor e información a las empresas era
vital obtener mayor contexto de su entorno de operación: numero
de edificios asociados a la cuenta, rubro de la empresa, numero de
empleados, etc. Entre mayor información se pudiera captar sobre la
empresa el sistema tendría mas facilidad en obtener un desglose
mas correcto y útil de sus consumos de energía.

El objetivo era dar la mayor utilidad posible con la menor
información disponible pero fomentando al usuario el dar mas
información para darle un mejor control de su consumo energético.

La compañía C3 Energy, en la que trabaje por medio de Software Next Door,
gano esta licitación puesto que ya contaba con la infraestructura
para el procesamiento de una gran
cantidad de datos de consumo de energía puesto que tenia un sistema
de monitoreo de consumo de energía pero enfocada a empresas de nivel
multinacional. El reto era que se pasaba de un alcance de una docena
de clientes, con los cuales se trataba directamente, a lidiar con
cientos de miles de empresas PyMES en las cuales se obtenía una
parte de la información por parte de PG\&E y otra parte por el
empresario.

En grandes rasgos el sistema tenia los siguientes requerimientos:
\begin{itemize}
\item Tener una interfase web, usando como autentificación
  sus credenciales de acceso en el portal web de PG\&E.
\item El sistema debe dar información útil aunque el usuario solo de
  el mínimo de información sobre su empresa.
\item El sistema debe proporcionar recomendaciones para disminuir sus
  gastos en energía en base a su consumo e información proporcionada
  hasta el momento
\item Soportar por lo menos a mil usuarios concurrentes
\end{itemize}

\section{Índice}
\begin{enumerate}
\item Introducción
  \begin{enumerate}[label*=\arabic*.]
  \item{Patrón Modelo-Vista-Controlador}
  \item{Servicios web RESTful}
  \item{Desarrollo guiado por pruebas}
  \item{Ruby, metaprogramación y lenguajes DSL}
  \end{enumerate}
\item Planteamiento del problema
\item Desarrollo
  \begin{enumerate}[label*=\arabic*.]
  \item Arquitectura previa
  \item Interfase de usuario
  \item Biblioteca Bezel
  \item Pruebas de integración
  \item Optimizaciones y cache
  \item Despliegue continuo
  \end{enumerate}
\item Conclusiones
\end{enumerate}

\vspace*{5cm}
\noindent\begin{tabular}{ll}
\makebox[2.5in]{\hrulefill} & \makebox[2.5in]{\hrulefill}\\
Héctor Enrique Gómez Morales& M. en A.O. Karla Ramírez Pulido\\
%Alumno&     Tutor\\% adds space between the two sets of signatures
\end{tabular}
\end{document}
