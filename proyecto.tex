\documentclass{article}
\usepackage{enumitem}
\usepackage[left=4cm,right=4cm,top=4cm,bottom=4cm,letterpaper]{geometry}
\usepackage[spanish]{babel}
\usepackage[utf8]{inputenc}

\renewcommand{\refname}{}
\addto{\captionsspanish}{\renewcommand{\refname}{}}

\author{Héctor E. Gómez Morales}
\title{Proyecto para la Opción de Titulación\\
  por Trabajo Profesional:\\
  Diseño e implementación del sistema PEAT\\
  (Progressive Energy Audit Tool)}
\begin{document}
\author{ Alumno: Héctor Enrique Gómez Morales - No. de Cuenta: 401048742 \and
  Tutor: Karla Ramírez Pulido}
\maketitle
\section{Introducción}
Este proyecto aborda el diseño e implementación del sistema \textit{Progressive
 Energy Audit Tool} (PEAT), el cual es un sistema web que
permite a los usuarios de pequeñas y medianas empresas (PyMES)
identificar y monitorear sus gastos de energía eléctrica, con el fin de obtener
un mayor control en los costos de ésta.

Este sistema fue desarrollado siendo parte de la compañía
\textbf{Software Next Door} en el cargo de \textbf{Senior Software Engineer}.

\vspace{2.5mm}

Las actividades que desarrollé durante este proyecto fueron:
\begin{itemize}
\item Implementación de la interfase de usuario.
\item Implementación de una biblioteca que no solo permite realizar consultas
  al \textit{backend} existente sino que permite la serialización/deserialización
  de objetos de Javascript a Ruby.
\item Diseño e implementación de pruebas unitarias, funcionales y de integración.
\item Implementación de un sistema de despliegue continuo de la aplicación, por
  medio de la automatización máxima de todos los pasos necesarios para lograr
  el despliegue del sistema.
\item Implementación de optimizaciones tanto del lado de la arquitectura como
  del código desarrollado, para asegurar el soporte de por lo menos mil usuarios
  concurrentes.
\end{itemize}

\section{Justificación}
En una gran cantidad de proyectos resulta ser suficiente el hacer uso de un
único lenguaje de programación para llevar a buen término el proyecto, es decir,
terminarlo de forma exitosa. Sin embargo conforme pasa el tiempo, las necesidades
de las compañías y usuarios cambian, lo cual provoca que hagamos uso de una serie de
tecnologías no previstas en un inicio, lo cual representa casi siempre el uso
de más de un lenguaje de programación. Lo anterior, es debido a que ningún
lenguaje de programación actualmente es el mejor en todos los contextos
posibles de uso.

En el proyecto de implementación del PEAT cuenta con tres partes principales
de desarrollo: (a) el \textit{backend}  implementado en su mayoría usando
Javascript, con un núcleo escrito en Java, (b) una biblioteca en Ruby que permite
la interacción del \textit{backend} con aplicaciones web y (c) el \textit{frontend}
escrito con Ruby, dada sus ventajas para realizar sistemas web.

Durante este proyecto se implementó una biblioteca no solo para hacer uso de
los servicios web ya existentes, sino para facilitar la creación
de clases y objetos en Ruby basados en la jerarquía de clases definida
en Javascript. Cabe mencionar que haciendo uso de una de las fortalezas de Ruby
como lo es la metaprogramación para la creación de clases y sus
atributos ``al vuelo'', según las especificaciones enviadas por el servidor,
manteniendo así la jerarquía de clases de un lenguaje a otro, ha sido uno
de los objetivos cumplidos durante el proyecto.

Otro punto de creciente importancia es el despliegue de una aplicación, ya que por un
lado hace algunos años las aplicaciones web se desarrollaban bajo el modelo de
tres capas: la base de datos, el servidor de la aplicación y el servidor web; actualmente se tienen mas servicios
que deben estar en línea sobre todo para garantizar un servicio concurrente:
caches, equilibradores de carga, servidores de cola, etcétera. Por lo que
automatizar el despliegue de una aplicación en sus diferentes contextos,
desarrollo, producción y pruebas, es de vital importancia para el desarrollo
de software en tiempo y en forma.

Durante este proyecto se hacen uso extensivo de técnicas como la metaprogramación
y el despliegue continuo, para permitir un desarrollo acelerado, con el fin de
obtener rápidamente retroalimentación del usuario final.

\section{Descripción general del trabajo}
Pacific Gas and Electric Company (PG\&E) es una compañía proveedora de gas natural
y electricidad, una de las más grandes compañías de Estados Unidos con sede en
San Francisco, California. El sistema PEAT es el resultado de una licitación
iniciada por PG\&E para el desarrollo de un sistema enfocado a usuarios PyMES.

Su desarrollo es de importancia crítica tanto para PG\&E, como para el gobierno de
California puesto que su correcto funcionamiento es un requisito en una nueva
ley de facturación de energía eléctrica desde el 2013. La nueva ley buscaba
lograr distribuir la carga de la red eléctrica, siendo una de las formas
principales para lograr esto el desincentivar el uso de la red
eléctrica en horas pico al así darle la facultad a las compañías como PG\&E
de cobrar tasas mucho más altas en estas horas.

PG\&E había realizado el despliegue de medidores inteligentes en la
mayor parte de sus clientes, por lo que se tenía acceso a información
muy detallada del consumo de energía de éstos.

Actualmente para darle el mayor valor e información a las empresas es necesario
el obtener más datos de su entorno de operación; por ejemplo el número
de edificios asociados a la cuenta, rubro de la empresa, número
de empleados, etcétera. Entre mas detalles se pueda captar sobre
la empresa el sistema tendrá más facilidad en obtener un desglose más robusto,
completo y útil de sus consumos de energía.

El objetivo general es dar la mayor utilidad posible con la menor
información disponible fomentando en el usuario el compartir dicha
información sobre su empresa, para así obtener un mejor control
acerca de su consumo energético.

\vspace{2.5mm}

Los objetivos secundarios que apoyan al objetivo general de este trabajo son:
\begin{itemize}
\item Implementación de una interfaz que permita la obtención
  de información del usuario de una forma eficaz y sencilla.
\item Autentificar a los usuarios mediante el uso de credenciales de acceso
  obtenidas en el portal web de PG\&E.
\item Dar información útil aunque el usuario solo proporcione el
  mínimo de información sobre su empresa.
\item Proporcionar recomendaciones para disminuir sus
  gastos en energía con base en el consumo e información proporcionada
  hasta el momento.
\item Soportar por lo menos a mil usuarios concurrentes.
\item Implementar la infraestructura para el despliegue continuo de la
  aplicación. 
\end{itemize}

Cabe mencionar que la compañía C3 Energy, en la que trabajé
por medio de Software Next Door, compañía quien ganó dicha licitación ya
que contaba con la infraestructura necesaria para el procesamiento
de una gran cantidad de datos de consumo de energía, tiene un sistema
de monitoreo de consumo de energía enfocada a empresas de nivel
multinacional. El reto era pasar de un sistema y procesos diseñados
para una docena de clientes de gran tamaño, a un sistema que diera
servicio a cientos de miles de clientes PyMES.

\section{Índice}
Introducción
\begin{enumerate}
\item Situación actual
  \begin{enumerate}[label*=\arabic*.]
  \item{Contexto}
  \item{Objetivo general}
  \item{Objetivos secundarios}
  \item{Propuesta de desarrollo}
  \end{enumerate}
\item Fundamentos teóricos para el desarrollo del sistema PEAT
  \begin{enumerate}[label*=\arabic*.]
  \item{Patrón Modelo-Vista-Controlador (MVC)}
  \item{Servicios web RESTful}
  \item{Desarrollo guiado por pruebas}
  \item{Ruby, metaprogramación y lenguajes DSL}
  \end{enumerate}
\item Diseño e implementación del sistema PEAT
  \begin{enumerate}[label*=\arabic*.]
  \item Arquitectura previa
  \item Interfase de usuario
    \begin{enumerate}[label*=\arabic*.]
    \item Configuración inicial
    \item Tablero de energía
    \item Preguntas sobre el entorno de operación
    \item Lista de recomendaciones para el usuario
    \end{enumerate}
  \item Biblioteca Bezel
    \begin{enumerate}[label*=\arabic*.]
    \item Arquitectura
    \item Metaprogramación
    \item Pruebas de rendimiento
    \end{enumerate}
  \item Pruebas de integración
    \begin{enumerate}[label*=\arabic*.]
    \item Pruebas de conexiones concurrentes
    \item Pruebas de integración con PG\&E
    \end{enumerate}
  \item Optimizaciones y cache
    \begin{enumerate}[label*=\arabic*.]
    \item Balanceo de carga
    \item Configuración Nginx y Passenger
    \item Piscina de conexiones HTTP con \textit{Passenger} y \textit{Patron}
    \item Inicialización e invalidación de cache
    \end{enumerate}
  \item Despliegue continuo
    \begin{enumerate}[label*=\arabic*.]
    \item Computo elastico en la nube (\textit{Amazon Elastic Compute Cloud})
    \item \textit{RVM} y \textit{Capistrano}
    \item Definición y configuración de ambientes
    \end{enumerate}
  \end{enumerate}
\item Conclusiones
\item Bibliografía
\end{enumerate}

\section{Bibliografía propuesta}
\nocite{*}
\bibliographystyle{acm}
\bibliography{biblio}

\vspace*{2cm}
\noindent\begin{tabular}{ll}
\makebox[2.5in]{\hrulefill} & \makebox[3in]{\hrulefill}\\
Héctor Enrique Gómez Morales& M. en I. y M. en A.O. Karla Ramírez Pulido\\
\end{tabular}
\end{document}
