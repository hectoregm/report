\chapter{Diseño e implementación del sistema PEAT}

Para lograr los objetivos del sistema \texttt{PEAT} se dividió en cuatro
componentes principales:

\begin{itemize}
\item Perfil sencillo del edificio (\textit{Simple Building Profile}): este
  componente realiza la configuración inicial del edificio, realizando solamente
  un número limitado de preguntas básicas para definir las características básicas
  del mismo.
\item Perfil detallado del edificio (\textit{Detailed Building Profile}):
  este componente se encarga del manejo de las características básicas, detalladas y
  opcionales de un edificio.
\item Recomendaciones (\textit{Recommendations}): este componente genera una lista
  de medidas de ahorro de energía considerando la información ingresada al sistema
  hasta el momento. Además debe permitir al usuario obtener información detallada
  sobre las recomendaciones dadas y permitir que el usuario se comprometa a
  llevar acabo una recomendación.
\item Plan de ahorro de energía (\textit{Energy Savings Plan}): este componente
  presenta el consumo de energía del edificio al usuario, también presenta las
  recomendaciones que el usuario se ha comprometido a poner en acción y el
  ahorro que representa llevar acabo estas medidas.
\end{itemize}

El  modelo de edificio es el principal modelo dentro de \texttt{PEAT}, por lo que
obtener una correspondencia correcta entre un edificio y su consumo energético
es de vital importancia para dar la información y recomendaciones más útiles y
fidedignas al usuario.

\section{Modelo de información}

El objetivo principal del modelo de información de \texttt{PEAT} es definir el
modelo de edificio y su relación con el consumo energético del usuario. Así
tenemos el concepto de edificio el cual es una estructura independiente a la
que se le esta proporcionando un servicio ya sea de electricidad o gas.

\subsection{Estructura de facturación de PG\&E}
Para obtener esta correspondencia entre edificio-consumo es necesario
tener en cuenta la estructura de facturación usada por PG\&E puesto
que de esta estructura es que se obtiene el consumo de electricidad y gas.

Los modelos se pueden categorizar en dos grupos: demográficos o geográficos.

El usuario, la cuenta y el contrato de servicio representan el lado demográfico
del modelo de información:

\begin{itemize}
\item Usuario: representa al empresario, el cual puede tener una o más cuentas.
\item Cuenta: representa la contabilidad financiera del usuario. Tiene uno o más
  contratos de servicio, uno por tipo de servicio (electricidad o gas), también
  se tiene una o más direcciones de servicio asociadas.
\item Contrato de servicio: representa los servicios que el cliente adquiere
  de PG\&E, como la electricidad o el gas y puede tener uno o más puntos
  de servicio asociados.
\end{itemize}

La dirección de servicio, el punto de servicio y el medidor representan
la parte geográfica del modelo de información:

\begin{itemize}
\item Dirección de servicio: es una ubicación física en donde se prestan los
  servicios, en \texttt{PEAT} son el modelo que se quiere asociar al modelo
  de edificio para obtener el consumo del edificio, el cual puede tener uno o más
  puntos de servicio.
\item Punto de servicio: es una coordenada geográfica en donde se conectan los
  servicios y puede tener uno o más medidores asociados.
\item Medidor: es un dispositivo instalado en un punto de servicio que registra el
  consumo del servicio proporcionado.
\end{itemize}

La relación entre todos estos modelos se puede ver en la Figura 3.1.

\jcimage{1.0}{imagenes/PGE-facturacion.png}{Modelos principales para la facturación
  en PG\&E.}

\subsection{Relación Edificio-Facturación}

Para que \texttt{PEAT} pueda cumplir sus requerimientos funcionales es necesario
definir claramente la relación entre un edificio y su consumo energético calculado
por uno o varios medidores.
A pesar de que los medidores mantienen la información de consumo, el relacionar
directamente un medidor con un edificio no es una tarea sencilla para un usuario,
porque aunque el usuario generalmente conoce el número de medidores que tiene no
conoce los identificadores para cada uno de éstos.

Dado el escenario anterior para facilitar la creación del perfil de un edificio se
solicita al usuario que haga la correspondencia entre las direcciones de servicio
asociadas a su cuenta y el perfil de edificio que se está creando.
El usuario al establecer esta correspondencia permite al sistema obtener de forma
automatizada los puntos de servicio y sus medidores asociados al edificio, con el
objetivo de obtener su consumo energético real.

\section{Casos de uso}

\subsection{Diagrama general}

En la Figura 3.2 se muestra el diagrama general de casos de uso del sistema
\texttt{PEAT}. Se tienen diez casos de uso en total, siendo siete de éstas definidas
para \texttt{PEAT} y tres casos para la biblioteca \texttt{Bezel}.

Los casos de uso se agrupan en los cuatro componentes principales de la
siguiente manera:

\begin{itemize}
\item Perfil sencillo del edificio (\textit{Simple Building Profile}): Tiene
  solo el caso de uso \textit{Crear perfil del edificio} pero este caso de uso
  es uno de los más complejos en cuestión de interacción con el usuario y su
  importancia para obtener las características básicas del edificio.
\item Perfil detallado del edificio (\textit{Detailed Building Profile}): Contiene
  los casos de uso \textit{Administrar perfil del edificio} y
  \textit{Obtener información del edificio} los cuales se encargan de la
  administración y obtención de información sobre el edificio.
\item Recomendaciones (\textit{Recommendations}): Contiene los casos de uso
  \textit{Administrar recomendaciones} y \textit{Ver recomendación}.
\item Plan de ahorro de energía (\textit{Energy Savings Plan}): Contiene los
  casos de uso \textit{Ver historial de consumo} y \textit{Administrar plan de
    ahorro}, que se enfocan a la visualización del consumo eléctrico del usuario
  y las posibles recomendaciones para disminuir su consumo.
\end{itemize}

\jcimageinline{1.1}{imagenes/Diagrama-General-CasosDeUso.png}{Diagrama general de
  casos de uso de \texttt{PEAT}.}

\subsection{Crear perfil del edificio}

Para crear el perfil inicial de un edificio es necesario obtener las siguientes
características:

\begin{itemize}
\item La industria que mejor describe el negocio del usuario.
\item El tamaño del edificio (área).
\item El tipo de edificio.
\item La(s) dirección(es) de servicio asociada(s) al edificio.
\end{itemize}

Todas estas características se consideran básicas, es decir, vitales para el
funcionamiento del sistema por lo que todas son requeridas para construir el perfil
del edificio.

En este caso de uso también se obtiene la relación entre el edificio y su consumo
a partir de las direcciones de servicio que el usuario asocie al perfil.

El caso de uso esta diseñado de forma que el proceso sea realizado solamente
una vez por edificio. Las características básicas se mantienen accesibles por
medio del perfil detallado del edificio.

\begin{usecase}
  \addtitle{Caso de Uso:}{Crear perfil del edificio.}
  \addfield{Descripción:}{Un usuario PyME ingresa al sistema  para crear un perfil
    inicial de un edificio, respondiendo ciertas preguntas básicas sobre el
    edificio.}
  \addfield{Actor:}{Usuario PyME}
  \additemizedfield{Precondiciones:}{
  \item El usuario ha sido autentificado por medio de un elemento (\textit{token})
    de autenticación (\textit{Single Sign On}, SSO) asignado por el sistema de PG\&E.
  \item Los datos de intervalo y de facturación del usuario se encuentran en el
    sistema.
  }
  \additemizedfield{Requerimientos no funcionales:}{
  \item El usuario debe ser capaz de completar el perfil del edificio en 20 segundos
    o menos.
  }
  \addscenario{Flujo normal:}{
  \item El sistema checa si el usuario tiene más de una cuenta.
    \begin{enumerate}
    \item Si hay más de una cuenta el sistema despliega una lista
      de cuentas, y el sistema requiere que el usuario elija una cuenta.
    \end{enumerate}
  \item El sistema despliega información sobre la cuenta: número de cuenta y
    direcciones de servicio asociadas.
    \begin{enumerate}
    \item Si la cuenta tiene más de una dirección de servicio
      el sistema requiere que el usuario seleccione por lo menos una dirección
      de servicio.
    \end{enumerate}
  \item El usuario selecciona las direcciones de servicio que apliquen
    al edificio.
  \item El usuario puede proporcionar un apodo para el edificio.
  \item El usuario debe indicar la industria que mejor describe su negocio.
  \item El usuario debe ingresar la siguiente información sobre el edificio:
    \begin{enumerate}
    \item Tipo.
    \item Rango de tamaño.
    \item Antigüedad.
    \end{enumerate}
  \item El usuario puede enviar los datos ingresados o cancelar.
  \item El usuario puede añadir otro edificio o si ya tiene al menos un
    perfil de edificio asociado puede proceder al plan de ahorro.
  }
  \addfield{Postcondiciones:}{
    Se genera un identificador único para el perfil del edificio generado, el cual
    está asociado a los datos de facturación de la cuenta seleccionada por el
    usuario PyME.
  }
\end{usecase}

\subsection{Administrar perfil del edificio}

Para el sistema \texttt{PEAT} se tiene las siguientes características:
\begin{itemize}
\item Básicas: se refiere a las características esenciales de un edificio: tipo,
  tamaño y antigüedad, las cuales son vitales para el funcionamiento del sistema.
\item Detalladas: son las características más significativas y fáciles de
  responder. Por ejemplo: horas de operación, número de empleados, número
  de pisos, etcétera.
\item Opcionales: son características técnicas más avanzadas sobre el equipo
  y estructura de un edificio.
\end{itemize}

Las características básicas son definidas al crear el perfil inicial de un
edificio como se documenta en el caso de uso anterior. Las características
detalladas y opcionales son definidas por medio de dos rutas:

\begin{itemize}
\item Por medio del perfil detallado del edificio, en la que se despliega
  las preguntas y respuestas sobre las características del edificio.
  Esta ruta esta documentada en este caso de uso.
\item Un componente que es embebido en varias partes del sistema
  que realiza una sola pregunta al usuario para ir obteniendo progresivamente
  más información sobre el edificio. Este componente llamado componente
  de ingreso progresivo (\textit{Progressive Profile Widget} PPW).
  Esta ruta es documentada en el caso de uso \textit{Obtener información
    del edificio}.
\end{itemize}

\begin{usecase}
  \addtitle{Caso de Uso:}{Administrar perfil del edificio.}
  \addfield{Descripción:}{
    El usuario tiene acceso al perfil completo del edificio, es decir
    a todas las preguntas y respuestas asociadas a las características
    básicas, detalladas y opcionales de un edificio.}
  \addfield{Actor:}{Usuario PyME}
  \additemizedfield{Precondiciones:}{
  \item El usuario ha sido autentificado por medio de un elemento (\textit{token})
    de autenticación (\textit{Single Sign On} SSO) asignado por el sistema de PG\&E.
  \item El usuario debió haber creado previamente el perfil básico del edificio.
  }
  \additemizedfield{Requerimientos no funcionales:}{
  \item Todas las acciones realizadas por el usuario deben ser procesadas
    en menos de un segundo.
  }
  \addscenario{Flujo normal:}{
  \item El sistema verifica si la cuenta tiene más de un edificio asociado.
    \begin{itemize}
    \item Si hay más de un edificio asociado a la cuenta, el usuario
      tiene que seleccionar el edificio correspondiente.
    \end{itemize}
  \item El sistema despliega una lista de preguntas y respuestas, dividida en
    secciones. Las secciones completadas están marcadas con una marca de terminado.
    \begin{itemize}
    \item Las secciones que se despliegan dependen del tipo de edificio.
      Por ejemplo: construcción, iluminación, calefacción y refrigeración.
    \item El usuario puede ver y actualizar sus respuestas a todas las
      preguntas en una sola pantalla.
    \end{itemize}
  \item El usuario selecciona una sección, esta se expande para mostrar todas
    las preguntas de la misma.
  \item El usuario responde una o varias preguntas.
    \begin{enumerate}
    \item Las preguntas son ordenadas dentro de cada sección según un orden
      predefinido. %\footnote{Las secciones también tienen un orden predefinido.}
    \item Según vaya respondiendo el usuario se va actualizando un indicador
      del progreso de llenado del perfil. Cada pregunta con su respuesta tiene el
      mismo valor para calcular el progreso de llenado.
    \end{enumerate}
  \item El usuario puede elegir continuar con la siguiente sección o seleccionar
    en el encabezado otra sección.
  \item El sistema almacena la información en cuanto el usuario selecciona
    una respuesta. Al obtener nueva información el sistema debe recalcular
    las recomendaciones asociadas al edificio.
  }
  \additemizedfield{Postcondiciones:}{
  \item Un perfil del edificio más detallado si el usuario dio respuesta a una
    pregunta sin contestar sobre su edificio.
  \item Si el usuario dio nueva información se recalculan las recomendaciones
    asociadas al edificio.
  }
\end{usecase}

\subsection{Obtener información del edificio}

Uno de los requerimientos más importante para \texttt{PEAT} es el incentivar y
obtener más información sobre las características de los edificios del usuario,
ya que entre más datos se obtenga de éstos se podrán generar mejores planes
de ahorro y recomendaciones para el usuario.

La forma en como se aborda esta necesidad es por medio de la implementación
de un componente que esta presente en las dos partes principales del sistema:
el plan de ahorro y la lista de recomendaciones. Este elemento llamado
componente de ingreso progresivo (\textit{Progressive Profile Widget}, PPW),
tal componente realiza una sola pregunta al usuario obteniendo progresivamente
mas información sobre el edificio.

\begin{usecase}
  \addtitle{Caso de Uso:}{Obtener información del edificio.}
  \addfield{Descripción:}{El sistema obtiene información
    progresivamente del usuario haciendo uso de un componente que
    realiza una pregunta al usuario en varias partes del sistema.
    Este es conocido como componente de ingreso progresivo (\textit{Progressive
      Profile Widget}, PPW).}
  \addfield{Actor:}{Usuario PyME}
  \additemizedfield{Precondiciones:}{
  \item El usuario ha sido autentificado por medio de un elemento (\textit{token})
    de autenticación (\textit{Single Sign On} SSO) asignado por el sistema de PG\&E.
  \item El usuario debió haber creado previamente el perfil básico del edificio.
  }
  \addfield{Requerimientos no funcionales:}{
    Las actualizaciones a la página deben hacerse en forma
    asíncrona y sin recargar la página completa.
  }
  \addscenario{Flujo normal:}{
  \item El sistema despliega el componente PPW en la parte izquierda en las
    pantallas de plan de ahorro y de recomendaciones.
  \item Dentro del PPW, se presenta al usuario una pregunta a responder:
    \begin{enumerate}
    \item Dependiendo de la pregunta, el usuario puede seleccionar una
      respuesta de una lista desplegable, seleccionar casillas,
      responder si o no con un botón radial o ingresar un valor
      en un campo.
    \item Cada tipo de edificio tiene una lista predefinida ordenada de
      preguntas a ser desplegadas en el componente.
    \item Algunas preguntas tienen imágenes asociadas para facilitar
      al usuario su respuesta.
    \end{enumerate}
  \item El usuario puede responder la pregunta o saltar la pregunta o
    no hacer nada en el PPW.
    \begin{enumerate}
    \item Si el usuario responde:
      \begin{enumerate}
      \item La respuesta provoca un refinamiento del perfil del edificio y
        el nuevo cálculo de las recomendaciones propuestas.
        \begin{enumerate}[i.]
        \item El perfil del edificio es actualizado, las recomendaciones
          son recalculadas.
        \item El sistema despliega un indicador de progreso mientras
          se recalculan las recomendaciones.
        \item Al finalizar el refinamiento el orden de las preguntas
          o los valores del plan de ahorro deberán ser actualizados, sin recargar
          la página completa.
        \end{enumerate}
      \item El sistema despliega la siguiente pregunta.
      \item El sistema almacena la respuesta dada.
      \end{enumerate}
    \item Si el usuario elige saltar la pregunta entonces el sistema
      despliega la siguiente pregunta sin necesitar de una respuesta por el usuario.
    \end{enumerate}
  \item Desde el PPW, el usuario puede seleccionar \textquote{Building Profile} para
    ver el perfil detallado del edificio.
  }
  \additemizedfield{Postcondiciones:}{
  \item Un perfil del edificio más detallado si el usuario dio respuesta a la
    pregunta desplegada sobre su edificio.
  \item Si el usuario dio nueva información se recalculan las recomendaciones
    asociadas al edificio.
  }
\end{usecase}

\subsection{Administrar recomendaciones}

En los casos de uso anteriores se obtiene del usuario el perfil del edificio,
al menos sus características básicas. Haciendo uso de esta información
el sistema \texttt{PEAT} genera recomendaciones sobre como disminuir el consumo
de energía a través de cambios de comportamiento y/o modernización de equipo.

\vspace{2.5mm}

Las recomendaciones se muestran en tres partes del sistema:
\begin{enumerate}
\item En el plan de ahorro.
\item En las recomendaciones.
\item En el perfil detallado del edificio.
\end{enumerate}

El comportamiento general al administrar las recomendaciones en estas tres
secciones es definido en este caso de uso, teniéndose en los casos de uso
posteriores los detalles particulares del comportamiento para esa sección.

\begin{usecase}
  \addtitle{Caso de Uso:}{Administrar recomendaciones.}
  \addfield{Descripción:}{El sistema despliega una lista de todas las
    recomendaciones que se ajustan al perfil del edificio.}
  \addfield{Actor:}{Usuario PyME}
  \additemizedfield{Precondiciones:}{
  \item El usuario ha sido autentificado por medio de un elemento (\textit{token})
    de autenticación (\textit{Single Sign On} SSO) asignado por el sistema de PG\&E.
  \item El usuario debió haber creado previamente el perfil básico del edificio.
  }
  \additemizedfield{Requerimientos no funcionales:}{
  \item Cualquier acción en una recomendación debe hacerse de forma
    asíncrona y sin necesitar de recargar la página completa.
  }
  \pagebreak
  \addscenario{Flujo normal:}{
  \item El sistema despliega las recomendaciones, en el plan de ahorro,
    la lista de recomendaciones o en el perfil detallado del edificio.
  \item El usuario puede tomar las siguientes acciones:
    \begin{itemize}
    \item Añadir al plan (de ahorro).
      \begin{enumerate}
      \item Cuando el usuario selecciona \textquote{Añadir al plan}, una ventana
        modal se abre y pregunta:
        \textquote{¿ Cuando piensa completar esta acción ?}.
      \item El sistema proporciona lista predefinida de rangos de tiempo:
        una semana, un mes, tres meses, etcétera.
      \item El usuario elige el rango de tiempo adecuado y selecciona la opción
        \textquote{Añadir al plan}.
      \item El sistema añade la recomendación al plan de ahorro del edificio.
      \end{enumerate}
    \item No aplica.
      \begin{enumerate}
      \item El sistema mueve la recomendación al final de la lista de
        recomendaciones.
      \item En cualquier momento, el usuario puede cambiar el estado de la
        recomendación, es decir, añadir la recomendación al plan de ahorro
        o indicar que no es aplicable al edificio.
      \end{enumerate}
    \item Ya terminado.
      \begin{enumerate}
      \item El sistema mueve la recomendación al final de la lista de
        recomendaciones.
      \item En cualquier momento, el usuario puede cambiar el estado de la
        recomendación, es decir, añadir la recomendación al plan de ahorro
        o indicar que no es aplicable al edificio.
      \end{enumerate}
    \end{itemize}
  \item El sistema refina y reordena la lista de recomendaciones después de que
    el usuario realiza cualquiera de las acciones anteriores.
  }
  \addscenario{Flujo alternativo:}{
  \item El usuario contesta alguna pregunta en el PPW o en el perfil detallado del
    edificio.
  \item El sistema refina y reordena la lista de recomendaciones después de que
    el usuario realiza cualquiera de las acciones anteriores.
  }
  \additemizedfield{Postcondiciones:}{
  \item Si el usuario añadió una recomendación a su plan de ahorro, entonces
    se tiene un plan de ahorro actualizado.
  \item Si el usuario realizó cualquier acción en una recomendación, entonces
    la lista de recomendaciones es actualizada y reordenada.
  }
\end{usecase}

\subsection{Ver recomendación}

En el plan de ahorro, la lista de recomendaciones o en el perfil detallado
del edificio solo se despliega el título y una breve descripción de cada
recomendación. Para dar mayor información sobre la recomendación a implementar se
proporciona al usuario más información sobre los pasos para lograr su
implementación.

\begin{usecase}
  \addtitle{Caso de Uso:}{Ver recomendación}
  \addfield{Descripción:}{El sistema despliega información adicional
    sobre una recomendación para que el usuario pueda elegir
    implementarla.}
  \addfield{Actor:}{Usuario PyME}
  \additemizedfield{Precondiciones:}{
  \item El usuario ha sido autentificado por medio de un elemento (\textit{token})
    de autenticación (\textit{Single Sign On} SSO) asignado por el sistema de PG\&E.
  \item El usuario debió haber creado previamente el perfil básico del edificio.
  }
  \addfield{Requerimientos no funcionales:}{
    Cualquier acción en una recomendación debe hacerse en forma
    asíncrona y sin necesitar una recarga completa de la página.
  }
  \pagebreak
  \addscenario{Flujo normal:}{
  \item El usuario ve una recomendación que capta su atención y selecciona el título
    de la recomendación.
  \item El sistema despliega información adicional sobre la recomendación. Por
    ejemplo, (1) cuanto ahorraría al realizar tal recomendación, (2) qué se necesita
    hacer para obtener este ahorro, etcétera.
  \item El usuario puede realizar las acciones definidas en el caso
    de uso anterior, es decir, añadir la recomendación al plan de ahorro,
    indicar que no aplica dicha recomendación o bien la recomendación ha sido
    completada.
  }
  \addfield{Postcondiciones:}{
    Si el usuario realizó cualquier acción en una recomendación, entonces
    la lista de recomendaciones es actualizada y reordenada.
  }
\end{usecase}

\subsection{Ver historial de consumo}

\begin{usecase}
  \addtitle{Caso de Uso:}{Ver historial de consumo}
  \addfield{Descripción:}{Se muestra el consumo energético del edificio
    mediante un conjunto de gráficas con múltiples vistas para diferentes
    rangos de tiempo y categorías de gasto (gas, electricidad y emisiones de carbón).
  }
  \addfield{Actor:}{Usuario PyME}
  \additemizedfield{Precondiciones:}{
  \item El usuario ha sido autentificado por medio de un elemento (\textit{token})
    de autenticación (\textit{Single Sign On} SSO) asignado por el sistema de PG\&E.
  \item El usuario debió haber creado previamente el perfil básico del edificio.
  }
  \addfield{Requerimientos no funcionales:}{
    Las actualizaciones a la página deben hacerse en forma asíncrona y sin necesitar
    una recarga completa de la página.
  }
  \addscenario{Flujo normal:}{
  \item El sistema despliega inicialmente una gráfica del gasto anual de
    electricidad y gas del edificio.
  \item El sistema muestra un menú con las siguientes opciones de visualización:
    \begin{itemize}
    \item Gasto: el cual es calculado en dolares con gráficas anuales, mensuales
      y diarias.
    \item Electricidad: se refiere al cálculo en kilowatt-hora con gráficas anuales,
      mensuales y diarias.
    \item Gas: el cual es medido en termia, con gráficas anuales, mensuales y
      diarias
    \item Emisiones de carbón: medido en toneladas, con gráficas anuales, mensuales
      y diarias.
    \item Análisis: gráfica que indica el consumo estimado según la categoría.
      Las categorías son: refrigeración, iluminación, calefacción, etcétera.
    \end{itemize}
  }
  \addfield{Postcondiciones:}{
    El sistema se mantiene en el mismo estado.
  }
\end{usecase}

\subsection{Administrar plan de ahorro}

El plan de ahorro es un conjunto de recomendaciones que el usuario se ha comprometido
a llevar a cabo en su edificio. La administración de esta lista es de vital
importancia para el sistema \texttt{PEAT} puesto que permite al usuario tomar
acciones concretas para reducir sus costos de energía.

\begin{usecase}
  \addtitle{Caso de Uso:}{Administrar plan de ahorro}
  \addfield{Descripción:}{El sistema debe permitir la administración
    del plan de ahorro del edificio, el cual consiste en un conjunto de
    recomendaciones que el usuario sea ha comprometido a realizar o bien que ya ha
    terminado para reducir su gasto energético.}
  \addfield{Actor:}{Usuario PyME}
  %\additemizedfield{Precondiciones:}{
  %\item El usuario ha sido autentificado por medio de un elemento (\textit{token})
  %  de autenticación (\textit{Single Sign On} SSO) asignado por el sistema de PG\&E.
  %\item El usuario debió haber creado previamente el perfil básico del edificio.
  %}
  \addfield{Requerimientos no funcionales:}{
    Las actualizaciones a la página deben hacerse en forma asíncrona y sin necesitar
    una recarga completa de la página.
  }
  \addscenario{Flujo normal:}{
  \item El usuario no ha agregado ninguna recomendación a su plan
    \begin{enumerate}
    \item Las tres mejores recomendaciones son desplegadas dentro del plan de ahorro.
    \item El sistema elige estas recomendaciones por medio de la siguiente lógica:
      \begin{itemize}
      \item Son ordenadas según su período de recuperación (definido por el ahorro
        anual en dólares / costo en dólares, por adelantado).
      \item Recomendaciones con costo inicial cero (por ejemplo, acciones basadas
        en comportamiento) se consideran que tienen un reembolso de dos años.
      \item Las recomendaciones que han sido marcadas como completadas o que no aplican
        no son parte de la selección para el plan de ahorro.
      \end{itemize}
    \end{enumerate}
  \item El usuario a agregado una o más recomendaciones a su plan de ahorro.
    \begin{itemize}
    \item La mejor recomendación es desplegada debajo del componente PPW.
    \item El sistema despliega una lista de las recomendaciones que el usuario
      a agregado a su plan.
      \begin{itemize}
      \item El sistema debe indicar que la lista de recomendaciones es el plan
        de ahorro.
      \item El usuario puede agregar más recomendaciones desde la lista de
        recomendaciones, cuando se añade una de estas solamente es
        visible en el plan de ahorro ya no debe de aparecer en la lista de
        recomendaciones
      \item El usuario puede remover recomendaciones de su plan seleccionando
        \textquote{Remover del plan} en un menú desplegable al lado del nombre de la
        acción.
      \end{itemize}
    %\item El usuario puede ordenar la lista por nombre, categoría, costo, ahorro
    %  o reembolso.
    \item El usuario puede obtener más información al seleccionar una recomendación.
    \item El usuario actualiza el estado de una recomendación a
      \textquote{Completada} o \textquote{Remover}.
      \begin{itemize}
      \item Las recomendaciones dentro del plan de ahorro solo pueden tener
        dos posibles estados: \textquote{Completada} y \textquote{Remover},
        el estado base de una recomendación agregada al plan es de
        \textquote{Añadida}.
      \item Las recomendaciones con estado de \textquote{Completada} permanecen
        en el plan de ahorro con la selección \textquote{Completada} desplegada
        al lado del nombre de la acción.
      \end{itemize}
    \end{itemize}
  }
  \addfield{Postcondiciones:}{
    Si el usuario realizó cualquier acción en una recomendación entonces el plan
    de ahorro es actualizado y reordenado.
  }
\end{usecase}

\section{Interfaz de usuario}

Dados los requerimientos funcionales y no funcionales presentados en los
casos de uso se encuentra que la interfaz de usuario debe tener las
siguientes características:

\begin{itemize}
\item La interfaz debe ser simple y enfocada a la información que el
  usuario necesita.
\item La interfaz debe permitir realizar la mayoría de sus
  acciones de forma asincrónica, sin necesidad de una recarga completa de
  la página.
\item Toda interacción debe completarse en menos de un segundo.
\end{itemize}

Para cumplir estos requerimientos fue necesario una cantidad considerable
de código Javascript en la parte del \textit{frontend}, para realizar peticiones
al servidor de forma asíncronas por medio de AJAX, y realizar las actualizaciones
a la vista después de recibir la respuesta del servidor.

Para permitir el reuso de código para la implementación de funcionalidad como
el \textit{componente de ingreso progresivo} era necesario dar estructura a los
componentes del \textit{frontend}, para este fin se eligió utilizar la biblioteca
\textit{Backbone} pues permite definir una arquitectura MVC, es decir, permite
definir modelos, vistas y controladores en la parte del \textit{frontend}.

\subsection{Descripción general y navegación}

La interfaz de usuario en el sistema \texttt{PEAT} está compuesta por ocho páginas
de las cuales tres son páginas de PG\&E y las restantes son gestionadas por
\texttt{PEAT}.

Todas las páginas del sistema muestran un menú de navegación que permite al
usuario navegar directamente a las páginas \textquote{Plan de ahorro},
\textquote{Recomendaciones} y \textquote{Perfil detallado del edificio}.

La descripción general de las páginas es la siguiente:

\begin{enumerate}
\item Un usuario ingresa al portal de PG\&E, luego a la página de ingreso del
  sistema \textit{MyEnergy} de PG\&E, en este sistema es donde los usuarios PyME
  realizan la administración de su perfil y cuentas con PG\&E. En el tablero de
  \textit{MyEnergy} se tiene una liga que invita al usuario a conocer sobre
  medidas para ahorrar, esta liga dirige al usuario al sistema \texttt{PEAT}
  (Ver Figura 3.3 1).
\item En el primer ingreso al sistema se dirige al usuario a la página
  \textquote{Crear perfil del edificio} (Ver Figura 3.3 2), aquí el usuario
  solo contesta las preguntas básicas sobre su edificio
  (Ver caso de uso \textit{Crear perfil del edificio}).
  El usuario regresa a esta página cuando requiere agregar más edificios a su cuenta.
  Al finalizar la creación del perfil se redirige al usuario a la página de
  \textquote{Plan de ahorro}.
\item La página de \textquote{Plan de ahorro} es la página principal del sistema
  \texttt{PEAT}, es decir, la página inicial (cuando el usuario cuenta por lo
  menos con un un perfil de edificio) al ingresar por medio de su tablero en
  \textit{MyEnergy} (Ver Figura 3.3 3).
  El plan de ahorro presenta la siguiente funcionalidad al usuario:
  \begin{itemize}
  \item Parte superior: se proporciona al usuario capacidad para interactuar
    con la facturación de sus gastos de energía por medio de gráficas comparativas
    (Ver caso de uso \textit{Ver historial de consumo}).
  \item Parte inferior: se muestra al usuario su plan de ahorro, es decir,
    las recomendaciones que se ha comprometido a realizar para reducir su gasto
    (Ver caso de uso \textit{Administrar plan de ahorro}).
  \end{itemize}
  El usuario puede navegar a la página de \textquote{Detalle de recomendación}
  mostrada en su plan de ahorro.
\item En la página de \textquote{Recomendaciones} se tienen dos componentes
  (Ver Figura 3.3 4):
  \begin{itemize}
  \item El componente de ingreso progresivo que invita al usuario a
    dar mas información sobre su edificio (Ver caso de uso \textit{Obtener
      información del edificio}).
  \item Una lista de recomendaciones para reducir sus costos de energía
    basados en el perfil de su edificio (Ver caso de uso \textit{Administrar
      recomendaciones}).
  \end{itemize}
  El usuario puede navegar a la página de \textquote{Detalle de recomendación}
  de las recomendaciones mostradas en la lista de recomendaciones.
\item La página de \textquote{Perfil detallado del edificio} (Ver Figura 3.3 5) es
  donde el usuario puede responder o actualizar las respuestas a las preguntas
  sobre su edificio (Ver caso de uso \textit{Administrar perfil del edificio}).
\item La página de \textquote{Detalle de recomendación} (Ver Figura 3.3 6) muestra
  los pormenores sobre las medidas que se necesitan realizar para lograr reducir
  el consumo en el edificio (Ver caso de uso \textit{Ver recomendación}).
\end{enumerate}

\rjcimage{0.7}{imagenes/UI-Navegacion.png}{Diagrama de alto nivel de
  la navegación en el sistema \texttt{PEAT}.}{ui-navegation}

\subsection{Encabezado general}

Todas las páginas del sistema \texttt{PEAT} comparten el mismo encabezado,
como se muestra en la Figura \ref{fig:global-header}, el cual tiene las
siguientes secciones:

\begin{enumerate}[a)]
\item Componente de selección de edificios: este control muestra un menú
  desplegable con todos los edificios asociados al usuario.
\item Nombre del sistema: el nombre es \textquote{Chequeo de energía del negocio}
  (\textit{Business Energy Checkup}), el nombre interno del sistema es \texttt{PEAT}.
\item Menú de navegación del sistema.
  \begin{itemize}
  \item Plan de ahorro (\textit{Energy Plan}): dirige al usuario a la página de
    \textquote{Plan de ahorro}.
  \item Formas de ahorrar (\textit{Ways to Save}): dirige al usuario a la página de
    \textquote{Recomendaciones}.
  \item Perfil de negocio (\textit{Business Profile}): dirige al usuario a la página
    de \textquote{Perfil detallado del edificio}.
  \item Tomar un recorrido (\textit{Take a tour}): se muestra al usuario
    las diferentes páginas y secciones del sistema.
  \end{itemize}
\item Título de la página actual.
\item Componente de potencial de ahorro: este control muestra en todo momento
  el potencial de ahorro del edificio y el ahorro que se obtiene por el plan
  de ahorro del edificio.
\end{enumerate}

\rjcimage{1.0}{imagenes/Disposicion-Encabezado.png}{Encabezado principal
  del sistema \texttt{PEAT}.}{global-header}


\subsection{Crear perfil del edificio}

La página para \textquote{Crear perfil del edificio} es la página inicial del
sistema cuando el usuario no tiene ningún perfil del edificio asociado, esta página
también es usada cuando el usuario agrega más edificios a su cuenta.

Esta página tuvo varias iteraciones por la retroalimentación obtenida por las
pruebas con los usuarios, en las primeras iteraciones la página requería que el
usuario contestara toda la información requerida según el caso de uso
\textit{Crear perfil del edificio} como se muestra en la Figura 3.5, lo cual
representaba mucho trabajo al usuario.

\jcimagefull{1.0}{imagenes/SBP-Wireframe.png}{Versión inicial para crear
  un perfil de edificio.}

Para facilitar al usuario el ingreso de información se decidió implementar
las siguientes mejoras:

\begin{enumerate}
\item Se redujo el número de opciones para el tipo de industria, inicialmente
  esta lista era numerosa lo que implicaba mas trabajo al usuario, se redujo
  a cinco opciones que son visibles inmediatamente.
\item Se dividió el ingreso de información en dos partes.
\end{enumerate}

En la versión final el sistema inicialmente pide al usuario seleccionar el tipo
de industria que mejor describe su negocio, las opciones son: comercial, industrial,
agrícola, multifamiliar y otro como se puede ver en la Figura 3.6.

\rjcimage{0.75}{imagenes/SBP-Modal1.png}{Se pide al usuario
  el seleccionar el tipo de industria de su negocio.}{sdp-modal1}

Después el sistema pide al usuario seleccionar su cuenta y dirección
de servicio, el alias del edificio, el tipo de edificio y su área
como se puede ver en la Figura 3.7.

\rjcimage{0.6}{imagenes/SBP-Modal2.png}{Se pide al usuario ingresar
  las características básicas de su edificio.}{sdp-modal2}

Al ingresar esta información el usuario es redirigido a la página de
\textquote{Plan de ahorro} del perfil creado.

\subsection{Plan de ahorro}

La página de \textquote{Plan de ahorro} es la página principal del sistema
\texttt{PEAT} dado que implementa la funcionalidad principal del sistema
pues tiene la capacidad para interactuar con los datos sobre los gastos de
energía y la administración del plan de ahorro del edificio.

La página está dividida en dos secciones principales, en la sección superior
se da el manejo de las gráficas para visualizar los gastos de energía y en la
sección inferior se tiene la administración del plan de ahorro.

\subsubsection{Visualización}

La sección superior consta de las siguientes partes:
\begin{itemize}
\item Menú de vistas: En este menú el usuario puede seleccionar
  que tipo de datos se despliega en las gráficas. Se tienen cuatro tipos
  de vista: gasto (\textit{Spending}), electricidad (\textit{Electricity}), gas
  (\textit{Gas}), emisiones de carbono (\textit{Carbon Emissions}) y análisis
  (\textit{Energy Use}). También en este menú en la parte derecha se
  tiene un estimado del ahorro que se puede obtener en el edificio
  si se implementan las recomendaciones (Ver Figura 3.8).
  \jcimage{1.0}{imagenes/Barra-Vistas-Ahorro.png}{Menú de vistas.}
\item Menú de rango de tiempo: Esta sección se encuentra enseguida del menú de
  vistas, aquí el usuario puede seleccionar entre los rangos de tiempo anual
  (\textit{Annually}), mensual (\textit{Monthly}) o diario (\textit{Daily}) de la
  vista seleccionada, en ciertas vistas se tienen opciones extra en la parte
  derecha como por ejemplo agregar datos de clima o de periodos anteriores
  a la vista actual (Ver Figura 3.9).
  \jcimage{1.0}{imagenes/Barra-Rango.png}{Menú de rango de tiempo.}
\item Área de visualización: En esta sección se despliega los datos de consumo
  del edificio mostrando los datos y rango de tiempo que el usuario ha seleccionado
  en los dos menús anteriores (Ver Figura 3.10).
  \jcimage{1.0}{imagenes/Grafica.png}{Visualización del gasto de electricidad.}
\end{itemize}

Para la implementación de las gráficas se utilizó el marco de trabajo
Ext JS, este marco de trabajo está diseñado para implementar aplicaciones
de tipo \textit{SPA} por medio del lenguaje de programación Javascript.
Ext JS implementa un gran número de controles de interfaz como paneles,
barras de herramientas, pestañas, etcétera.

Ext JS es una elección natural para el sistema \texttt{PEAT} dado que tiene un
excelente paquete de graficación el cual permite presentar visualmente una gran
cantidad de datos por medio de una amplia gama de gráficas: líneas, barras,
circulares, etcétera. Otro punto a favor de Ext JS es que el equipo de programadores
del \textit{frontend} estaba ampliamente familiarizado con este marco de trabajo.

Dada la capacidad de Ext JS para crear interfaces se tenía el plan inicial
de implementar toda la interfaz con este marco de trabajo sin embargo Ext JS tiene
una gran desventaja ya que requiere de tiempo para inicializar sus componentes
demasiado grande, (mas de un segundo), lo cual se contrapone a los requerimientos
del cliente.
Así se decidió hacer uso solamente del módulo de graficación de Ext JS y
hacer la implementación de los controles necesarios para la interfaz desde cero
haciendo uso de \texttt{Backbone}.

Para cada rango de tiempo se utiliza un tipo de gráfico y/o visualización diferente:

\begin{itemize}
\item Anual: se hace una comparativa entre el gasto del edificio en los últimos doce
  meses, el promedio de gasto de un edificio de características similares y
  el gasto de un edificio eficiente en su consumo energético. Para hacer estas
  comparaciones se utiliza gráficas de barras como se puede ver en la Figura 3.11.
  \jcimage{1.0}{imagenes/Vista-Anual.png}{Visualización anual del gasto de
    electricidad.}
\item Mensual: se muestra una gráfica de líneas del gasto del edificio
  en los últimos 12 meses, además permite comparar con el gasto del año anterior
  como se puede observar en la Figura 3.12. Además se pueden agregar los datos sobre
  el clima durante el período que se está visualizando.
  \jcimage{0.9}{imagenes/Vista-Mensual.png}{Visualización mensual del gasto de
    electricidad.}
\item Diario: Se muestra una gráfica similar a la vista mensual, pero en este caso
  se visualizan los últimos treinta días de gasto como se muestra en la Figura 3.13.
  \jcimage{0.9}{imagenes/Vista-Diaria.png}{Visualización diaria del gasto de
    electricidad, con la comparativa al gasto del año pasado y datos del
    clima.}
\item Análisis: Se muestra un análisis del gasto del edificio en categorías
  por medio de una gráfica circular como se puede observar en la Figura 3.14, este
  análisis es realizado usando aprendizaje de máquina por lo que este análisis es
  solo una estimación.
  \jcimage{0.9}{imagenes/Vista-Analisis.png}{Visualización del análisis del gasto de
    electricidad.}
\end{itemize}

Aunque las figuras solo muestran el caso del gasto en electricidad las
visualizaciones son idénticas para el gasto en gas.

\subsubsection{Administración}

La sección inferior como se muestra en la Figura 3.15 consta de las siguientes partes:
\begin{itemize}
\item Componente de ingreso progresivo: en este se despliega una pregunta
  sobre el edificio del usuario, las características de este componente
  se verán en detalle en la sección \ref{subsec:recomendaciones}.
\item Componente lista de recomendaciones: en este componente se despliega la
  lista de recomendaciones que forman el plan de ahorro del edificio,
  las características de este componente se verán a detalle en la sección
  \ref{subsec:componentes}.
  \jcimage{0.9}{imagenes/Plan-Energia-Inicial.png}{Plan de energía inicial con
    recomendaciones.}
\end{itemize}

\subsection{Recomendaciones}
\label{subsec:recomendaciones}

La página de \textquote{Recomendaciones} muestra al usuario una lista de
sugerencias acerca de las medidas que se pueden realizar en el edificio
para lograr disminuir el consumo energético del edificio. En esta lista se
tienen todas las recomendaciones aplicables que no están ya en el plan
de ahorro del edificio.

La página tiene dos secciones: la sección lateral contiene el componente de
ingreso progresivo y en la sección principal se tiene el componente lista
de recomendaciones.

\subsubsection{Componente de lista de recomendaciones}

El componente de lista de recomendaciones consta de las siguientes secciones:

\begin{itemize}
\item Encabezado: se muestran las características principales por las cuales
  el usuario puede ordenar la lista ya sea por nombre, categoría, costo, ahorro
  y recuperación de la inversión, como se puede ver en la Figura 3.16.
  \jcimage{1.0}{imagenes/Recomendaciones-Encabezado.png}{Encabezado de la lista
    de recomendaciones.}
\item Renglón: por cada recomendación en la lista se despliegan las
  características principales definidas en el encabezado, además se tiene un botón
  que permite al usuario cambiar el estado de la recomendación.
  Para cada recomendación se tiene las siguientes secciones como se puede
  ver en la Figura 3.17:
  \begin{enumerate}[a)]
  \item Imagen representativa de la recomendación.
  \item Descripción de una línea de la recomendación.
  \item Categoría de la recomendación.
  \item Costo de implementar la recomendación.
  \item Ahorro que trae la recomendación en un año.
  \item Tiempo para recuperar la inversión, es decir, cuanto tiempo el ahorro
    obtenido al implantar la recomendación es igual o superior al costo
    de su implementación.
  \item Breve descripción de las medidas necesarias para implementar la
    recomendación.
  \item Componente añadir al plan, este componente permite al usuario definir el
    estado de la recomendación, es decir, si la recomendación es parte
    del plan de ahorro, ya fue implementada o no aplica al edificio.
  \end{enumerate}
  \jcimage{1.0}{imagenes/Recomendaciones-Renglon.png}{Las secciones de una
    recomendación.}
\end{itemize}

\subsubsection{Componente añadir al plan de ahorro}

El componente \textquote{Añadir al plan} permite al usuario cambiar el estado
de una recomendación, los estados posibles son:

\begin{itemize}
\item Por implementar: en este estado la recomendación no es parte
  del plan de ahorro del edificio, este hace referencia al estado inicial de
  las recomendaciones.
\item Añadido: este estado indica que el usuario ha elegido implementar la
  recomendación y por lo tanto es parte del plan de ahorro del edificio.
\item No aplicable: este estado indica que la recomendación no es aplicable al
  edificio.
\item Completada: se refiere a la recomendación terminada la cual ya fue implementada
  con éxito en el edificio, ésta se considera parte del plan de ahorro.
\end{itemize}

Mediante la interacción con este componente el usuario modifica el estado
de una recomendación. Este componente tiene dos formas de interacción:
\begin{itemize}
\item Presionando su parte izquierda: se realiza la acción de agregar la
  recomendación al plan de ahorro.
\item Presionando su parte derecha: se despliega un menú de acciones
  secundarias que dependen del estado actual de la recomendación.
\end{itemize}

Como se indico anteriormente para realizar la transición del estado
\textquote{Por implementar} al estado \textquote{Añadido} el usuario solo tiene
que presionar el control en su parte izquierda como se puede ver en las Figuras
\ref{fig:boton-inicial} y \ref{fig:boton-final}.

\rjcimage{0.4}{imagenes/Boton-Plan-Inicial.png}{Vista inicial del control
  de añadir al plan.}{boton-inicial}
\rjcimage{0.4}{imagenes/Boton-Plan-Anadido.png}{Vista cuando la recomendación
  a sido añadida al plan.}{boton-final}

El sistema despliega una ventana como se puede ver en la Figura
\ref{fig:estimacion-recomendacion} preguntando al usuario una estimación
del tiempo en que se podrá implementar las medidas propuestas por la
recomendación que se esta agregando al plan de ahorro.
\rjcimage{0.5}{imagenes/Recomendaciones-Estimacion.png}{Al agregar una
  recomendación al plan el sistema pregunta el tiempo estimado para completar
  la recomendación.}{estimacion-recomendacion}

Para realizar otras acciones sobre la recomendación el usuario presiona la parte
derecha del componente en la parte donde se encuentra un triángulo que indica la
presencia de un menú despegable.
Estando la recomendación es un estado inicial \textquote{Por implementar} las
acciones en el menu despegable son para indicar que se ha completado la recomendación
(estado \textquote{Completada}) o que no es aplicable
(estado \textquote{No aplicable}) como se puede observar en la Figura
\ref{fig:menu-anadir-inicial}.

\rjcimage{0.5}{imagenes/Boton-Plan-Menu-Inicial.png}{Las acciones de
  completar o indicar no aplicable en el menú despegable.}{menu-anadir-inicial}

Cuando la recomendación se encuentra en el estado de \textquote{Añadido} las
acciones en el menú despegable son las relacionadas con completar, remover o indicar
que no es aplicable a la recomendación como se muestra en la Figura
\ref{fig:menu-anadir-anadido}.

\rjcimage{0.35}{imagenes/Boton-Plan-Menu-Anadido.png}{Las acciones de
  completar, remover o indicar no aplicable en el menú despegable.}{menu-anadir-anadido}

\subsection{Detalle de recomendación}

La página de \textquote{Detalle de recomendación} describe de forma detallada la
recomendación y las medidas que se tienen que implementar para lograr reducir
el consumo energético del edificio.

La página está dividida en las siguientes secciones:
\begin{itemize}
\item Encabezado: se muestra una descripción breve de la recomendación y a su
  derecha se tiene el componente \textquote{Añadir al plan} como se muestra
  en la Figura \ref{fig:recomendacion-encabezado}.
  \rjcimage{1.0}{imagenes/Recomendacion-Titulo.png}{Encabezado de la recomendación
    para la instalación de un sistema de paneles solares.}{recomendacion-encabezado}
\item Descripción: se describe a detalle la información sobre las medidas
  necesarias para reducir el consumo energético del edificio.
\item Preguntas:  Como se puede observar en la Figura
  \ref{fig:recomendacion-descripcion} después de la descripción se pueden
  tener una o varias preguntas al usuario para obtener mas datos con el fin de
  establecer una mejor estimación sobre los costos y ahorro esperados al completar
  la recomendación.
  En la sección lateral se despliega el resultado del cálculo estimado
  de costos y ahorro al completar la recomendación, estos datos son
  actualizados de forma asíncrona cuando el usuario contesta o cambia
  la respuesta a alguna de las preguntas presentadas en el contenido
  principal.
  \rjcimage{1.0}{imagenes/Recomendacion-Descripcion.png}{Descripción detallada
    de la recomendación, con la estimación de costo y ahorro en la sección lateral.}{recomendacion-descripcion}
\end{itemize}

\subsection{Perfil detallado del edificio}

La página de \textquote{Perfil detallado del edificio} permite al usuario el ingreso
o actualización de las características de su edificio por medio de un
conjunto de preguntas y respuestas divididas en categorías, como se puede
observar en la Figura \ref{fig:perfil-categorias}.

La página se encuentra dividida en varias secciones\footnote{Una sección por
  categoría}:
\begin{itemize}
\item Edificio (\textit{Building}).
\item Iluminación (\textit{Lighting}).
\item Calefacción, ventilación y aire acondicionado (\textit{Heating, Ventilation and Air Conditioning} HVAC).
\item Refrigeración (\textit{Refrigeration}).
\item Cocina (\textit{Cooking}).
\item Calefacción de agua (\textit{Water-Heating}).
\item Miscelánea (\textit{Misc}).
\end{itemize}

Varias categorías como cocina y refrigeración solo son mostradas
para ciertos tipos de edificios.

\rjcimage{1.0}{imagenes/Perfil-Secciones.png}{Las preguntas y respuestas sobre
  el edificio se organizan en categorías.}{perfil-categorias}

El usuario selecciona una categoría seleccionando el título de una categoría,
para que esa sección presente las preguntas y respuestas asociadas a ésta.

Cada sección está dividida, como se muestra en la Figura \ref{fig:perfil-seccion},
en las siguientes partes:
\begin{enumerate}
\item Encabezado de la sección: cada encabezado tiene un ícono característico,
  el nombre de la categoría y en la parte derecha un marcador indica
  si todas las preguntas de la sección han sido respondidas por el usuario
  (Ver Figura \ref{fig:perfil-seccion} 1).
\item Preguntas y respuestas: las preguntas son breves y se refieren a una
  característica particular del edificio. La mayoría de las preguntas
  tiene respuestas definidas\footnote{Se cuenta solo con un reducido número de
    preguntas donde se espera el ingreso de un valor numérico}. Las respuestas se
  presentan de las siguientes formas (Ver Figura \ref{fig:perfil-seccion} 2):
  \begin{enumerate}[a)]
  \item Menú desplegable: para preguntas que tienen mas de tres respuestas
    posibles excluyentes.
  \item Casillas de selección múltiple: para preguntas en que las respuestas
    pueden ser validas al mismo tiempo.
  \item Campo de texto limitado (solo un valor numérico): para las preguntas
    en que se necesita un valor numérico exacto.
  \item Botón de opción (con texto o imagen): para preguntas que tiene un
    número limitado de respuestas posibles. Se hace uso de imágenes para las
    preguntas más técnicas para facilitar al usuario el contestar la pregunta.
  \end{enumerate}
\item Botón siguiente: para progresar a la siguiente categoría (Ver Figura \ref{fig:perfil-seccion} 3).
\end{enumerate}

\rjcimage{0.85}{imagenes/Perfil-Preguntas.png}{Las preguntas de las secciones
  Edificio y HVAC.}{perfil-seccion}

\subsection{Componentes reusables}
\label{subsec:componentes}

En \texttt{PEAT} se tienen varios componentes como el componente de ingreso
progresivo o el componente para añadir al plan de ahorro cuya presencia es necesaria
en varias de las páginas del sistema por la funcionalidad que proporcionan.
Para permitir el reuso de código y diseño de estos componentes se decidió hacer uso
de la biblioteca \texttt{Backbone}, dado que permite dar estructura (definiendo
modelos, vistas y controladores) al implementar la funcionalidad requerida.

Los componentes que se implementaron son los siguientes:
\begin{itemize}
\item Componente de ingreso progresivo.
\item Componente del potencial de ahorro.
\item Componente para lista de recomendaciones.
\item Componente para añadir al plan de ahorro.
\end{itemize}

\subsubsection{Componente de ingreso progresivo}

Este componente muestra al usuario una pregunta y sus posibles respuestas sobre el
perfil del edificio. Este componente es usado en las páginas de
\textquote{Plan de ahorro} y de \textquote{Recomendaciones}.

El componente, como se puede ver en la Figura \ref{fig:ingreso-progresivo}, tiene
las siguientes partes:
\begin{enumerate}[a)]
\item Pregunta y respuesta: se despliega una sola pregunta y
  las respuestas se presentan según los tipos de respuesta expuestos
  en el perfil del edifico.
\item Botón saltar (\textit{Skip}): permite al usuario omitir dicha pregunta
  mostrada y pasar a la siguiente pregunta sin contestar del perfil
  de su edificio.
\item Botón siguiente (\textit{Next}): permite al usuario enviar la respuesta
  seleccionada de la pregunta mostrada.
\item Liga al perfil detallado: permite al usuario navegar al perfil
  detallado de su edificio.
\end{enumerate}
\rjcimage{0.5}{imagenes/PPW.png}{Componente de ingreso progresivo.}{ingreso-progresivo}

\subsubsection{Componente del potencial de ahorro}

Este componente permite al usuario conocer el potencial de ahorro del edificio y su
ahorro esperado establecido en su plan de ahorro.

\subsubsection{Componente lista de recomendaciones}

Este componente lista las recomendaciones que se pueden realizar en un edificio para
reducir sus gastos energéticos, la descripción detallada de este componente se
encuentra en la Sección \ref{subsec:recomendaciones}.

\subsubsection{Componente para añadir al plan de ahorro}

Este componente permite modificar el estado de una recomendación,
la descripción detallada de éste se encuentra en la Sección
\ref{subsec:recomendaciones}. Este componente es usado en las páginas de
\textquote{Plan de ahorro}, \textquote{Recomendaciones} y
\textquote{Detalle recomendación}.
