\chapter*{Introducción}
\markboth{INTRODUCCIÓN}{INTRODUCCIÓN}
\addcontentsline{toc}{chapter}{Introducción}
El presente trabajo aborda el diseño e implementación del sistema \textit{Progressive
  Energy Audit Tool}\ (\texttt{PEAT}), el cual es un sistema web que permite
a los usuarios de pequeñas y medianas empresas (PyMES) identificar y monitorear
sus gastos de energía eléctrica, con el fin de obtener un mayor control
en los costos de ésta.

Este trabajo se realizó siendo parte de la empresa C3 Energy en el cargo de Senior
Software Engineer por un período de 10 meses, de Mayo de 2012 a Febrero de 2013.

En una gran cantidad de proyectos resulta ser suficiente el hacer uso de un
único lenguaje de programación para llevar a buen término el proyecto, es decir,
terminarlo de forma exitosa. Sin embargo conforme pasa el tiempo, las necesidades
de las compañías y usuarios cambian, lo cual provoca que hagamos uso de una serie de
tecnologías no previstas en un inicio, lo que representa casi siempre el uso
de más de un lenguaje de programación\footnote{Debido a que ningún
lenguaje de programación actualmente es el mejor en todos los contextos
posibles de uso.}.

El proyecto de implementación del sistema PEAT cuenta con tres partes principales
de desarrollo: (a) el \textit{backend}  implementado en su mayoría usando
el lenguaje de programación Javascript, con un núcleo escrito en el lenguaje de
programación Java, (b) una biblioteca en el lenguaje de programación Ruby que permite
la interacción del \textit{backend} con aplicaciones web y (c) el \textit{frontend}
escrito con Ruby, dada sus ventajas para realizar sistemas web.

Durante este proyecto se implementó una biblioteca no solo para hacer uso de
los servicios web ya existentes, sino para facilitar la creación
de clases y objetos en Ruby basados en la jerarquía de clases definida
en Javascript. Cabe mencionar que haciendo uso de una de las fortalezas de Ruby
como lo es la metaprogramación para la creación de clases y sus
atributos ``al vuelo'', según las especificaciones enviadas por el servidor,
se mantiene así la jerarquía de clases de un lenguaje a otro.

Otro punto de creciente importancia es el despliegue de una aplicación, ya que por un
lado hace algunos años las aplicaciones web se desarrollaban bajo el modelo de
tres capas: la base de datos, el servidor de la aplicación y el servidor web;
actualmente se tienen más servicios que deben estar en línea sobre todo para
garantizar un servicio concurrente; repositorios de memoria, balanceadores de carga,
servidores de cola, etcétera. Por lo que automatizar el despliegue de una aplicación
en sus diferentes contextos, desarrollo, producción y pruebas, es de vital
importancia para el desarrollo de software en tiempo y en forma.

Durante este proyecto se hace uso extensivo de técnicas como la metaprogramación
y el despliegue continuo, para permitir un desarrollo acelerado, con el fin de
obtener rápidamente retroalimentación del usuario final.

\section*{Objetivo general}

El objetivo general del sistema \texttt{PEAT} es dar la mayor utilidad posible
a usuarios PyMES con la menor información disponible, fomentando en el usuario
el compartir más información sobre su empresa, obteniendo un mejor control
acerca de su consumo energético.

El sistema debe además permitir el ingreso progresivo de información, por medio
de una serie de preguntas específicas al usuario, dando mejores recomendaciones
para bajar su consumo energético conforme el sistema obtiene mas información.

A partir de la información obtenida el sistema debe permitir el monitoreo
y revisión del consumo energético de forma detallada, ya sea en horas,
días, meses y años.

\section*{Objetivos secundarios}

Los objetivos secundarios que apoyan al objetivo general de este trabajo son:
\begin{itemize}
\item Implementación de una interfaz que permita la obtención
  de información del usuario de una forma eficaz y sencilla.
\item Dar información útil aunque el usuario solo proporcione el
  mínimo de información sobre su empresa.
\item Proporcionar recomendaciones para disminuir sus
  gastos en energía con base en el consumo e información proporcionada
  hasta el momento.
\item Autentificar a los usuarios mediante el uso de credenciales de acceso
  obtenidas en el portal web de Pacific Gas and Electric Company (PG\&E).
\item Diseñar e implementar un conjunto de pruebas unitarias, funcionales
  y de integración para los módulos críticos del sistema.
\item Soportar por lo menos a mil usuarios concurrentes.
\item Implementar la infraestructura para el despliegue continuo de la
  aplicación, permitiendo una retroalimentación contínua sobre el
  funcionamiento del sistema.
\end{itemize}

\section*{Organización del trabajo}

Este trabajo está dividido en cuatro capítulos los cuales son:
\begin{itemize}
\item Capítulo 1 titulado, \textit{Situación original}: se presenta una descripción
  del contexto que dio pie al desarrollo del sistema \texttt{PEAT}, después se da una
  descripción de la arquitectura y sistemas con los que se tuvo como punto de
  partida para su implementación. Finalmente se define la propuesta de desarrollo
  del sistema.
\item Capítulo 2 que lleva por nombre, \textit{Fundamentos teóricos}: se expone un
  resumen de los conceptos teóricos mas influyentes en el diseño e implementación del
  sistema \texttt{PEAT}.
\item Capítulo 3, \textit{Diseño e implementación del sistema PEAT}: se desarrolla y
  describe la implementación del sistema \texttt{PEAT}, dando la justificación
  de las decisiones tomadas durante su implementación.
\item Capítulo 4, \textit{Despliegue del sistema}: se expone la infraestructura
  del ambiente de producción y la implementación de un proceso de despliegue
  continuo  del sistema \texttt{PEAT} y las decisiones tomadas para obtener un
  rendimiento óptimo del sistema.
\end{itemize}

El sistema \texttt{PEAT} es resultado del trabajo en conjunto de tres compañías:
Pacific Gas and Electric Company (PG\&E), C3 Energy y Software Next Door.
Cabe mencionar que PG\&E es una compañía proveedora de gas natural y electricidad, una de las
más grandes compañías de Estados Unidos con sede en San Francisco, California.
El sistema \texttt{PEAT} es el resultado de una licitación iniciada por PG\&E,
siendo ganadora de dicha licitación la compañía C3 Energy.
