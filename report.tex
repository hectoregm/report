\documentclass[letterpaper,twoside,openright]{book}
%\usepackage[left=4cm,right=4cm,top=4cm,bottom=4cm,letterpaper]{geometry}
\usepackage[spanish]{babel}
\usepackage[utf8]{inputenc}
\usepackage{graphicx}
\usepackage{mathptmx}

%\author{Héctor E. Gómez Morales}
%\title{Reporte Trabajo Profesional: Sistema PEAT (Progressive Energy
%  Audit Tool)}

\begin{document}

\thispagestyle{empty}

\frontmatter

\begin{minipage}{.3\textwidth}
  \flushleft
  \center{\includegraphics[scale=.09]{imagenes/unam.pdf}}

  \vspace{20pt}

  \center{
    \rule{.5pt}{.6\textheight}
    \hspace{7pt}
    \rule{2pt}{.6\textheight}
    \hspace{7pt}
    \rule{.5pt}{.6\textheight}
  } \\

  \center{\includegraphics[scale=.22]{imagenes/ciencias.pdf}}
\end{minipage}
\begin{minipage}{.7\textwidth}
\flushright

\center{

  \center{
    \LARGE{U}\large{NIVERSIDAD} \LARGE{N}\large{ACIONAL} 
    \LARGE{A}\large{UTÓNOMA} \\[10pt]
    \large{DE} 
    \LARGE{M}\large{ÉXICO} 
  } \\
  \rule{\textwidth}{2pt}
  \\
  \hrulefill\\[1cm]
  
  \LARGE{F}\large{ACULTAD DE } \LARGE{C}\large{IENCIAS}\\[2cm]

  \large{
Diseño e implementación del sistema PEAT (Progressive Energy Audit Tool)  }\\[2cm]

  \huge{
REPORTE DE TRABAJO PROFESIONAL  }\\[1cm]

  \large{QUE PARA OBTENER EL TÍTULO DE:}\\[1cm]

  \large{
Licenciado en Ciencias de la Computación  }\\[1cm]

  \large{PRESENTA:}\\[1cm]

  \large{
Héctor Enrique Gómez Morales  }\\[1cm]

  \large{
TUTORA  }\\[1cm]

  \large{
Karla Ramírez Pulido  }
}

\end{minipage}

%\maketitle
\tableofcontents

\chapter{Introducción}
\section*{Progressive Energy Audit Tool (PEAT)}
Este trabajo aborda el diseño e implementación del sistema \textit{Progressive
  Energy Audit Tool}\ (PEAT), el cual es un sistema web que permite a los usuarios
de pequeñas y medianas empresas (PyMES) identificar y monitorear sus gastos
de energía eléctrica, con el fin de obtener un mayor control en los costos de ésta.

Este sistema fue desarrollado siendo parte de la compañía
\textbf{Software Next Door} en el cargo de \textbf{Senior Software Engineer}.

Pacific Gas and Electric Company (PG\&E) es una compañía proveedora de gas natural
y electricidad, una de las más grandes compañías de Estados Unidos con sede en
San Francisco, California. El sistema PEAT es el resultado de una licitación
iniciada por PG\&E para el desarrollo de un sistema enfocado en usuarios PyMES.

Su desarrollo es de importancia crítica tanto para PG\&E, como para el gobierno de
California puesto que su correcto funcionamiento es un requisito en una nueva
ley de facturación de energía eléctrica desde el 2013. La nueva ley buscaba
lograr distribuir la carga de la red eléctrica, siendo una de las formas
principales para lograr esto el desincentivar el uso de la red
eléctrica en horas pico al así darle la facultad a las compañías como PG\&E
de cobrar tasas mucho más altas en estas horas.

PG\&E había realizado el despliegue de medidores inteligentes en la
mayor parte de sus clientes, por lo que se tenía acceso a información
muy detallada del consumo de energía de éstos.

Actualmente para darle el mayor valor e información a las empresas es necesario
el obtener más datos de su entorno de operación; por ejemplo el número
de edificios asociados a la cuenta, rubro de la empresa, número
de empleados, etcétera. De esta forma, entre mas detalles se pueda captar sobre
la empresa el sistema tendrá más facilidad en obtener un desglose más robusto,
completo y útil de sus consumos de energía.

\mainmatter

\chapter{Situación actual}

\section{Contexto}
De 2000 a 2002 se tuvo una crisis energética en el estado de California en los
Estados Unidos, esta crisis provoco que el gobierno del estado empezara a
tomar decisiones de largo plazo en su estrategia energética. En los años
subsecuentes se aprobó nuevas leyes para incentivar el uso de energía renovable,
la producción de energía, etcétera.

Para el 2010 se aprobó una nueva legislación que permitía un nuevo esquema
de cobro a las empresas generadoras de electricidad. Este nuevo esquema
llamado \textit{hora de consumo} (\textit{time-of-use} TOU) define que las
tasas de cobro varían en función de la temporada y la hora del día en que
se usa la energía eléctrica, en contraste con las tasas fijas tradicionales.

En un esquema \texttt{TOU} se definen tres tarifas y sus zonas de tiempo asociadas:

\begin{itemize}
\item  Demanda baja: es la tarifa mas baja, se aplica durante las horas de la
  mañana, en la noche y los fines de semana.
\item Demanda mediana: es la tarifa intermedia, se aplica durante las horas de
  10 am a 1pm y de 7 pm a 9 pm.
\item Demanda Alta: es la tarifa mas cara, se aplica durante las horas de
  1 pm a 7 pm.
\end{itemize}

El objetivo de un esquema \texttt{TOU} es el desplazar la carga del sistema
eléctrico, alentando a los usuarios a cambiar su demanda durante periodos
de alta demanda a periodos de menor demanda.

Como fecha de arranque de este nuevo esquema se puso el mes de noviembre del 2012,
las empresas como PG\&E podían iniciar la transición a un esquema \texttt{TOU}
de sus usuarios de pequeñas y medianas empresas (PyMES), aunque una parte vital
para permitir esta transición es que PG\&E brindara herramientas para que el
usuario final pudiera analizar sus gastos de energía.

En este contexto es que nace el sistema \textit{Progressive Energy Audit Tool}
\ (PEAT) para solventar la necesidad de información del usuario final
sobre sus gastos de energía en una forma detallada para que este pueda
identificar y monitorear sus gastos de energía eléctrica.

Desde inicios del 2010 PG\&E había realizado el despliegue de medidores
inteligentes en la mayor parte de sus clientes. Estos medidores permiten
obtener datos muy detallados sobre el consumo de energia de lo usuarios.
El uso de estos datos hacia factible el uso de esquemas \texttt{TOU}
para usuarios finales, asi para principios del 2012 PG\&E tenia ya un
grupo piloto de PyMES

\section{Objetivo general}

El objetivo general del sistema es dar la mayor utilidad posible con la menor
información disponible, fomentando en el usuario final el compartir dicha
información sobre su empresa, obteniendo un mejor control
acerca de su consumo energético.


\backmatter
\nocite{*}
\bibliographystyle{acm}
\bibliography{biblio}

\end{document}
