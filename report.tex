\documentclass[letterpaper,twoside,openright]{book}
%\usepackage[left=4cm,right=4cm,top=4cm,bottom=4cm,letterpaper]{geometry}
\usepackage[spanish,mexico]{babel}
\usepackage[utf8]{inputenc}
\usepackage{graphicx}
\usepackage{mathptmx}
\usepackage{array}
\usepackage[table]{xcolor}
\usepackage{listings}
\usepackage[backend=biber,style=trad-abbrv]{biblatex}
\usepackage{usecases}
\usepackage{enumerate}
\usepackage{csquotes}
\usepackage{color}
\usepackage{placeins}

\let\Oldsection\section
\renewcommand{\section}{\FloatBarrier\Oldsection}

\let\Oldsubsection\subsection
\renewcommand{\subsection}{\FloatBarrier\Oldsubsection}

\let\Oldsubsubsection\subsubsection
\renewcommand{\subsubsection}{\FloatBarrier\Oldsubsubsection}

\definecolor{codegreen}{rgb}{0,0.6,0}
\definecolor{codegray}{rgb}{0.5,0.5,0.5}
\definecolor{codepurple}{rgb}{0.58,0,0.82}
\definecolor{backcolour}{rgb}{0.95,0.95,0.92}

\usepackage{inconsolata}
\usepackage{hyperref}
\hypersetup{
    colorlinks,
    citecolor=black,
    filecolor=black,
    linkcolor=black,
    urlcolor=black
}

\lstdefinestyle{mystyle}{
    backgroundcolor=\color{backcolour},   
    commentstyle=\color{codegreen},
    keywordstyle=\color{magenta},
    numberstyle=\tiny\color{codegray},
    stringstyle=\color{codepurple},
    basicstyle=\footnotesize,
    basicstyle=\ttfamily\small,
    numberstyle=\footnotesize,
    breakatwhitespace=false,         
    breaklines=true,                 
    captionpos=b,                    
    keepspaces=true,                 
    numbers=left,                    
    numbersep=5pt,                  
    showspaces=false,                
    showstringspaces=false,
    showtabs=false,                  
    tabsize=2,
    inputencoding=utf8,
    extendedchars=true,
    literate={á}{{\'a}}1 {é}{{\'e}}1 {ó}{{\'o}}1 {í}{{\'i}}1 {ú}{{\'u}}1 {ñ}{{\~n}}1,
    escapeinside={\%*}{*)}
}
 
\lstset{style=mystyle}

\addbibresource{biblio.bib}

\newcommand{\jcimage}[3]{\begin{figure}[h!]\centering\includegraphics[width=#1\textwidth]{#2}\caption{#3}\end{figure}\vskip10pt}

\newcommand{\rjcimage}[4]{\begin{figure}[h!]\centering\includegraphics[width=#1\textwidth]{#2}\caption{#3}\label{fig:#4}\end{figure}}

\newcommand{\rjcimagefull}[4]{\begin{figure}[p]\centering\includegraphics[width=#1\textwidth]{#2}\caption{#3}\label{fig:#4}\end{figure}}

\newcommand{\jcimageinline}[3]{\begin{figure}[tb]\centering\includegraphics[width=#1\textwidth]{#2}\caption{#3}\end{figure}}
\newcommand{\jcimagefull}[3]{\begin{figure}[p]\centering\includegraphics[width=#1\textwidth]{#2}\caption{#3}\end{figure}}
\renewcommand{\arraystretch}{1.5}

\newcommand{\imagefull}[4]{\begin{figure}\makebox[\textwidth][c]{\includegraphics[width=#1\textwidth]{#2}}\caption{#3}\label{fig:#4}\end{figure}}

\begin{document}

\thispagestyle{empty}

\frontmatter

\begin{minipage}{.3\textwidth}
  \flushleft
  \center{\includegraphics[scale=.09]{imagenes/unam.pdf}}

  \vspace{20pt}

  \center{
    \rule{.5pt}{.6\textheight}
    \hspace{7pt}
    \rule{2pt}{.6\textheight}
    \hspace{7pt}
    \rule{.5pt}{.6\textheight}
  } \\

  \center{\includegraphics[scale=.22]{imagenes/ciencias.pdf}}
\end{minipage}
\begin{minipage}{.7\textwidth}
\flushright

\center{

  \center{
    \LARGE{U}\large{NIVERSIDAD} \LARGE{N}\large{ACIONAL} 
    \LARGE{A}\large{UTÓNOMA} \\[10pt]
    \large{DE} 
    \LARGE{M}\large{ÉXICO} 
  } \\
  \rule{\textwidth}{2pt}
  \\
  \hrulefill\\[1cm]
  
  \LARGE{F}\large{ACULTAD DE } \LARGE{C}\large{IENCIAS}\\[2cm]

  \large{
Diseño e implementación del sistema PEAT (Progressive Energy Audit Tool)  }\\[2cm]

  \huge{
REPORTE DE TRABAJO PROFESIONAL  }\\[1cm]

  \large{QUE PARA OBTENER EL TÍTULO DE:}\\[1cm]

  \large{
Licenciado en Ciencias de la Computación  }\\[1cm]

  \large{PRESENTA:}\\[0.5cm]

  \large{
Héctor Enrique Gómez Morales  }\\[1cm]

  \large{
TUTORA  }\\[0.5cm]

  \large{
M. en I. Karla Ramírez Pulido  }\\[0.5cm]

  \large{
    2016  }\\
}

\end{minipage}

\null\newpage
\null\newpage
\noindent
\textbf{Hoja de Datos del Jurado}\\

\noindent
1. Datos del alumno\\
Gómez\\
Morales\\
Héctor Enrique\\
5518505001\\
Universidad Nacional Autónoma de México\\
Facultad de Ciencias\\
Ciencias de la Computación\\
401048742\\

\noindent
2. Datos del tutor\\
M en I\\
Karla\\
Ramírez\\
Pulido\\

\noindent
3. Datos del sinodal 1\\
Dra\\
María de Luz\\
Gasca\\
Soto\\

\noindent
4. Datos del sinodal 2\\
Dr\\
Daniel\\
Trejo\\
Medina\\

\noindent
5. Datos del sinodal 3\\
Dr\\
Jorge Luis\\
Ortega\\
Arjona\\

\noindent
6. Datos del sinodal 4\\
Act\\
Carlos Ernesto\\
López\\
Natarén\\

\noindent
7. Datos del trabajo escrito\\
Diseño e implementación del sistema PEAT (Progressive Energy Audit Tool)\\
104 p\\
2016\\


\chapter*{Agradecimientos}
\markboth{AGRADECIMIENTOS}{AGRADECIMIENTOS} % encabezado

Agradezco el inagotable amor y apoyo de mis padres, sin ellos este trabajo
nunca hubiera sido posible.

A mis hermanos Javier y Lilian que durante muchos años fueron mis segundos padres
y de los cuales he aprendido tanto.

A mi tutora Karla Ramírez Pulido le tengo un eterno agradecimiento por la constante
ayuda y asesoría dada no solamente para la elaboración de este reporte si no durante
mi paso por la facultad. También agradezco la oportunidad de haber sido su ayudante
en uno de mis cursos favoritos que es Lenguajes de Programación.

Agradezco a mis sinodales Dr. Maria de Luz Gasca Soto, Dr. Daniel Trejo Medina, Dr
Jorge Luis Ortega Arjona y al Act. Carlos Ernesto López Natarén por los comentarios
realizados ya que éstos enriquecieron mi trabajo.

A mis amigos David García Gudiño, Víctor Hugo Ortiz Hernández y
María Margarita López Titla por su apoyo y amistad que ha resistido el tiempo y
distancia.

También agradezco la amistad y apoyo de mis colegas Juan Germán Castañeda Echevarría,
Jesús Alejandro Juárez Robles y César Octavio López Natarén por su profesionalismo y
gran pasión por nuestra profesión que ha sido una gran inspiración para mi.

Agradezco a la Facultad de Ciencias y a la Universidad Nacional Autónoma de México
por darme acceso a tan excelentes profesores que me han formado.

\tableofcontents

\mainmatter

\chapter*{Introducción}
\addcontentsline{toc}{chapter}{Introducción}
El presente trabajo aborda el diseño e implementación del sistema \textit{Progressive
  Energy Audit Tool}\ (\texttt{PEAT}), el cual es un sistema web que permite
a los usuarios de pequeñas y medianas empresas (PyMES) identificar y monitorear
sus gastos de energía eléctrica, con el fin de obtener un mayor control
en los costos de ésta.

En una gran cantidad de proyectos resulta ser suficiente el hacer uso de un
único lenguaje de programación para llevar a buen término el proyecto, es decir,
terminarlo de forma exitosa. Sin embargo conforme pasa el tiempo, las necesidades
de las compañías y usuarios cambian, lo cual provoca que hagamos uso de una serie de
tecnologías no previstas en un inicio, lo que representa casi siempre el uso
de más de un lenguaje de programación\footnote{Debido a que ningún
lenguaje de programación actualmente es el mejor en todos los contextos
posibles de uso.}.

En el proyecto de implementación del sistema PEAT cuenta con tres partes principales
de desarrollo: (a) el \textit{backend}  implementado en su mayoría usando
el lenguaje de programación Javascript, con un núcleo escrito en el lenguaje de
programación Java, (b) una biblioteca en el lenguaje de programación Ruby que permite
la interacción del \textit{backend} con aplicaciones web y (c) el \textit{frontend}
escrito con Ruby, dada sus ventajas para realizar sistemas web.

Durante este proyecto se implementó una biblioteca no solo para hacer uso de
los servicios web ya existentes, sino para facilitar la creación
de clases y objetos en Ruby basados en la jerarquía de clases definida
en Javascript. Cabe mencionar que haciendo uso de una de las fortalezas de Ruby
como lo es la metaprogramación para la creación de clases y sus
atributos ``al vuelo'', según las especificaciones enviadas por el servidor,
se mantiene así la jerarquía de clases de un lenguaje a otro.

Otro punto de creciente importancia es el despliegue de una aplicación, ya que por un
lado hace algunos años las aplicaciones web se desarrollaban bajo el modelo de
tres capas: la base de datos, el servidor de la aplicación y el servidor web;
actualmente se tienen mas servicios que deben estar en línea sobre todo para
garantizar un servicio concurrente; caches, equilibradores de carga, servidores
de cola, etcétera. Por lo que automatizar el despliegue de una aplicación en sus
diferentes contextos, desarrollo, producción y pruebas, es de vital importancia
para el desarrollo de software en tiempo y en forma.

Durante este proyecto se hace uso extensivo de técnicas como la metaprogramación
y el despliegue continuo, para permitir un desarrollo acelerado, con el fin de
obtener rápidamente retroalimentación del usuario final.

\section*{Objetivo general}

El objetivo general del sistema \texttt{PEAT} es dar la mayor utilidad posible
a usuarios PyMES con la menor información disponible, fomentando en el usuario
el compartir más información sobre su empresa, obteniendo un mejor control
acerca de su consumo energético.

El sistema debe además permitir el ingreso progresivo de información, por medio
de una serie de preguntas específicas al usuario, dando mejores recomendaciones
para bajar su consumo energético conforme el sistema obtiene mas información.

A partir de la información obtenida el sistema debe permitir el monitoreo
y revisión del consumo energético de forma detallada, ya sea en horas,
días, meses y años.

\section*{Objetivos secundarios}

Los objetivos secundarios que apoyan al objetivo general de este trabajo son:
\begin{itemize}
\item Implementación de una interfaz que permita la obtención
  de información del usuario de una forma eficaz y sencilla.
\item Dar información útil aunque el usuario solo proporcione el
  mínimo de información sobre su empresa.
\item Proporcionar recomendaciones para disminuir sus
  gastos en energía con base en el consumo e información proporcionada
  hasta el momento.
\item Autentificar a los usuarios mediante el uso de credenciales de acceso
  obtenidas en el portal web de PG\&E.
\item Diseñar e implementar un conjunto de pruebas unitarias, funcionales
  y de integración para los módulos críticos del sistema.
\item Soportar por lo menos a mil usuarios concurrentes.
\item Implementar la infraestructura para el despliegue continuo de la
  aplicación, permitiendo una retroalimentación contínua sobre el
  funcionamiento del sistema.
\end{itemize}

% FIXME: Usar un mejor subtitulo
\section*{Organización del trabajo}

Este trabajo está dividido en tres capítulos los cuales son:
\begin{itemize}
\item Capítulo 1 titulado, \textit{Situación actual}: se presenta una descripción
  del contexto que dio pie al desarrollo del sistema \texttt{PEAT}, después se da una
  descripción de la arquitectura y sistemas con los que se tuvo como punto de
  partida para su implementación. Finalmente se define la propuesta de desarrollo
  del sistema.
\item Capítulo 2 que lleva por nombre, \textit{Fundamentos teóricos}: se expone un
  resumen de los conceptos teóricos mas influyentes en el diseño e implementación del
  sistema \texttt{PEAT}.
\item Capítulo 3, \textit{Diseño e implementación del sistema PEAT}: se desarrolla y
  describe la implementación del sistema \texttt{PEAT}, dando la justificación
  de las decisiones tomadas durante su implementación.
\end{itemize}

El sistema \texttt{PEAT} es resultado del trabajo en conjunto de tres compañías:
Pacific Gas and Electric Company (PG\&E), C3 Energy y Software Next Door.

Cabe mencionar que PG\&E es una compañía proveedora de gas natural y electricidad, una de las
más grandes compañías de Estados Unidos con sede en San Francisco, California.
El sistema \texttt{PEAT} es el resultado de una licitación iniciada por PG\&E,
siendo ganadora de dicha licitación la compañía C3 Energy.

\chapter{Situación actual}

\section{Contexto}
De 2000 a 2002 se tuvo una crisis energética en el estado de California en los
Estados Unidos, esta crisis provoco que el gobierno del estado empezara a
tomar decisiones de largo plazo en su estrategia energética. En los años
subsecuentes se aprobó nuevas leyes para incentivar el uso de energía renovable,
la producción de energía, etcétera.

Para el 2010 se aprobó una nueva legislación que permitía un nuevo esquema
de cobro a las empresas generadoras de electricidad. Este nuevo esquema
llamado \textit{hora de consumo} (\textit{time-of-use} TOU) define que las
tasas de cobro varían en función de la temporada y la hora del día en que
se usa la energía eléctrica, en contraste con las tasas fijas tradicionales.

En un esquema \texttt{TOU} se definen tres tarifas y sus zonas de tiempo asociadas:

\begin{itemize}
\item  Demanda baja: es la tarifa mas baja, se aplica durante las horas de la
  mañana, en la noche y los fines de semana.
\item Demanda mediana: es la tarifa intermedia, se aplica durante las horas de
  10 am a 1pm y de 7 pm a 9 pm.
\item Demanda Alta: es la tarifa mas cara, se aplica durante las horas de
  1 pm a 7 pm.
\end{itemize}

El objetivo de un esquema \texttt{TOU} es el desplazar la carga del sistema
eléctrico, alentando a los usuarios a cambiar su demanda durante periodos
de alta demanda a periodos de menor demanda.

Como fecha de arranque de este nuevo esquema se puso el mes de noviembre del 2012,
las empresas como PG\&E podían iniciar la transición a un esquema \texttt{TOU}
de sus usuarios de pequeñas y medianas empresas (PyMES), aunque una parte vital
para permitir esta transición es que PG\&E brindara herramientas para que el
usuario final pudiera analizar sus gastos de energía.

En este contexto es que nace el sistema \textit{Progressive Energy Audit Tool}
\ (PEAT) para solventar la necesidad de información del usuario final
sobre sus gastos de energía en una forma detallada para que este pueda
identificar y monitorear sus gastos de energía eléctrica.

Desde inicios del 2010 PG\&E había realizado el despliegue de medidores
inteligentes en la mayor parte de sus clientes. Estos medidores permiten
obtener datos muy detallados sobre el consumo de energia de lo usuarios.
El uso de estos datos hacia factible el uso de esquemas \texttt{TOU}
para usuarios finales, asi para principios del 2012 PG\&E tenia ya un
grupo piloto de PyMES

\section{Objetivo general}

El objetivo general del sistema es dar la mayor utilidad posible con la menor
información disponible, fomentando en el usuario final el compartir dicha
información sobre su empresa, obteniendo un mejor control
acerca de su consumo energético.

\chapter{Fundamentos teóricos}

\section{Patrón Modelo-Vista-Controlador (MVC)}
El patrón Modelo Vista Controlador (MVC) es probablemente el patrón
mas utilizado y citado para el desarrollo de interfaces de usuario y sistemas web.
MVC consiste de tres tipos de objetos:

\begin{itemize}
\item Modelo: Representación de la información de dominio del sistema.
\item Vista: Representación visual del modelo.
\item Controlador: Define la forma en que la interfaz reacciona a la entrada
  del usuario.
\end{itemize}

\jcimage{1.0}{imagenes/MVC-Rails.png}{Patrón MVC para sistemas web.}

MVC fue ideado originalmente para aplicaciones gráficas convencionales,
donde los desarrolladores encontraron que la separación de responsabilidades,
entre la presentación (vista y controlador) y el dominio (modelo), fomentadas
por el patrón llevan a un menor acoplamiento lo que hacia al código
mucho mas fácil de escribir y mantener.

MVC desacopla vistas y modelos mediante el establecimiento de un
protocolo de suscripción / notificación. La vista debe asegurarse
de que su aspecto visual refleje el estado del modelo. Cada vez que cambian
los datos del modelo, el modelo notifica a las vistas que depende de ella.
Este enfoque permite conectar múltiples vistas a un modelo para proporcionar
diferentes presentaciones. También puede crear nuevas vistas para un modelo
sin reescribir este ultimo \cite[pag.~4]{14_gamma_1995}.

\subsection{MVC y Rails}
En el marco de trabajo Rails se hace uso de MVC como patrón de arquitectura
para implementar sistemas web. En Rails los modelos se definen haciendo
uso de la biblioteca ActiveRecord, esta biblioteca implementa el
patrón de mapeo objeto-relacional (\textit{Object-relational mapping}
(\textit{ORM})) para facilitar el acceso de información contenida en
bases de datos relacionales, dado que es el caso típico en sistemas web
convencionales.

En Rails, la vista es responsable de la creación de la respuesta dada para
ser mostrada en un navegador. En su forma mas simple, una
vista es un trozo de código HTML que muestra un texto fijo. Mas típicamente
se requiere mostrar contenido dinámico creado por una acción en un controlador.
El contenido dinámico es generado por medio de plantillas, el esquema
de plantillas más común es llamada \textit{Embedded Ruby} (ERB),
el cual inserta pedazos de código Ruby dentro de una vista, similar a la forma
como se hace en otros marcos de trabajo como PHP o JSP. También se pude hacer uso de
ERB para incrustar pedazos de código Javascript en el servidor
para ser ejecutados en el navegador, lo cual permite crear interfases
dinámicas haciendo uso de \textit{Asynchronous JavaScript and XML} (AJAX).

Finalmente en Rails los controladores son el centro lógico del sistema. Coordinan
la interacción entre el usuario, las vistas y el modelo
\cite[pag.~29]{15_agile_hansson}.

\subsection{MVC y PEAT}
\texttt{PEAT} saca provecho del patrón MVC de las siguientes maneras:

\begin{enumerate}
\item Modelo: Dado que en un principio los servicios web de recomendaciones y
  desagregación estaban en construcción se hizo uso de la biblioteca ActiveRecord
  para tener datos reales estáticos para permitir la implementación
  de la interfaz de usuario. Posteriormente se reemplazaron estos modelos
  por nuevos modelos que hacían uso de los servicios web del \textit{backend},
  por medio de la biblioteca \texttt{Bezel}. Dado que hay un desacoplamiento
  entre los modelos y las vistas esto no implico grandes cambios al hacer el
  reemplazo.
\item Vista: Haciendo uso de plantillas se generan representaciones HTML
  y JSON de los principales modelos del sistema. Para ciertos modelos como
  las recomendaciones se tenia una tercera representación en forma de PDF
  del modelo.
% FIXME: Escribir sobre beneficios en contexto de los controladores
\end{enumerate}

\section{Servicios web RESTful}
La \textit{World Wide Web Consortium} (W3C) define que un servicio web
en general es un sistema de software diseñado para dar suporte a interacciones
maquina-maquina a través de una red informática. Su implementación es por la
necesidad de que diferentes sistemas puedan intercambiar datos entre ellos.

% Cita del glosario de la W3C
% http://www.w3.org/TR/2004/NOTE-ws-gloss-20040211/#webservice

%Un servicio web es un método de comunicación entre dos dispositivos sobre una red
%informática, su implementación es por la necesidad de que diferentes sistemas
%puedan intercambiar datos entre ellos.

La Transferencia de Estado Representacional (REST) es una arquitectura
de software para la implementación de servicios web. En REST se define la
existencia de recursos (elementos de información), donde cada recurso tiene un
conjunto de representaciones posibles. Por ejemplo una lista de bugs por arreglar
(recurso) puede ser presentado en forma de un documento XML, una pagina HTML
o un archivo CSV (representaciones). Además se tienen cuatro características
principales \cite[pag.~79]{1_richardson_2007}:

\begin{itemize}
\item Protocolo cliente/servidor sin estado (\textit{stateless}): cada mensaje
  HTTP contiene toda la información necesaria para comprender la petición.
  Esto implica que ni el cliente ni el servidor necesitan recordar ningún
  tipo de estado.
\item Conectividad (\textit{connectedness}): Las representaciones son un hipermedio
  en el cual se tienen ligas a otros recursos. Como resultado de esto, es posible
  navegar de un recurso REST a muchos otros, sin necesidad de una
  infraestructura adicional.
\item Direccionabilidad (\textit{addressability}): La capacidad para
  identificar los recursos del sistema. Cada recurso es direccionable únicamente
  a través de su Identificador de Recursos Uniforme (URI).
\item Interfaz uniforme (\textit{uniform interface}): Se tiene un conjunto de
  operaciones bien definidas que se aplican a todos los recursos del sistema.
  Se usan los verbos de HTTP pare definir las operaciones mas importantes
  que son GET, POST, PUT, PATCH y DELETE.
\end{itemize}

Los servicios web que implementan una arquitectura REST se suelen llamar
servicios web RESTful.

\subsection{REST y Rails}

La arquitectura REST es parte vital de Rails, todo el enrutamiento y
manejo de peticiones se basa en esta arquitectura.

En REST se hace uso de un conjunto finito de verbos para operar sobre otro
conjunto de objetos. Dado que estamos usando HTTP como capa de transporte, los
verbos corresponden a los métodos HTTP (GET, POST, PUT, PATCH, y DELETE).
Los objetos corresponden a los recursos del sistema, los cuales son etiquetados
usando URLs.

Un navegador solicita paginas de Rails al hacer una petición para una dirección URI
haciendo uso de un método HTTP especifico, como GET o POST. Cada método es una
petición para realizar una operación sobre el recurso.

Haciendo uso de la interfaz uniforme, Rails define toda un conjunto de rutas
para un recurso, tomando como ejemplo el concepto de edificio. Se define
el recurso y sus rutas asociadas con lo siguiente:

\begin{verbatim}
resources :buildings
\end{verbatim}

En la Figura 2.2 se pueden ver las rutas y verbos asociados para las principales
operaciones sobre el recurso \textit{buildings}, esto acelera en gran medida
el desarrollo de servicios web.

\jcimage{1.0}{imagenes/REST-Rails.png}{Rutas y verbos para el recurso \textit{buildings}.}

\subsection{REST y PEAT}
La arquitectura REST influye en las tres partes principales de \texttt{PEAT}

\begin{itemize}
\item \textit{Frontend}: Por medio de Rails se implementa un servicio web RESTful
  para proveer de información a la interfaz gráfica del sistema.
\item \textit{Backend}: Los servicios web que componen el \textit{backend}, como el
  sistema de recomendaciones, son del tipo RESTful. El recurso principal del sistema
  es el de edificio (\textit{Building}) del cual se tienen alrededor los demás
  recursos del sistema.
\item \textit{Middleware}: La biblioteca \texttt{Bezel} saca provecho del hecho
  de que el \textit{backend} sea de tipo RESTful para facilitar la integración del
  sistema, dado que hacer un mapeo de los recursos a clases y objetos es
  casi directa.

\end{itemize}

\chapter{Diseño e implementación del sistema PEAT}

Para lograr los objetivos del sistema \texttt{PEAT} se dividió en cuatro
componentes principales:

\begin{itemize}
\item Perfil sencillo del edificio (\textit{Simple Building Profile}): este
  componente realiza la configuración inicial del edificio, realizando solamente
  un numero limitado de preguntas básicas para definir las características básicas
  del edificio.
\item Perfil detallado del edificio (\textit{Detailed Building Profile}):
  este componente se encarga del manejo de las características básicas, detalladas y
  opcionales de un edificio.
\item Recomendaciones (\textit{Recommendations}): este componente genera una lista
  de medidas de ahorro de energía considerando la información ingresada al sistema
  hasta el momento. Aparte debe permitir al usuario obtener información detallada
  sobre las recomendaciones dadas y finalmente debe permitir indicar que el usuario
  se compromete a llevar acabo una recomendación.
\item Plan de ahorro de energía (\textit{Energy Savings Plan}): este componente
  presenta el consumo de energía del edificio al usuario, también presenta las
  recomendaciones que el usuario se ha comprometido a poner en acción.
\end{itemize}

Como se puede ver el modelo de edificio es el principal modelo dentro de
\texttt{PEAT}, por lo que obtener una correcta correspondencia entre un
edificio y su consumo energético es de vital importancia para dar la
información y recomendaciones mas útiles y fidedignas al usuario.

\section{Modelo de información}

El objetivo principal del modelo de información de \texttt{PEAT} es definir el
modelo de edificio y su relación con el consumo energético del usuario. Así
tenemos el concepto de edificio el cual es una estructura independiente a la
cual se le esta proporcionando un servicio ya sea electricidad o gas.

\subsection{Estructura de facturación de PG\&E}
Para obtener esta correspondencia entre edificio-consumo es necesario
tener en cuenta la estructura de facturación usada por PG\&E puesto
que de esta estructura es que se obtiene el consumo de electricidad y gas.

El usuario, la cuenta y el contrato de servicio representan el lado demográfico
del modelo de información, la cual cambia bastante:

\begin{itemize}
\item Usuario: representa al cliente. Tiene una o más cuentas.
\item Cuenta: representa la contabilidad financiera del usuario. Tiene uno o más
  contratos de servicio, uno por tipo de servicio (electricidad o gas), también
  tiene una o más direcciones de servicio asociadas.
\item Contrato de servicio: representa los servicios que el cliente adquiere
  de PG\&E, como la electricidad o el gas, puede tener uno o más puntos
  de servicio asociados.
\end{itemize}

La dirección de servicio, el punto de servicio y el medidor representan
la parte geográfica del modelo de información, esta información raramente cambia:

\begin{itemize}
\item Dirección de servicio: es una ubicación física en donde se prestan los
  servicios, en \texttt{PEAT} son el modelo que se quiere asociar al modelo
  de edificio para obtener el consumo del edificio, puede tener uno o más puntos
  de servicio.
\item Punto de servicio: es una coordenada geográfica en donde se conectan los
  servicios, puede tener uno o más medidores asociados.
\item Medidor: es un dispositivo instalado en un punto de servicio que registra el
  consumo del servicio proporcionado.
\end{itemize}

La relación entre todos estos modelos se puede ver en la Figura 3.1.

\jcimage{1.0}{imagenes/PGE-facturacion.png}{Modelos principales para la facturación
  en PG\&E.}

\subsection{Relación Edificio-Facturación}

Para que \texttt{PEAT} pueda cumplir sus requerimientos funcionales es necesario
definir claramente la relación entre un edificio y su consumo energético medido
por uno o varios medidores.
Aunque los medidores son los que tienen la información de consumo el relacionar
directamente un medidor con un edificio no es amigable para el usuario, porque
aunque el usuario generalmente conoce el numero de medidores que tiene no conoce
los identificadores para cada uno de estos.

Por lo que para facilitar la creación del perfil de un edificio se pide al usuario
que haga la correspondencia entre las direcciones de servicio asociadas a su cuenta
y el perfil de edificio que se esta creando.
El usuario al dar esta correspondencia permite al sistema obtener de forma
automatizada los puntos de servicio y sus medidores asociados al edificio y así
obtener su consumo energético real.

\section{Casos de uso}

\subsection{Diagrama general}

En la Figura 3.2 se muestra el diagrama general de casos de uso del sistema
\texttt{PEAT}. Se tienen diez casos de uso en total siendo siete casos para
\texttt{PEAT} y tres casos para la biblioteca \texttt{Bezel}.

Los casos de uso se agrupan en los cuatro componentes principales de la
siguiente manera:

\begin{itemize}
\item Perfil sencillo del edificio (\textit{Simple Building Profile}): Tiene
  solo el caso de uso \textit{Crear perfil del edificio} pero este caso de uso
  es uno de los mas complejos en cuestión de interacción con el usuario y su
  importancia para obtener las características básicas del edificio.
\item Perfil detallado del edificio (\textit{Detailed Building Profile}): Contiene
  los casos de uso \textit{Administrar perfil del edificio} y
  \textit{Obtener información del edificio} los cuales se encargan de la
  administración y obtención de información sobre el edificio.
\item Recomendaciones (\textit{Recommendations}): Contiene los casos de uso
  \textit{Administrar recomendaciones} y \textit{Ver recomendación}.
\item Plan de ahorro de energía (\textit{Energy Savings Plan}): Contiene los
  casos de uso \textit{Ver historial de consumo} y \textit{Administrar plan de
    ahorro}
  que se enfocan a la visualización del consumo eléctrico del usuario y las
  posibles recomendaciones para disminuir su consumo.
\end{itemize}

\jcimageinline{1.0}{imagenes/Diagrama-General-CasosDeUso.png}{Diagrama general de
  casos de uso de \texttt{PEAT}.}

\subsection{Crear perfil del edificio}

Para crear el perfil inicial de un edificio es necesario obtener las siguientes
características:

\begin{itemize}
\item La industria que mejor describe el negocio del usuario.
\item El tamaño del edificio (área).
\item El tipo de edificio.
\item La(s) dirección(es) de servicio asociada(s) al edificio.
\end{itemize}

Todas estas características se consideran básicas, es decir vitales para el
funcionamiento del sistema por lo todas son requeridas para construir el perfil
del edificio.

En este caso de uso también se obtiene la relación entre el edificio y su consumo
a partir de las direcciones de servicio que el usuario asocie al perfil.

El caso de uso esta diseñado de forma que el proceso sea realizado solamente
una vez por edificio. Las características básicas se mantienen accesibles por
medio del perfil detallado del edificio.

\begin{usecase}
  \addtitle{Caso de Uso:}{Crear perfil del edificio.}
  \addfield{Descripción:}{Un usuario PyME entra al sistema  para crear un perfil
    inicial de un edificio, respondiendo ciertas preguntas básicas sobre el
    edificio.}
  \addfield{Actor:}{Usuario PyME}
  \additemizedfield{Precondiciones:}{
  \item El usuario a sido autentificado por medio de un token de autenticación
    (\textit{Single Sign On} SSO) asignado por el sistema de PG\&E.
  \item Los datos de intervalo y de facturación del usuario están en el sistema.
  }
  \additemizedfield{Requerimientos no funcionales:}{
  \item El usuario debe ser capaz de completar el perfil del edificio en 20 segundos
    o menos.
  }
  \addscenario{Flujo normal:}{
  \item El sistema checa si el usuario tiene mas de una cuenta.
    \begin{enumerate}
    \item Si hay mas de una cuenta el sistema despliega una lista
      de cuentas, y el sistema requiere que el usuario elija una cuenta.
    \end{enumerate}
  \item El sistema despliega información sobre la cuenta: numero de cuenta y
    direcciones de servicio asociadas.
    \begin{enumerate}
    \item Si la cuenta tiene mas de una dirección de servicio
      el sistema requiere que el usuario elija por lo menos una dirección
      de servicio.
    \end{enumerate}
  \item El usuario selecciona las direcciones de servicio que apliquen
    al edificio.
  \item El usuario puede proporcionar un apodo para el edificio.
  \item El usuario debe indicar la industria que mejor describe su negocio.
  \item El usuario debe ingresar la siguiente información sobre el edificio:
    \begin{enumerate}
    \item Tipo.
    \item Rango de tamaño.
    \item Antigüedad.
    \end{enumerate}
  \item El usuario puede enviar los datos ingresados o cancelar.
  \item El usuario puede añadir otro edificio o si ya tiene al menos un
    perfil de edificio asociado puede proceder al plan de ahorro.
  }
  \addfield{Postcondiciones:}{
    Se genera un identificador único para el perfil del edificio generado, el cual
    esta asociado a los datos de facturación de la cuenta seleccionada por el
    usuario PyME.
  }
\end{usecase}

% FIXME: Insertar diagrama del modelo de información inducido por la creación
% de un perfil

\subsection{Administrar perfil del edificio}

Para el sistema \texttt{PEAT} se tiene las siguientes características:
\begin{itemize}
\item Básicas: son las características básicas de un edificio: tipo, tamaño y
  antigüedad, son vitales para el funcionamiento del sistema.
\item Detalladas: son las características mas significativas y fáciles de
  responder, ejemplos son: horas de operación, numero de empleados, numero
  de pisos, etcétera.
\item Opcionales: son características más avanzadas/técnicas sobre el equipo
  y estructura de un edificio.
\end{itemize}

Las características básicas son definidas al crear el perfil inicial de un
edificio como se documenta en el caso de uso anterior. Las características
detalladas y opcionales son definidas por medio de dos rutas:

\begin{itemize}
\item Por medio del perfil detallado del edificio, en la que se despliega
  las preguntas y respuestas sobre las características del edificio.
  Esta ruta es documentada en este caso de uso.
\item Un componente que es embebido en varias partes del sistema
  que realiza una sola pregunta al usuario para ir obteniendo progresivamente
  mas información sobre el edificio. Este componente tiene el nombre
  de Componente de Ingreso Progresivo (\textit{Progressive Profile Widget} PPW).
  Esta ruta es documentada en el caso de uso \textit{Obtener información
    del edificio}
\end{itemize}

\begin{usecase}
  \addtitle{Caso de Uso:}{Administrar perfil del edificio.}
  \addfield{Descripción:}{
    El usuario tiene acceso al perfil completo del edificio, es decir
    a todas las preguntas y respuestas asociadas a las características
    básicas, detalladas y opcionales de un edificio.}
  \addfield{Actor:}{Usuario PyME}
  \additemizedfield{Precondiciones:}{
  \item El usuario a sido autentificado por medio de un token de autenticación
    (\textit{Single Sign On} SSO) asignado por el sistema de PG\&E.
  \item El usuario a creado previamente el perfil básico del edificio.
  }
  \additemizedfield{Requerimientos no funcionales:}{
  \item Todas las acciones realizadas por el usuario deben ser procesadas
    en menos de un segundo.
  }
  \addscenario{Flujo normal:}{
  \item El sistema checa si la cuenta tiene mas de un edificio asociado.
    \begin{itemize}
    \item Si hay mas de un edificio asociado a la cuenta el usuario
      tiene que seleccionar el edificio correspondiente.
    \end{itemize}
  \item El sistema despliega una lista de preguntas y respuestas divida en secciones.
    Las secciones completadas están marcadas con una marca de completado.
    \begin{itemize}
    \item Las secciones que se despliegan dependen del tipo de edificio,
      por ejemplo: construcción, iluminación, calefacción y refrigeración.
    \item El usuario puede ver y actualizar sus respuestas a todas las
      preguntas en una sola pantalla.
    \end{itemize}
  \item El usuario selecciona una sección, esta se expande para mostrar todas
    las preguntas de la sección.
  \item El usuario responde una o varias preguntas.
    \begin{enumerate}
    \item Las preguntas son ordenadas dentro de cada sección según un orden
      predefinido. Las secciones también tienen un orden predefinido.
    \item Según vaya respondiendo el usuario se va actualizando un indicador
      del progreso de llenado del perfil. Cada pregunta/respuesta tiene el
      mismo valor para calcular el progreso de llenado.
    \end{enumerate}
  \item El usuario puede elegir continuar con la siguiente sección o dar click
    en el encabezado de otra sección.
  \item El sistema salva la información en cuanto el usuario selecciona
    una respuesta. Al obtener nueva información el sistema debe recalcular
    las recomendaciones asociadas al edificio.
  }
  \additemizedfield{Postcondiciones:}{
  \item Un perfil del edificio mas detallado si el usuario dio respuesta a una
    pregunta sin contestar sobre su edificio.
  \item Si el usuario dio nueva información se recalculan las recomendaciones
    asociadas al edificio.
  }
\end{usecase}

\subsection{Obtener información del edificio}

Uno de los requerimientos mas importante para \texttt{PEAT} es el incentivar y
obtener mas información sobre las características de los edificios del usuario.
Esto es porque entre mas información se obtiene de estos se pueden generar
mejores planes de ahorro y recomendaciones para el usuario.

La forma en como se aborda esta necesidad es por medio de la implementación
de un componente que esta presente en las dos partes principales del sistema:
el plan de ahorro y la lista de recomendaciones. Este componente tiene el
nombre de Componente de Ingreso Progresivo de Información (\textit{Progressive
  Profile Widget} PPW). Este componente realiza una sola pregunta al usuario
obteniendo progresivamente mas información sobre el edificio.

\begin{usecase}
  \addtitle{Caso de Uso:}{Obtener información del edificio.}
  \addfield{Descripción:}{El sistema va obteniendo información
    progresivamente del usuario haciendo uso de un componente que
    realiza una pregunta al usuario en varias partes del sistema.
    Este componente tiene el nombre de Componente de Ingreso Progresivo de
    Información (\textit{Progressive Profile Widget} PPW).}
  \addfield{Actor:}{Usuario PyME}
  \additemizedfield{Precondiciones:}{
  \item El usuario a sido autentificado por medio de un token de autenticación
    (\textit{Single Sign On} SSO) asignado por el sistema de PG\&E.
  \item El usuario a creado previamente el perfil básico del edificio.
  }
  \additemizedfield{Requerimientos no funcionales:}{
  \item Las actualizaciones a la pagina deben hacerse en forma
    asíncrona y sin necesitar una recarga completa de la pagina.
  }
  \addscenario{Flujo normal:}{
  \item El sistema despliega el componente PPW en la parte izquierda en las
    pantallas de plan de ahorro y de recomendaciones.
  \item Dentro del PPW, se presenta al usuario una pregunta a responder
    \begin{enumerate}
    \item Dependiendo de la pregunta, el usuario puede seleccionar una
      respuesta de una lista desplegable, seleccionar casillas,
      responder Si o No con un botón radial o ingresar un valor
      en un campo.
    \item Cada tipo de edificio tiene una lista predefinida ordenada de
      preguntas a ser desplegadas en el componente.
    \item Algunas preguntas tienen imágenes asociadas para facilitar
      al usuario su respuesta.
    \end{enumerate}
  \item El usuario puede o bien responder la pregunta, saltar la pregunta o
    no hacer nada en el PPW.
    \begin{enumerate}
    \item Si el usuario responde:
      \begin{enumerate}
      \item La respuesta provoca un refinamiento del perfil del edificio y
        el nuevo calculo de las recomendaciones propuestas.
        \begin{enumerate}[i.]
        \item El perfil del edificio es actualizado, las recomendaciones
          son recalculadas.
        \item El sistema despliega un indicador de progreso mientras
          se recalculan las recomendaciones.
        \item Al finalizar el refinamiento el orden de las preguntas
          o los valores del plan de ahorro deben ser actualizados, esto
          debe hacerse sin necesitar una recarga completa de la pagina.
        \end{enumerate}
      \item El sistema despliega la siguiente pregunta.
      \item El sistema persiste la respuesta dada.
      \end{enumerate}
    \item Si el usuario elige saltar la pregunta entonces el sistema
      despliega la siguiente pregunta sin necesitar de una respuesta por el usuario.
    \end{enumerate}
  \item Desde el PPW, el usuario puede dar click a ''Building Profile''\ para ver
    el perfil detallado del edificio.
  }
  \additemizedfield{Postcondiciones:}{
  \item Un perfil del edificio mas detallado si el usuario dio respuesta a la
    pregunta desplegada sobre su edificio.
  \item Si el usuario dio nueva información se recalculan las recomendaciones
    asociadas al edificio.
  }
\end{usecase}

\subsection{Administrar recomendaciones}

En los casos de uso anteriores se obtiene del usuario el perfil del edificio
y por lo menos sus características básicas. Haciendo uso de esta información
el sistema \texttt{PEAT} genera recomendaciones sobre como disminuir el consumo
de energía a través de cambios de comportamiento y/o modernización de equipo.

Las recomendaciones se muestran en tres partes del sistema:
\begin{enumerate}
\item En el plan de ahorro.
\item En las recomendaciones.
\item En el perfil detallado del edificio.
\end{enumerate}

\begin{usecase}
  \addtitle{Caso de Uso:}{Administrar recomendaciones.}
  \addfield{Descripción:}{El sistema despliega una lista de todas las
    recomendaciones que se ajustan al perfil del edificio del usuario.}
  \addfield{Actor:}{Usuario PyME}
  \additemizedfield{Precondiciones:}{
  \item El usuario a sido autentificado por medio de un token de autenticación
    (\textit{Single Sign On} SSO) asignado por el sistema de PG\&E.
  \item El usuario a creado previamente el perfil básico del edificio.
  }
  \additemizedfield{Requerimientos no funcionales:}{
  \item
  }
  \addscenario{Flujo normal:}{
  \item El sistema despliega las recomendaciones, ya sea en el plan
    de ahorro, la lista de recomendaciones o en el perfil detallado del edificio.
  \item El usuario puede tomar las siguientes acciones:
    \begin{itemize}
    \item Añadir al plan (de ahorro).
      \begin{enumerate}
      \item Cuando el usuario da click a ''Añadir al plan'', una ventana modal
        se abre y pregunta: ''¿ Cuando piensa completar esta acción ?''.
      \item El sistema proporciona lista predefinida de rangos de tiempo:
        una semana, un mes, tres meses, etcétera.
      \item El usuario elige el rango de tiempo adecuado y da click
        a ''Add to plan''
      \item El sistema añade la recomendación al plan de ahorro del edificio.
      \end{enumerate}
    \item No aplica.
      \begin{enumerate}
      \item El sistema mueve la recomendación al final de la lista de
        recomendaciones.
      \item En cualquier momento, el usuario puede cambiar el estado de la
        recomendación, es decir añadir la recomendación al plan de ahorro
        o indicar que no es aplicable al edificio.
      \end{enumerate}
    \item Ya completado.
      \begin{enumerate}
      \item El sistema mueve la recomendación al final de la lista de
        recomendaciones.
      \item En cualquier momento, el usuario puede cambiar el estado de la
        recomendación, es decir añadir la recomendación al plan de ahorro
        o indicar que no es aplicable al edificio.
      \end{enumerate}
    \end{itemize}

  \item El sistema refina y reordena la lista de recomendaciones después de que
    el usuario realiza cualquiera de las acciones anteriores.
  }
  \Addfield{Postcondiciones:}{
  }
\end{usecase}

\section{Interfase de usuario}

Dentro de los requerimientos de \texttt{PEAT} se tenia la implementación de una
interfaz de usuario que fuera lo suficientemente eficaz y sencilla, para así obtener
la mayor cantidad de información del usuario.

\subsection{Crear perfil del edificio}

\section{Biblioteca Bezel}

Para realizar la integración entre el sistema \texttt{PEAT} y el servicio web C3,
el API proporcionado por el servidor C3, se implemento una biblioteca con el nombre
de \texttt{Bezel}.

\texttt{PEAT} hace uso del marco de trabajo Rails que a su vez hace
uso del patrón de arquitectura MVC. En Rails los modelos por lo general hacen uso
de la biblioteca \texttt{ActiveRecord} la cual permite implementar modelos basados
en tablas de bases de datos relacionales. En \texttt{PEAT} los datos
son proporcionados no directamente por una base de datos relacional sino por
el servicio web C3 proporcionado por el servidor C3.

Dentro del marco de trabajo Rails también se tiene la biblioteca
\texttt{ActiveResource} la cual implementa el caso de uso en que los datos son
proporcionados por un servicio web tipo RESTful. En un primer acercamiento se trato
de realizar la integración de \texttt{PEAT} con el servicio web C3 haciendo uso de
esta biblioteca pero se tenían las siguientes dificultades:

\begin{itemize}
\item Presupone que el servicio web hace uso de todos los métodos HTTP (GET, POST,
  UPDATE, DELETE) para definir las acciones básicas sobre un recurso.
\item No permite la definición de asociaciones entre recursos.
\end{itemize}

Aunque el servicio web C3 es de tipo RESTful este no hace uso de los métodos
HTTP para realizar acciones sobre los recursos del servicio, en el servicio web
C3 todas las acciones que se quieren realizar se realizan usando el método HTTP
POST y se manda la acción a realizar como un parámetro dentro de la petición HTTP.

La otra dificultad es que teníamos un gran numero de asociaciones entre los
recursos definidos por el servicio web C3 pero la biblioteca \texttt{ActiveResource}
no proporcionaba ninguna ayuda para implementar estas asociaciones.

Dada las dificultades descritas hacer uso de \texttt{ActiveResource} no era una
solución viable pero sirvió como inspiración para implementar la biblioteca
\texttt{Bezel} para proporcionar acceso al servicio web C3.

El servidor C3 cabe recordar esta implementado en dos lenguajes de programación, la
gran mayoría implementado con Javascript mientras que el núcleo y funciones criticas
están implementadas en Java. El servidor C3 puede mandar tanto XML como JSON como
respuesta a peticiones realizadas al servidor, pero la representación en JSON es la
mas usada dado que el sistema de tipos usando por el sistema esta definido en
Javascript.

\texttt{Bezel} permite la reconstrucción automática de los tipos definidos por el
servidor C3 en Javascript a modelos compatibles con el marco de trabajo Rails
y el lenguaje de programación Ruby. Esta reconstrucción automática permite
evitar desfases entre la definición de los tipos entre los dos sistemas y
agilizo el desarrollo del sistema \texttt{PEAT}. Esta reconstrucción automática
se hace por medio de un lenguaje de dominio especifico que permite implementar
modelos basados en tipos del servicio web C3 aprovechando que este servicio
es tipo RESTful y las capacidades de metaprogramación de Ruby.

\texttt{Bezel} tiene dos modos principales de uso:
\begin{itemize}
\item Como un cliente REST con ciertas adecuaciones para una mejor
  integración con el API del servidor C3.
\item Como un lenguaje de dominio especifico interno para Ruby para la
  implementación automatizada de tipos y sus asociaciones del servicio web C3.
\end{itemize}

\subsection{Servicio web C3}

El servicio web C3 es de tipo RESTful es decir cada recurso o tipo tiene una única
dirección URL asociada. Las direcciones URL tienen la siguiente forma:

\vspace{2.5mm}
\texttt{POST} /api/$<$\textit{version}$>$/$<$\textit{tenant}$>$/$<$\textit{module}$>$;$<$\textit{tag}$>$/$<$\textit{type}$>$?action=$<$\textit{action-name}$>$

\begin{itemize}
\item \textit{version}: indica la versión del servicio a la cual se esta accediendo,
  el sistema \texttt{PEAT} hace uso de la versión 1.0.
\item \textit{tenant}: indica el producto o cliente, esto es porque cada
  producto y/o cliente cierta funcionalidad especifica. Para \texttt{PEAT} se tiene
  \textquote{peat} y \textquote{c3}.
\item \textit{module}: indica el módulo al que se esta accediendo, un conjunto
  de tipos se definen en un módulo en particular, para el sistema \texttt{PEAT} el
  módulo mas usado es el módulo \textquote{peat} pero también se hace uso
  de los módulos \textquote{billing} y \textquote{structure}.
\item \textit{tag}: indica el tipo de ambiente, es decir si el servicio web es para
  ambiente de pruebas, control de calidad, producción, etcétera.
  \textquote{peatprod} y \textquote{peatqa} son algunas de las \textit{tags}
  que se usaron para \texttt{PEAT}.
\item \textit{type}: indica el tipo al que se quiere acceder, \textquote{Building},
  \textquote{Location} y \textquote{BuildingEnergyConservationOption} son algunos
  de los tipos que se usaron para \texttt{PEAT}.
\item \textit{action-name}: indica la acción que se quiere realizar sobre
  el tipo indicado anteriormente. \textquote{fetch}, \textquote{upsert} y
  \textquote{remove} son unas de las acciones que se pueden realizar.
\end{itemize}

Como se comento anteriormente el servicio web C3 no es totalmente de tipo RESTful
principalmente porque se hace uso del mismo método HTTP POST para todas las acciones,
haciendo uso del \textquote{action-name} para ejecutar la acción correspondiente.
Para la implementación de \texttt{Bezel} solamente se espera que el servicio web
tenga direcciones únicas para cada tipo definido.

Muchas acciones aceptan parámetros los cuales son almacenados dentro del cuerpo
de la petición POST, por lo que el único parámetro que se manda por medio
de la URL es el nombre la acción a realizar.

\subsection{Arquitectura General}

La biblioteca \texttt{Bezel} esta conformada por un cinco clases y cinco módulos
siendo las clases principales \texttt{Bezel::Client} y \texttt{Bezel::Base}.

En el lenguaje de programación Ruby se tiene el concepto de módulo (\texttt{module}),
el cual no es mas que una agrupación de objetos bajo un único nombre. Los objetos
pueden ser constantes, métodos, clases y otros módulos.

Los módulos tienen dos usos: se pueden utilizar un módulo como una manera conveniente
de agrupar objetos o se puede incorporar los objetos del módulo a una clase haciendo
uso de \texttt{include} o \texttt{extend}.

\lstinputlisting[language=Ruby]{code/module-namespace.rb}

El ejemplo anterior muestra el uso de los módulos como un espacio de nombre,
en la linea 2 se define la constante \texttt{PI} y en las lineas 3 y 7 se definen
los métodos \texttt{sin} y \texttt{cos}, estos métodos son usados fuera del
módulo por medio del nombre del módulo como ocurre en la linea 12, en esta
linea se hace referencia tanto a la constante \texttt{PI} como al método
\texttt{sin}.

\lstinputlisting[language=Ruby]{code/module-include.rb}

Ahora en este ejemplo se muestra el uso de los módulos para incorporar métodos
de clase o de instancia. En las lineas 1-11 se define el módulo \texttt{Boolean}
el cual define dos métodos \texttt{boolean?} y \texttt{to\_bool}, mientras que
en las lineas 13-17 se define el módulo \texttt{RandomString} que define un método
\texttt{random}. En las lineas 19-22 no se esta definiendo una nueva clase
\texttt{String} si no que Ruby permite modificar cualquier clase en tiempo de
ejecución aunque esta clase sea parte de la biblioteca estándar, así en la linea 20
se hace uso de \texttt{include} para agregar los métodos del módulo dado como
parámetro como métodos de instancia, el uso de estos métodos se como se ver en las
lineas 24-25. Finalmente en la linea 21 se hace uso de la declaración \texttt{extend}
para agregar los métodos del módulo dado como métodos de clase como se puede
ver en la linea 26.

Por lo anterior se tiene que el uso de módulos es una parte importante para
organizar y compartir funcionalidad en el lenguaje de programación Ruby.
Para la biblioteca \texttt{Bezel} tenemos el módulo \texttt{Bezel} que contiene
a todas las clases y módulos de la biblioteca. Además se tienen cinco submódulos
que permiten organizar de forma coherente la funcionalidad de la biblioteca.

Las clases y módulos que conforman la biblioteca son
(Ver Figura \ref{fig:diagrama-clase}):
\begin{itemize}
\item \texttt{Bezel}: módulo principal que contiene a los demás submódulos y
  clases de la biblioteca.
\item \texttt{Bezel::Config}: módulo que define los métodos y valores validos
  para configurar el cliente con el servicio web C3.
\item \texttt{Bezel::Client}: clase que define un cliente para usar el servicio
  web C3.
\item \texttt{Bezel::Target}: clase que representa un tipo dado un contexto
  especifico (\textit{tenant}, \textit{tag}, \textit{module}).
\item \texttt{Bezel::Action}: clase que representa una acción sobre
  un tipo representado por un \texttt{Bezel::Target}.
\item \texttt{Bezel::Base}: clase abstracta que permite definir un modelo en Ruby
  en base a un tipo del servido web C3.
\item \texttt{Bezel::Connections}: módulo que define las acciones básicas y de
  búsqueda sobre un modelo.
\item \texttt{Bezel::Associations}: módulo que define las asociaciones entre
  modelos.
\item \texttt{Bezel::CacheBase}: clase que redefine las acciones básicas y
  de búsqueda sobre un modelo haciendo uso de un cache \texttt{Bezel::Cache}.
\item \texttt{Bezel::Cache}: clase que define un API para realizar operaciones
  de lectura/escritura sobre un cache.
\end{itemize}

\rjcimage{1.0}{imagenes/Diagrama-Clase-Bezel.png}{Diagrama de clase de la biblioteca
  Bezel}{diagrama-clase}

\subsubsection{Dependencias}

En la implementación de \texttt{Bezel} se hace uso de las siguiente bibliotecas
de apoyo:

\begin{itemize}
\item moneta: proporciona un API uniforme para realizar
  operaciones de lectura y escritura en un repositorio de cache.
\item faraday: proporciona un API uniforme para realizar
  peticiones HTTP.
\item hashie: proporciona un conjunto de metodos que extienden la funcionalidad
  de la clase \texttt{Hash} en Ruby.
\item activemodel: proporciona interfaces para modelos que deben operar sobre
  Rails.
\end{itemize}

\subsection{\texttt{Bezel}}

El módulo \texttt{Bezel} no solo sirve como el espacio de nombres principal para la
biblioteca si no que además implementa varios métodos auxiliares para facilitar la
configuración y creación de clientes para acceder al servicio web C3.

\lstinputlisting[language=Ruby]{code/bezel.rb}

En el código anterior se tiene la implementación abreviada de \texttt{Bezel},
en la linea 2 por medio de \texttt{extend} se añaden varios métodos de clase
para configurar la biblioteca. En las lineas 4-20 se definen los
métodos auxiliares mas importantes:

\begin{itemize}
\item \texttt{new} (lineas 5-7): regresa una nueva instancia \texttt{Bezel::Client}.
\item \texttt{invoke} (lineas 9-11): realiza una petición al servicio web C3
  por medio del cliente asociado al módulo \texttt{Bezel}.
\item \texttt{client} (lineas 13-15): usando una variable de clase se inicializa o
  reusa una instancia \texttt{Bezel::Client}. Este método permite reutilizar un
  mismo cliente y conexión al servidor C3 evitando el costo de establecer un
  nuevo cliente y en establecer una conexión HTTP con el servidor C3.
\item \texttt{context} (lineas 17-19): permite cambiar el contexto de las
  peticiones que se realizan dentro de un bloque, se regresa al contexto original
  al finalizar de ejecutar el código dentro del bloque.
\end{itemize}

\subsubsection{\texttt{Bezel::Config}}

El módulo \texttt{Bezel::Config} define los atributos que permiten configurar
el comportamiento de la biblioteca y de los clientes que se obtienen por medio
de ella. Este módulo se usa en \texttt{Bezel} para definir una configuración global
de la biblioteca como en \texttt{Bezel::Client} para tener también una configuración
local por cliente.

\lstinputlisting[language=Ruby]{code/config.rb}

En el módulo \texttt{Config} se definen dos constantes:

\begin{itemize}
\item \texttt{VALID\_OPTIONS\_KEYS} (lineas 4-13): un arreglo de símbolos que indica
  que atributos son validos.
\item \texttt{DEFAULT\_VALUES} (lineas 15-24): una tabla hash donde se indican los
  valores por defecto de los atributos validos.
\end{itemize}

Haciendo uso del método \texttt{attr\_accessor} en la linea 26 se definen por cada
símbolo en \texttt{VALID\_OPTIONS\_KEYS} una atributo de instancia con sus
correspondientes métodos de acceso y escritura. El método \texttt{attr\_accessor} es
un ejemplo del uso de metaprogramación en Ruby, dado que en base al símbolo dado se
genera un atributo y sus métodos de acceso.

% FIXME: Buscar traducción para callback

En la linea 30 se tiene el método \texttt{extended} el cual es un callback que es
llamado cuando un módulo es extendido, es decir es utilizado como parámetro al
método \texttt{extend}. Para el módulo \texttt{Config} se tiene que cuando el módulo
es extendido, como es en el caso de \texttt{Config} dentro del módulo \texttt{Bezel},
se llama al método \texttt{reset} para inicializar todos los atributos definidos en
el módulo con sus valores por defecto. El método \texttt{extended} es solo uno
de varios métodos callback que proporciona Ruby para la metaprogramación.

Otro método muy importante para la metaprogramación es \texttt{send} el cual es usado
en la linea 35 para inicializar los atributos definidos por \texttt{Config}, el
método \texttt{send} invoca el método identificado con un símbolo o cadena dada,
pasando los argumentos posteriores como argumentos a la invocación del método.
Haciendo uso de \texttt{send} y de las dos constantes definidas en el modulo
se puede inicializar cada uno de los atributos.

\subsection{\texttt{Bezel::Client}}

La clase \texttt{Client} representa un cliente que permite realizar peticiones al 
servicio web C3, por medio del módulo \texttt{Config} se permite configurar
cada cliente instanciado.

\lstinputlisting[language=Ruby]{code/client.rb}

En la linea 3 se tiene un \texttt{include} que añade las constantes
y atributos para configurar al cliente. En las lineas 7-13 se define el método
\texttt{conn} que inicializa una conexión HTTP según la configuración definida
para el cliente y pidiendo que las respuestas mandadas por el servidor
sean de tipo JSON. 

En las lineas 13-49 se tiene la implementación del metodo \texttt{invoke}, en 
la linea 14 se define un \textit{target} de tipo \texttt{Bezel::Target} usando 
tanto el contexto del cliente, es decir el \textit{tenant} y \textit{tag} mas 
la información del \textit{module} y \textit{type} dados en los argumentos.
En la linea 15 se define una acción de tipo \texttt{Bezel::Action} que representa
una acción sobre un tipo con sus parámetros.

En la linea 17 se define la URL a la cual se hará la petición al servicio web C3,
haciendo uso de la información de los objetos previos, se convierte los parámetros
de la acción al formato JSON para ser mandados en el cuerpo de la petición POST
al servicio web.

% FIXME: explicar linea 32 de addCredentials

En las lineas 22 a 30 se definen los encabezados y cuerpo de la petición POST al
servicio web. Finalmente en la linea 37 se realiza la petición POST por medio
de la conexión creada por el método \texttt{conn} y con los para metros definidos
anteriormente.

Posteriormente el resultado de la petición es almacenado en el objeto \texttt{action}
y este objeto es regresado como resultado de la invocación del método
\texttt{invoke}.

\subsection{\texttt{Bezel::Base}}

\texttt{Bezel::Base} es una clase abstracta la cual define la mayor parte de su
funcionalidad en base al nombre de la clase que hereda de ella. Es decir no se hace
uso directo de \texttt{Bezel::Base} si no que al heredar de esta clase abstracta
se obtiene funcionalidad para crear modelos basados en tipos provenientes
del servicio web C3.

\lstinputlisting[language=Ruby]{code/base-example.rb}

El código anterior ejemplifica el uso de \texttt{Bezel::Base} para definir el modelo
por medio de herencia, en la linea 1 se define \texttt{Recommendation}. En la linea 2
se tiene el método de clase \texttt{set\_c3\_type} que permite al modelo definir
a que tipo del sistema web C3 esta ligado,  en este caso el modelo esta ligado al
tipo \texttt{BuildingEnergyConservationOption}. En la linea 4 se define que el modelo
tiene una asociación con el modelo \texttt{EnergyConservationOption} en forma única
por medio del metodo \texttt{has\_one}. Ya definido el modelo en la linea 7 se 
hace la búsqueda de una recomendación por medio de su identificador único por medio
del método \texttt{find}. En la linea 8 se obtiene el costo total de la recomendación
buscada anteriormente por medio de \texttt{totalCost}. Los métodos
\texttt{set\_c3\_type}, \texttt{has\_one} y \texttt{find} son definidos por
\texttt{Bezel::Base} pero el método \texttt{totalCost} igual es definido en tiempo
de ejecución por medio de la metaprogramación.

% \lstinputlisting[language=Ruby]{code/base-extend-include.rb}

\subsubsection{Métodos de clase}

Los métodos de clase que se definen en \texttt{Bezel::Base} son para
definir métodos de configuración del modelo y para realizar operaciones
de búsqueda sobre el modelo.

\lstinputlisting[language=Ruby]{code/base-class-methods.rb}

Los métodos principales son:
\begin{itemize}
  \item \texttt{config\_option} (lineas 6-19): este método permite la definición
    de nuevos atributos sobre el modelo en tiempo de ejecución por medio del uso
    de método \texttt{define\_method} que permite crear métodos en tiempo
    de ejecución, con este método y de \texttt{attr\_writer} se definen tanto
    atributos de clase como de instancia. El método toma dos parámetros
    el primero es el nombre de la configuración a crear y el segundo
    es el valor por defecto de la configuración.
  \item \texttt{find} (linea 21): este método permite realizar búsquedas sobre
    el modelo por medio de un identificador, también permite realizar búsquedas
    mas avanzadas por medio de parámetros como \texttt{filter}.
  \item \texttt{invoke} (lineas 25-27): este método permite realizar
    un acción sobre el modelo, como se puede ver en la linea 26 se hace uso del
    metodo \texttt{invoke} del modulo \texttt{Bezel} haciendo
    uso del contexto del modelo, es decir su modulo y tipo correspondiente.
  \item \texttt{inherited} (lineas 29-34): este método es parte de API de
    metaprogramacion de Ruby, el metodo es invocado cuando una subclase de
    \texttt{Bezel::Base} es creada. La subclase es pasada como el parametro
    \texttt{klass} y por medio del metodo \texttt{config\_option} se definen
    tres configuraciones:
    \begin{itemize}
    \item \texttt{c3\_type} (linea 30): define el tipo asociado al modelo
      siendo el nombre de la subclase su valor por defecto.
    \item \texttt{c3\_module} (linea 31): define el modulo asociado al modelo
      siendo su valor por defecto el modulo \textquote{peat}.
    \item \texttt{c3\_include} (linea 32): define si se hace una carga adelantada
      de información de las asociaciones del modelo. El valor por defecto es que
      no se haga ninguna carga adelantada.
    \end{itemize}
    Finalmente en la linea 33 se configura que el modelo por defecto no hace uso
    de un cache.
    % FIXME: Hablar sobre los tipos de cache
  \item \texttt{c3\_cached} (lineas 36 a 42): este método permite indicar
    que el modelo hace uso de un cache y las opciones de configuración para este.
    Para activar el cache es necesario tener una instancia de cache definida en
    \texttt{Bezel.cache}, si esto ocurre entonces se hace un \texttt{include}
    del modulo \texttt{Bezel::CacheBase} (linea 38) que agrega el uso y
    mantenimiento del cache a las acciones básicas y de búsqueda del modelo.
\end{itemize}

%\lstinputlisting[language=Ruby]{code/base-instance-methods.rb}

%\subsubsection{\texttt{Bezel::Associations}}

%\lstinputlisting[language=Ruby]{code/associations.rb}

%\subsubsection{\texttt{Bezel::Connections}}

%\lstinputlisting[language=Ruby]{code/connections.rb}

%\subsubsection{\texttt{Bezel::CacheBase}}

%\lstinputlisting[language=Ruby]{code/cache_base.rb}

\chapter{Despliegue a producción}

La etapa para desplegar una aplicación es una etapa de creciente importancia ya
que por un lado hace algunos años las aplicaciones web se desarrollaban bajo el
modelo de tres capas: la base de datos, el servidor de la aplicación y el servidor
web; actualmente se tienen mas servicios que deben estar en línea sobre todo para
garantizar un servicio concurrente; caches, equilibradores de carga, servidores
de cola, etcétera. Por lo que automatizar el despliegue de una aplicación en sus
diferentes contextos, desarrollo, producción y pruebas, es de vital importancia
para el desarrollo de software en tiempo y en forma.

\section{Despliegue continuo}

El despliegue continuo (\textit{Continuous Delivery}, CD) es una practica de software
en la cual se implementa software de tal forma de que éste pueda ser desplegado a
producción en cualquier momento\cite{27_martin_fowler_cd}. Su objetivo es crear,
probar y liberar software más rápido y con mayor frecuencia.

Se considera que un sistema implementa CD de forma correcta cuando
\cite{27_martin_fowler_cd}:
\begin{itemize}
\item El sistema puede ser desplegado a producción en cualquier momento del
  ciclo de desarrollo.
\item Se tiene una rápida retroalimentación sobre la capacidad del sistema
  para ser desplegado en producción al realizarse cambios en el sistema.
\item Se puede desplegar en forma fácil cualquier versión del software para
  cualquier entorno.
\end{itemize}

Los principales beneficios de esta practica son:
\begin{itemize}
\item Reducción del riesgo de despliegue: dado que se realizan despliegues
  a producción de forma recurrente los cambios contenidos en cada despliegue
  son menores por lo que hay menores posibilidades de errores y si un error
  se presenta en mas fácil de arreglar.
\item Retroalimentación del usuario: uno de los mayores riesgos en la ingeniería
  del software es el desarrollar un sistema que no es útil para el cliente o usuario.
  Por lo que obtener retroalimentación sobre el desarrollo del sistema lo mas
  rápido posible y de forma frecuente permite descubrir que tan valioso es el sistema
  al usuario.
\end{itemize}

En el despliegue continuo se tiene el concepto de \textit{deployment pipeline}
el cual es un conjunto de validaciones por las que el sistema debe pasar para
llegar a producción. El propósito de un \textit{deployment pipeline} es:
\begin{itemize}
\item Visibilidad: todas las etapas del sistema de despliegue son visibles
  para todos los miembros del equipo.
\item Retroalimentación: Los miembros del equipo obtiene información sobre los
  problemas cuando se producen de modo que son capaces de solucionarlos lo mas
  rápido posible.
\item Despliegue: por medio de un proceso totalmente automatizado se puede
  desplegar y liberar cualquier versión del sistema en cualquier ambiente.
\end{itemize}

Los riesgos de lograr un despliegue a producción en \texttt{PEAT} eran mas grandes
de lo habitual no solamente por la cantidad de requerimientos y la complejidad
del sistema si no que además PG\&E requería la implementación de varios ambientes
previos a desplegar el sistema en producción. Así para \texttt{PEAT} el implementar
el despliegue continuo fue de vital importancia para cumplir con los necesidad
del cliente.

Para PG\&E era necesario que se tuvieran los siguientes ambientes:
\begin{itemize}
\item \textit{peattest}: este ambiente es para un grupo de usuarios seleccionados
  por PG\&E para obtener retroalimentación sobre el sistema.
\item \textit{peatqa}: este ambiente es para un equipo de PG\&E para realizar
  control de calidad.
\item \textit{peatprod}: este ambiente es el ambiente final de produccion.
\end{itemize}

Para uso interno en C3 Energy se tenia además el ambiente \textit{peatstage}
el cual era usado para realizar pruebas de rendimiento y de control de calidad.

En total se tenían cuatro ambientes, los cuales tenían que tener la misma
configuración al ambiente de producción. Dados estos requerimientos la implementación
de despliegue continuo para el sistema \texttt{PEAT} fue una decisión
casi natural para permitir manejar el despliegue del sistema para cualquier
combinación de versión-ambiente requerido.


La integración continua es una practica de software en la cual los integrantes
de un proyecto integran su trabajo (código, diseño, etcétera) de forma frecuente
realizando un despliegue al ambiente de producción al menos una vez al día
\cite{26_martin_fowler_ci}. Esta practica busca reducir de forma significativa los
problemas de integración, los cuales se incrementan en sistemas de gran tamaño
que tienen varios módulos que interactúan entre si.

Cada integración es verificada por medio de un constructor automatizado, el cual
hace uso de las pruebas del sistema\footnote{Pruebas de unidad, funcional e
  integración} para detectar errores de integración lo más rápido posible.


\backmatter

\chapter{Conclusiones}

En este documento se expuso el diseño e implementación del sistema \texttt{PEAT} que
permite a sus usuarios el identificar y monitorear su consumo energético. En C3
Energy se tienen varios sistemas en producción haciendo uso de lenguajes de
programación como Java y Javascript y herramientas como Rhino y Ext JS.

En la implementación de \texttt{PEAT} se presentaron nuevas retos como la necesidad
de una interfaz simple y de rápida respuesta a las acciones del usuario y
dar servicio a un numero considerable de usuarios.
Estos requerimientos no se podían solucionar de forma óptima haciendo uso de los
lenguajes y herramientas usadas hasta ese momento. Es por eso que al iniciar
la implementación del sistema \texttt{PEAT} se eligió tener un ambiente de
desarrollo multi-lenguaje en el cual se hace uso de un lenguaje de programación
o herramienta en los contextos en que éstos dan la mayor ventaja. Un ambiente
multi-lenguaje trae consigo sus propios retos como la integración con sistemas ya
existentes y el despliegue exitoso de los sistemas a producción.

Haciendo uso de las capacidades de metaprogramación del lenguaje de programación
Ruby y de la creación de un lenguaje de dominio específico en la biblioteca
\texttt{Bezel} se facilitó la implementación de los modelos necesarios para
el funcionamiento del sistema de forma eficiente y con la flexibilidad necesaria
para lograr la integración del sistema \texttt{PEAT} con los subsistemas ya
existentes, permitiendo al programador definir modelos y sus asociaciones con
pocas lineas de código.

El uso de lenguajes de dominio específico fue el aspecto mas sobresaliente
para el desarrollo exitoso del sistema \texttt{PEAT} por su ayuda en la
integración con los subsistemas ya existentes y en la automatización de la
configuración y despliegue del sistema.

El uso del despliegue continuo y de pruebas con usuarios finales fueron vitales para
que el sistema cumpliera con su objetivo de dar la mayor utilidad posible
a los usuarios PyMES sobre su consumo energético. La retroalimentación continúa
obtenida por los ambientes de prueba permitió detectar deficiencias tanto en el
\texttt{backend} como en la interfaz del usuario, siendo la principal deficiencia
detectada la dificultad de ingreso de la información inicial para crear un perfil
de un edificio. Gracias a esta retroalimentación se pudo obtener una interfaz
mas intuitiva para el usuario sin afectar el tiempo de entrega.

También por medio del despliegue continuo se logró tener un lanzamiento a la etapa
de producción sin problemas, dado el que con anterioridad se realizaron varios
despliegues en ambientes de prueba los cuales eran idénticos al ambiente de
producción.

\pagebreak

El sistema \texttt{PEAT} fue recibido muy bien por los usuarios y el cliente PG\&E
ya que desde hace cuatro años se encuentra en funcionamiento dando servicio a cerca
de 250,000 usuarios PyMES a la fecha\cite{30_pge_annual_report}. Su éxito provocó que
el sistema fuera también adoptado por otras compañías proveedoras de electricidad y
gas como San Diego Gas \& Electric\cite{32_reuters_c3} y Southern California Edison,
que prácticamente abarca el estado de California en los Estados Unidos de
América\cite{31_energy_map}.


\printbibliography

\end{document}
