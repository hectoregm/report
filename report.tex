\documentclass{article}
\usepackage[spanish]{babel}
\usepackage[utf8]{inputenc}
\author{Héctor E. Gómez Morales}
\title{Reporte Trabajo Profesional: Proyecto PEAT (Progressive Energy
  Audit Tool)}
\begin{document}
\maketitle
\tableofcontents
\section{Introducción}
Este reporte se enfoca al diseño e implementacion de la aplicacion
PEAT (Progressive Energy Audit Tool) que es una aplicacion web que
ayuda a usuarios de PyMES identificar y monitorear su gasto en
energia, para de esta forma darle un mayor control en sus costos de
energia.

PEAT es producto de una licitacion auspiciada por Pacific Gas and
Electric Company (PG\&E), proveedora de gas natural y electricidad
para casi dos tercios del norte de California, USA, para
el desarrollo de una aplicacion web enfocada a usuarios PyMES.

Su desarrollo era de importancia critica para PG\&E y al gobierno de
California puesto que su funcionamiento era un requisito en una nueva
ley de facturacion de energia electrica en 2013. La nueva ley buscaba
lograr distribuir la carga de la red electrica, una de las formas
principales para lograr esto era desincentivar el uso de la red
electrica en horas picos al darle la facultad a las utilidades como PG\&E
de cobrar tasas mucho mas altas en estas horas.
\end{document}
