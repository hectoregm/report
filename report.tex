\documentclass{article}
\usepackage[spanish]{babel}
\usepackage[utf8]{inputenc}
\author{Héctor E. Gómez Morales}
\title{Reporte Trabajo Profesional: Proyecto PEAT (Progressive Energy
  Audit Tool)}
\begin{document}
\maketitle
\tableofcontents
\section{Introducción}
Este reporte se enfoca al diseño e implementacion de la aplicacion
PEAT (Progressive Energy Audit Tool) que es una aplicacion web que
ayuda a usuarios de PyMES identificar y monitorear su gasto en
energia, para de esta forma darle un mayor control en sus costos de
energia.

PEAT es producto de una licitacion auspiciada por Pacific Gas and
Electric Company (PG\&E), proveedora de gas natural y electricidad
para casi dos tercios del norte de California, USA, para
el desarrollo de una aplicacion web enfocada a usuarios PyMES.

Su desarrollo era de importancia critica para PG\&E y al gobierno de
California puesto que su funcionamiento era un requisito en una nueva
ley de facturacion de energia electrica en 2013. La nueva ley buscaba
lograr distribuir la carga de la red electrica, una de las formas
principales para lograr esto era desincentivar el uso de la red
electrica en horas picos al darle la facultad a las utilidades como PG\&E
de cobrar tasas mucho mas altas en estas horas.
\section{Planteamiento del problema}
Aunque PG\&E tiene todos los datos de consumo electrico y/o gas de
sus clientes, y dado el despliegue previo de los nuevos medidores
inteligentes en la mayor parte de sus clientes se tenia acceso a
informacion muy detallada del consumo de energia.

Para darle la mayor informacion y valor a las empresas era
vital obtener mayor contexto de su entorno de operacion: numero
de edificios asociados a la cuenta, rubro de la empresa, numero de
empleados, etc. Entre mayor informacion se pudiera captar sobre la
empresa el sistema daria un desglose mas detallado y util de sus
consumos de energia. El objetivo era dar la mayor utilidad posible con
la menor informacion disponible pero fomentando al usuario el dar
mas informacion para darle un mejor monitoreo de su consumo.

La compañia C3 Energy, en la que trabaje y que gano esta licitacion,
contaba con una infraestructura para el procesamiento de una gran
cantidad de datos de consumo de energia puesto que tenia un sistema
de monitoreo de consumo de energia pero enfocada a empresas
multinacionales. El reto ahora era que pasaba de tener una docena
de clientes, con los cuales se trataba directamente, a lidear con
cientos de miles de empresas PyMES en las cuales se obtenia una
parte de la informacion por parte de PG\&E y otra parte por el
empresario.
\end{document}
