\chapter{Despliegue a producción}

La etapa para desplegar una aplicación es una etapa de creciente importancia ya
que por un lado hace algunos años las aplicaciones web se desarrollaban bajo el
modelo de tres capas: la base de datos, el servidor de la aplicación y el servidor
web; actualmente se tienen mas servicios que deben estar en línea sobre todo para
garantizar un servicio concurrente; caches, equilibradores de carga, servidores
de cola, etcétera. Por lo que automatizar el despliegue de una aplicación en sus
diferentes contextos, desarrollo, producción y pruebas, es de vital importancia
para el desarrollo de software en tiempo y en forma.

\section{Despliegue continuo}

El despliegue continuo (\textit{Continuous Delivery}, CD) es una práctica de software
en la cual se implementa software de tal forma de que éste pueda ser desplegado a
producción en cualquier momento\cite{27_martin_fowler_cd}. Su objetivo es crear,
probar y liberar software más rápido y con mayor frecuencia.

\vspace{2.5mm}

Se considera que un sistema implementa el despliegue continuo cuando
\cite{27_martin_fowler_cd}:
\begin{itemize}
\item El sistema puede ser desplegado a producción en cualquier momento del
  ciclo de desarrollo.
\item Se tiene una rápida retroalimentación sobre la capacidad del sistema
  para ser desplegado en producción al realizarse cambios en el sistema.
\item Se puede desplegar de forma sencilla cualquier versión del software para
  cualquier entorno.
\end{itemize}

Los principales beneficios de esta práctica son:
\begin{itemize}
\item Reducción del riesgo de despliegue: dado que se realizan despliegues
  a producción de forma recurrente los cambios contenidos en cada despliegue
  son menores por lo que hay menores posibilidades de errores y si un error
  se presenta en mas fácil de localizar y arreglar.
\item Retroalimentación del usuario: uno de los mayores riesgos en la ingeniería
  del software es el desarrollar un sistema que no es útil para el cliente o usuario.
  Por lo tanto obtener retroalimentación sobre el desarrollo del sistema de forma
  rápida y frecuente permite evaluar que tan útil es el sistema para el usuario.
\end{itemize}

En \texttt{PEAT} los riesgos de lograr un despliegue a producción exitoso
eran mas grande de lo habitual dada la complejidad del sistema y la cantidad
de requerimientos. Además PG\&E requería la implementación de varios ambientes
previos al despliegue del sistema en el ambiente de producción, entonces
el implementar el proceso de despliegue continuo para \texttt{PEAT} fue de vital
importancia para cumplir con las necesidades del cliente.

Para PG\&E era necesario que se tuvieran los siguientes ambientes:
\begin{itemize}
\item \textit{peattest}: este ambiente es para un grupo de usuarios seleccionados
  por PG\&E para obtener retroalimentación sobre el sistema.
\item \textit{peatqa}: este ambiente es para un equipo de PG\&E para realizar
  control de calidad y de rendimiento.
\item \textit{peatprod}: este ambiente es el ambiente final de producción.
\end{itemize}

Para uso interno en C3 Energy se tenia además el ambiente \textit{peatstage}
el cual era usado para realizar pruebas de rendimiento y de control de calidad.

En total se tenían cuatro ambientes, los cuales tenían que tener que ser lo mas
parecidos a la configuración final del ambiente de producción para que las pruebas
de calidad y de rendimiento generaran información que fuera lo mas cercano al
comportamiento del sistema en el ambiente de producción. Dados estos requerimientos
la implementación del proceso de despliegue continuo para el sistema \texttt{PEAT}
fue una decisión casi natural para permitir manejar el despliegue del sistema para
cualquier combinación de versión-ambiente requerido.

\subsection{Proceso de despliegue}

En el despliegue continuo se tiene al proceso de despliegue (\textit{deployment
  pipeline}) como concepto central, el cual en esencia es la implementación
automatizada de los procesos de configuración, construcción, prueba y despliegue de
un sistema \cite{28_humble_farley_2011}.

Este proceso otorgar las siguientes ventajas:
\begin{itemize}
\item Visibilidad: todas las etapas del sistema de despliegue son visibles
  para todos los miembros del equipo.
\item Retroalimentación: Los miembros del equipo obtiene información sobre los
  problemas cuando se producen de modo que son capaces de solucionarlos lo mas
  rápido posible.
\item Despliegue: por medio de un proceso totalmente automatizado se puede
  desplegar y liberar cualquier versión del sistema en cualquier ambiente.
\end{itemize}

El proceso de despliegue se conforma de varias de etapas.

%% La integración continua es una practica de software en la cual los integrantes
%% de un proyecto integran su trabajo (código, diseño, etcétera) de forma frecuente
%% realizando un despliegue al ambiente de producción al menos una vez al día
%% \cite{26_martin_fowler_ci}. Esta practica busca reducir de forma significativa los
%% problemas de integración, los cuales se incrementan en sistemas de gran tamaño
%% que tienen varios módulos que interactúan entre si.

%% Cada integración es verificada por medio de un constructor automatizado, el cual
%% hace uso de las pruebas del sistema\footnote{Pruebas de unidad, funcional e
%%   integración} para detectar errores de integración lo más rápido posible.
