\chapter{Despliegue a producción}

La etapa para desplegar una aplicación es una etapa de creciente importancia ya
que por un lado hace algunos años las aplicaciones web se desarrollaban bajo el
modelo de tres capas: la base de datos, el servidor de la aplicación y el servidor
web; actualmente se tienen mas servicios que deben estar en línea sobre todo para
garantizar un servicio concurrente; caches, equilibradores de carga, servidores
de cola, etcétera. Por lo que automatizar el despliegue de una aplicación en sus
diferentes contextos, desarrollo, producción y pruebas, es de vital importancia
para el desarrollo de software en tiempo y en forma.

\section{Despliegue continuo}

El despliegue continuo es una practica de software en la cual los integrantes
de un proyecto integran su trabajo (código, diseño, etcétera) de forma frecuente
realizando un despliegue al ambiente de producción al menos una vez al día
\cite{26_martin_fowler_ci}. Esta practica busca reducir de forma significativa los
problemas de integración, los cuales se incrementan en sistemas de gran tamaño
que tienen varios módulos que interactúan entre si.

Cada integración es verificada por medio de un constructor automatizado, el cual
hace uso de las pruebas del sistema\footnote{Pruebas de unidad, funcional e
  integración} para detectar errores de integración lo más rápido posible.
