\chapter{Despliegue a producción}

La etapa para desplegar una aplicación es una etapa de creciente importancia ya
que por un lado hace algunos años las aplicaciones web se desarrollaban bajo el
modelo de tres capas: la base de datos, el servidor de la aplicación y el servidor
web; actualmente se tienen mas servicios que deben estar en línea sobre todo para
garantizar un servicio concurrente; caches, equilibradores de carga, servidores
de cola, etcétera. Por lo que automatizar el despliegue de una aplicación en sus
diferentes contextos, desarrollo, producción y pruebas, es de vital importancia
para el desarrollo de software en tiempo y en forma.

\section{Despliegue continuo}

El despliegue continuo (\textit{Continuous Delivery}, CD) es una practica de software
en la cual se implementa software de tal forma de que éste pueda ser desplegado a
producción en cualquier momento\cite{27_martin_fowler_cd}. Su objetivo es crear,
probar y liberar software más rápido y con mayor frecuencia.

Se considera que un sistema implementa CD de forma correcta cuando
\cite{27_martin_fowler_cd}:
\begin{itemize}
\item El sistema puede ser desplegado a producción en cualquier momento del
  ciclo de desarrollo.
\item Se tiene una rápida retroalimentación sobre la capacidad del sistema
  para ser desplegado en producción al realizarse cambios en el sistema.
\item Se puede desplegar en forma fácil cualquier versión del software para
  cualquier entorno.
\end{itemize}

Los principales beneficios de esta practica son:
\begin{itemize}
\item Reducción del riesgo de despliegue: dado que se realizan despliegues
  a producción de forma recurrente los cambios contenidos en cada despliegue
  son menores por lo que hay menores posibilidades de errores y si un error
  se presenta en mas fácil de arreglar.
\item Retroalimentación del usuario: uno de los mayores riesgos en la ingeniería
  del software es el desarrollar un sistema que no es útil para el cliente o usuario.
  Por lo que obtener retroalimentación sobre el desarrollo del sistema lo mas
  rápido posible y de forma frecuente permite descubrir que tan valioso es el sistema
  al usuario.
\end{itemize}

En el despliegue continuo se tiene el concepto de \textit{deployment pipeline}
el cual es un conjunto de validaciones por las que el sistema debe pasar para
llegar a producción. El propósito de un \textit{deployment pipeline} es:
\begin{itemize}
\item Visibilidad: todas las etapas del sistema de despliegue son visibles
  para todos los miembros del equipo.
\item Retroalimentación: Los miembros del equipo obtiene información sobre los
  problemas cuando se producen de modo que son capaces de solucionarlos lo mas
  rápido posible.
\item Despliegue: por medio de un proceso totalmente automatizado se puede
  desplegar y liberar cualquier versión del sistema en cualquier ambiente.
\end{itemize}

Los riesgos de lograr un despliegue a producción en \texttt{PEAT} eran mas grandes
de lo habitual no solamente por la cantidad de requerimientos y la complejidad
del sistema si no que además PG\&E requería la implementación de varios ambientes
previos a desplegar el sistema en producción. Así para \texttt{PEAT} el implementar
el despliegue continuo fue de vital importancia para cumplir con los necesidad
del cliente.

Para PG\&E era necesario que se tuvieran los siguientes ambientes:
\begin{itemize}
\item \textit{peattest}: este ambiente es para un grupo de usuarios seleccionados
  por PG\&E para obtener retroalimentación sobre el sistema.
\item \textit{peatqa}: este ambiente es para un equipo de PG\&E para realizar
  control de calidad.
\item \textit{peatprod}: este ambiente es el ambiente final de produccion.
\end{itemize}

Para uso interno en C3 Energy se tenia además el ambiente \textit{peatstage}
el cual era usado para realizar pruebas de rendimiento y de control de calidad.

En total se tenían cuatro ambientes, los cuales tenían que tener la misma
configuración al ambiente de producción. Dados estos requerimientos la implementación
de despliegue continuo para el sistema \texttt{PEAT} fue una decisión
casi natural para permitir manejar el despliegue del sistema para cualquier
combinación de versión-ambiente requerido.


La integración continua es una practica de software en la cual los integrantes
de un proyecto integran su trabajo (código, diseño, etcétera) de forma frecuente
realizando un despliegue al ambiente de producción al menos una vez al día
\cite{26_martin_fowler_ci}. Esta practica busca reducir de forma significativa los
problemas de integración, los cuales se incrementan en sistemas de gran tamaño
que tienen varios módulos que interactúan entre si.

Cada integración es verificada por medio de un constructor automatizado, el cual
hace uso de las pruebas del sistema\footnote{Pruebas de unidad, funcional e
  integración} para detectar errores de integración lo más rápido posible.
