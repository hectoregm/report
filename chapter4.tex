\chapter{Despliegue del sistema}

El despliegue de un sistema es una etapa de creciente importancia dado que los
sistemas actuales van requiriendo mas servicios y subsistemas de apoyo para
su funcionamiento. En la actualidad para garantizar un servicio estable para
un gran número de usuarios es necesario contar con repositorios de memoria,
equilibradores de carga, servidores de cola, etcétera. Esta creciente complejidad
incrementa los riesgos para lograr un despliegue a producción exitoso.

En este contexto se encuentra \texttt{PEAT} puesto que uno de sus requerimientos
principales es dar soporte a por lo menos mil usuarios concurrentes, en este
capítulo se describe la arquitectura del ambiente de producción, el proceso
de despliegue continuo para el despliegue del sistema \texttt{PEAT} en los
ambientes de prueba y producción, así como las optimizaciones realizadas
tanto al sistema como a la arquitectura de los ambientes para permitir un mayor
rendimiento posible.

\section{Arquitectura del ambiente de producción}
\label{sec:production}

Para el despliegue de los ambientes requeridos para \texttt{PEAT} se hizo uso de
\textit{Amazon Web Services} (AWS) en particular del servicio
\textit{Amazon Elastic Compute Cloud} (EC2) que permite la renta de servidores
para correr aplicaciones.

El uso de EC2 presentaba las siguientes ventajas:
\begin{itemize}
\item Permite un gran márgen de maniobra en la configuración de los servidores
  puesto que permite elegir entre una gran variedad de sistema operativos,
  capacidad de cómputo, etcétera.
\item Permite la creación y configuración automatizada de los servidores
  permitiendo que el sistema pueda escalar rápidamente según las necesidades
  de cómputo.
\item Una comunicación mas rápida y segura con el \texttt{backend} dado que el
  servidor C3 se ejecuta también bajo servidores EC2.
\end{itemize}

\rjcimage{1.0}{imagenes/Ambiente-Produccion.png}{Arquitectura ambiente de producción.}{produccion}

El ambiente de producción tiene los siguientes componentes (Ver Figura \ref{fig:produccion}):
\begin{itemize}
\item Equilibradores de carga: AWS tiene el servicio \textit{Elastic Load Balancing}
  (ELB) el cual permite obtener nivel altos de tolerancia al distribuir el trafico
  de forma automática entre varias instancias EC2 las cuales se encuentran en
  varias zonas de disponibilidad\footnote{Zonas que se encuentran en
    ubicaciones físicas distintas que cuentan con su propia infraestructura,
    independiente y físicamente distinta.} (Ver Figura \ref{fig:produccion} ELB).
\item Servidores EC2: son servidores con capacidades de cómputo que pueden ser
  activadas en cuestión de minutos. Para el sistema \texttt{PEAT} se usan instancias
  del tipo \texttt{M3}\footnote{Estas instancias proporcionan 2 CPU virtuales y
    2 GB de memoria.}. Para el servidor C3 se usan instancias de tipo \texttt{M4}
  \footnote{Estas instancias proporcionan 4 CPU virtuales y 16 GB de memoria.}
  (Ver Figura \ref{fig:produccion} M3 y M4).
\item Oracle BD: base de datos usada para almacenar la información usada
  por el servidor C3 es administrada por una base de datos Oracle Database 10g
  (Ver Figura \ref{fig:produccion} Oracle).
\item Apache Cassandra: un \textit{cluster} de nodos corriendo Apache Cassandra
  donde se realiza análisis en tiempo real de los datos contenidos en la
  base de datos de Oracle y también se usa como capa de repositorio de memoria del servidor C3
  (Ver Figura \ref{fig:produccion} Cassandra).
\end{itemize}

Se pude observar en la figura \ref{fig:produccion} dos zonas de disponibilidad
esto permite tener una mayor redundancia al no tener un único punto de error.

Se tienen dos equilibradores de carga el primero dirige las peticiones de los
usuarios a una instancia EC2 que corre instancias del sistema \texttt{PEAT},
el segundo equilibrador de carga es usado para distribuir la carga de peticiones
al servidor C3 por parte de todas las instancias \texttt{PEAT}.

\subsection{Phusion Passenger y Nginx}

Para el funcionamiento de \texttt{PEAT} es necesario tener tanto un servidor web
como un servidor de aplicación.
Para \texttt{PEAT} se eligió utilizar Phusion Passenger como servidor de aplicación
por las siguientes razones:

\begin{itemize}
\item Gran soporte para sistemas implementados con Ruby.
\item Crea y apaga instancias del sistema según la demanda, lo que permite tener
  buen rendimiento cuando se tiene mucha demanda pero conservando recursos cuando
  sea posible.
\item Reduce el uso de memoria para el sistema, por medio de técnicas como
  la precarga de aplicación y usando la gestión de memoria de copia en escritura
  (\textit{copy-on-write}, COW).
  % FIXME: Poner enlace a la sección donde se habla de esto en la optimización
\item Hace uso de todos los núcleos de CPU disponibles para el sistema.
\end{itemize}

Phusion Passenger permite la integración con los servidores web Apache o Nginx,
haciendo que estos se encarguen tanto del manejo de archivos estáticos
\footnote{Imágenes, hojas de estilo, código Javascript, etcétera} como de distribuir
la carga de peticiones entrantes.

Se eligió integrar Nginx como servidor web por dos razones (1) su configuración es
mucho mas simple que en Apache y (2) permite fácilmente agregar encabezados a las
peticiones entrantes sin afectar el rendimiento del sistema lo que permitió un
análisis más profundo de las peticiones procesadas por el sistema \texttt{PEAT}.

De esta forma tenemos que en cada instancia \texttt{M3} se tiene los
siguientes componentes (Ver Figura \ref{fig:passenger}):
\begin{itemize}
\item Nginx: proporciona los archivos estáticos del sistema y es también el
  equilibrador de carga de las instancias (Ver Figura \ref{fig:passenger} Nginx).
\item \textit{Preloader}: es un proceso que carga todo el sistema \texttt{PEAT},
  este proceso es copiado para obtener las instancias \texttt{PEAT} que manejan las
  peticiones (Ver Figura \ref{fig:passenger} Nginx Preloader).
\item Instancias \texttt{PEAT}: son instancias del sistema \texttt{PEAT} obtenidos
  por medio de \textit{Preloader} el numero mínimo de instancias se fijo en diez
  teniéndose la capacidad de crear hasta veinte instancias en forma automática.
\end{itemize}

\rjcimage{1.0}{imagenes/PEAT-EC2.png}{Arquitectura ambiente de producción.}{passenger}

Como se ve en la Figura \ref{fig:passenger} el servidor Nginx recibe una petición
por parte del equilibrador de carga para \texttt{PEAT}, dependiendo de la carga
en los instancias \texttt{PEAT} la petición se manda a una instancia \texttt{PEAT}
libre o es puesta en una cola de espera. Passenger detecta si la carga va aumentando
y determina automáticamente si es necesario crear nuevas instancias del sistema
\texttt{PEAT} por medio del proceso \texttt{Preloader}.

Aunque en la Figura \ref{fig:passenger} se tienen cinco instancias \texttt{PEAT}
ejecutándose en producción se tienen por lo menos diez instancias ejecutándose en
todo momento con la posibilidad de crearse hasta 20 instancias al tenerse más
trafico.

Mediante varias pruebas de carga se determinó que el sistema debía manejar picos
de demanda de hasta 30,000 peticiones por minuto para dar soporte al menos
1,000 usuarios concurrentes. Cada instancia PEAT es capaz de procesar hasta
300 peticiones por minuto por lo que cada servidor \texttt{M3} procesa entre
3,000 a 6,000 peticiones por minuto. Para el ambiente de producción se determinó
usar seis servidores \texttt{M3} por lo que la capacidad base es de 18,000
peticiones por minuto pero con capacidad de crecer con la demanda hasta
36,000 peticiones por minuto.

Los ambientes de prueba tenían la misma estructura solo cambiando el número
de servidores según las necesidades y pruebas a realizar.

\subsection{Inicialización e invalidación del repositorio de memoria}

Uno de los principales cuellos de botella encontrados durante el desarrollo
del sistema \texttt{PEAT} fue la gran cantidad de peticiones realizadas al
servicio web. Para reducir el numero de peticiones se determinó que un gran
número de éstas se podían evitar teniendo un repositorio de memoria de los modelos
que solo pueden ser modificados por el administrador de sistema \texttt{PEAT}
y que además son actualizados cada seis meses. 

Los modelos que se pueden considerar que contienen información estática son:
\begin{itemize}
\item \textit{BuildingQuestion}: este modelo representa una pregunta sobre
  el perfil de un edifico, se tienen 45 preguntas por tipo de edificio, en
  total se tienen 500 preguntas.
\item \textit{Recommendation}: este modelo representa una recomendación
  con todos su información de costo, beneficios, etcétera. Se inicio con un
  total de cien recomendaciones.
\item \textit{ValidAnswer}: las respuestas válidas para una pregunta, dado
  que la mayoría de las preguntas tiene cuatro posibles respuestas se tiene
  un estimado de dos mil respuestas en el sistema.
\end{itemize}

Se determinó que la mejor solución fue que al iniciar el sistema \texttt{PEAT}
se precargara el repositorio de memoria con la información de los modelos descritos
anteriormente por medio del servicio web C3. De esta forma se redujo el numero de
peticiones de forma substancial pero teniendo el inconveniente de un incremento
en el tiempo de inicialización de una instancia del sistema \texttt{PEAT}.

Para minimizar el impacto de un mayor tiempo de inicialización se hizo uso de las
características de precarga y de gestión de memoria de copia en escritura
(\textit{copy-on-write}, COW) proporcionados por Phusion Passenger. Al iniciar
Phusion Passenger se inicia el proceso \texttt{Preloader} que realiza la
inicialización del sistema \texttt{PEAT} y la precarga de modelos en el
repositorio de memoria.

En la Figura \ref{fig:cache} se puede ver que por medio de \textit{Preloader}
se comparte en memoria el código del marco de trabajo Rails, el código del sistema
\texttt{PEAT} y el repositorio de memoria de los modelos descritos anteriormente.
Cada instancia solo usa memoria para sus propios objetos utilizados para manejar las
peticiones entrantes.

De esta manera al iniciar una nueva instancia del sistema se realiza una copia del
proceso \textit{Preloader} el cual contiene los modelos precargados en el
repositorio de memoria por lo que la inicialización de nuevas instancias es casi
instantánea.

\rjcimage{0.8}{imagenes/Passenger-Cache.png}{Diagrama de la memoria compartida entre
  \textit{Preloader} y las instancias \texttt{PEAT}.}{cache}

Para la invalidación del repositorio de memoria de los modelos se tenia una acción
para el administrador del sistema que permite limpiar el repositorio de memoria y
actualizar los modelos por medio del servicio web C3.

% Https://aws.amazon.com/es/ec2/instance-types/ EC2
% https://aws.amazon.com/es/ec2/faqs/


\section{Despliegue continuo}

El despliegue continuo (\textit{Continuous Delivery}, CD) es una buena práctica de
software en la cual se implementa software de tal forma de que éste pueda ser
desplegado a producción en cualquier momento\cite{27_martin_fowler_cd}. Su objetivo
es crear, probar y liberar software más rápido y con mayor frecuencia.

\vspace{2.5mm}

Se considera que un sistema implementa el despliegue continuo cuando
\cite{27_martin_fowler_cd}:
\begin{itemize}
\item El sistema puede ser desplegado a producción en cualquier momento del
  ciclo de desarrollo.
\item Se tiene una rápida retroalimentación sobre la capacidad del sistema
  para ser desplegado en producción al realizarse cambios en el sistema.
\item Se puede desplegar de forma sencilla cualquier versión del software para
  cualquier entorno.
\end{itemize}

Los principales beneficios de esta práctica son:
\begin{itemize}
\item Reducción del riesgo de despliegue: dado que se realizan despliegues
  a producción de forma recurrente los cambios contenidos en cada despliegue
  son menores por lo que hay menores posibilidades de errores y si un error
  se presenta en mas fácil de localizar y arreglar.
\item Retroalimentación del usuario: uno de los mayores riesgos en la ingeniería
  del software es el desarrollar un sistema que no es útil para el cliente o usuario.
  Por lo tanto obtener retroalimentación sobre el desarrollo del sistema de forma
  rápida y frecuente permite evaluar que tan útil es el sistema para el usuario.
\end{itemize}

En \texttt{PEAT} los riesgos asociados al despliegue de producción exitoso
eran mayores a lo habitual, dada la complejidad del sistema y la cantidad
de requerimientos. Además PG\&E requería el despliegue del sistema en ambientes
de pruebas previo a un despliegue del sistema en el ambiente de producción.
La implementación de un proceso de despliegue continuo para \texttt{PEAT} fue de
vital importancia para cumplir con las necesidades del cliente.

\vspace{2.5mm}

Para PG\&E era necesario que se tuvieran los siguientes ambientes:
\begin{itemize}
\item \textit{peattest}: este ambiente es usado para hacer pruebas con usuarios
  seleccionados por PG\&E para obtener retroalimentación sobre el sistema.
\item \textit{peatqa}: este ambiente es usado por PG\&E para realizar control de
  calidad y de rendimiento.
\item \textit{peatprod}: este ambiente es el ambiente final de producción.
\end{itemize}

Para uso interno en C3 Energy se tenía además el ambiente \textit{peatstage},
el cual era usado para realizar pruebas de rendimiento y de control de calidad.

En total se contaba con cuatro ambientes, los cuales debían ser lo mas parecidos
en su arquitectura y configuración a la del ambiente de producción para que así las
pruebas de calidad y de rendimiento regresaran resultados cercanos al comportamiento
del sistema en el ambiente de producción.
Dados estos requerimientos la implementación del proceso de despliegue continuo para
el sistema \texttt{PEAT} era la mejor forma para manejar el despliegue del sistema
en esta diversidad de ambientes.

\subsection{Proceso de despliegue}

En el despliegue continuo se tiene al proceso de despliegue (\textit{deployment
  pipeline}) como concepto central, el cual, en esencia es la implementación
automatizada de los procesos de configuración, construcción, prueba y despliegue de
un sistema \cite{28_humble_farley_2011}.

\vspace{2.5mm}

Este proceso tiene las siguientes ventajas:
\begin{itemize}
\item Visibilidad: todas las etapas del sistema de despliegue son visibles
  para todos los miembros del equipo.
\item Retroalimentación: los miembros del equipo obtiene información sobre los
  problemas en el momento en que ocurren de modo que son capaces de dar solución
  a éstos rápidamente.
\item Despliegue: por medio de un proceso totalmente automatizado se puede
  desplegar y liberar cualquier versión del sistema en cualquier ambiente y en
  cualquier momento.
\end{itemize}

El proceso de despliegue se conforma de una serie de etapas de validación por las
que el sistema debe pasar para llegar al ambiente de producción. Para el sistema
\texttt{PEAT} las etapas de validación son las siguientes (Ver Figura
\ref{fig:pipeline}):
\begin{itemize}
\item Control de versiones: el equipo de desarrollo ingresa los cambios
  realizados al sistema por medio del control de versiones Git, esto inicia
  el proceso de despliegue.
\item Pruebas de unidad: al detectarse un cambio en el repositorio se ejecutan las
  pruebas de unidad del sistema.

  Se usa la biblioteca \texttt{RSpec} para implementar las pruebas de unidad
  para el código en Ruby, mientras que para el código en Javascript se hace uso de
  la biblioteca \texttt{Jasmine} para las pruebas de unidad.

  El equipo de desarrollo recibe retroalimentación del resultado de las pruebas,
  si el resultado es positivo se pasa a la siguiente etapa.
  En la primera secuencia en la Figura \ref{fig:pipeline} se tiene el caso
  cuando las pruebas de unidad son negativas y se detiene el proceso de despliegue.
\item Pruebas de aceptación: se hace uso de la biblioteca \textit{Cucumber}
  para implementar las pruebas de aceptación, de la misma forma que la etapa de
  pruebas de unidad si se obtiene un resultado positivo se continúa a la siguiente
  etapa en caso contrario se detiene el proceso de despliegue.
\item Ambientes de prueba: se hace el despliegue del sistema en los ambientes de
  prueba, es decir, a los ambientes \textit{peatstage}, \textit{peattest} y
  \textit{peatqa}. En esta etapa se realizan pruebas de rendimiento y de calidad
  de forma manual y automatizada.
\item Producción: se hace un despliegue a producción cuando la versión en los
  sistemas de prueba han sido revisado de forma satisfactoria.
\end{itemize}

\rjcimage{1.1}{imagenes/Deployment-Pipeline.png}{Proceso de despliegue para
  \texttt{PEAT}.}{pipeline}

Las primeras etapas del proceso se desarrollan en forma completamente automatizada
hasta llegar a la etapa de despliegue a los ambientes de pruebas, en esta etapa
al principio se realizaban pruebas de calidad y rendimiento de forma manual, estas
actividades se fueron automatizando una por una  haciendo uso de herramientas como
\texttt{SauceLabs} y \texttt{CloudTest}. La ejecución para el despliegue al ambiente
de producción se realiza de forma manual.

\subsubsection{Herramientas}

El proceso de despliegue se hace uso de un gran número de herramientas, las más
importantes son:

\begin{itemize}
\item \texttt{AWS}: \textit{Amazon Web Services} (AWS), es un conjunto de servicios
  de cómputo en la nube que permite la implementación de sistemas escalables.
\item \texttt{Jenkins}: es un servidor de despliegue continuo, que permite
  definir las etapas de un proceso de despliegue y las acciones que se realizan
  según el resultado de la etapa.
\item \texttt{Capistrano}: es una herramienta que facilita la creación de tareas
  para el despliegue de sistemas en servidores remotos.
\item \texttt{CloudTest}: es una herramienta para la ejecución de pruebas de
  carga y rendimiento para aplicaciones móviles y web.
\item \texttt{SauceLabs}: es una herramienta que permite ejecutar las pruebas de un
  sistema en mas de 500 combinaciones de navegadores, sistema operativos
  y dispositivos.
\end{itemize}

\subsection{Configuración y despliegue automatizado de ambientes}

En la Sección \ref{sec:production} se hablo sobre la arquitectura del ambiente
de producción para \texttt{PEAT} en donde se indico el uso de servidores EC2
proporcionados por \textit{Amazon Web Services} (AWS).

\subsubsection{Capistrano}

Para configurar los servidores EC2 para \texttt{PEAT} se usa \textit{Capistrano}
que es una herramienta para definir y ejecutar tareas en
múltiples servidores remotos por medio de SSH\footnote{Es un protocolo que permite
  acceder a servidores remotos de forma segura a través de una red
  \cite{29_ssh_protocol}.}. Esta herramienta define un DSL basado en Ruby para la
definición de tareas de despliegue y mantenimiento.

\textit{Capistrano} contiene tareas predefinidas para el despliegue de sistemas
basados en el marco de trabajo Rails, sin embargo para el sistema \texttt{PEAT} se
implementaron tareas adicionales para realizar un despliegue exitoso.

\subsubsection{Configuración de ambientes}

\lstinputlisting[language=Ruby]{code/deploy-vars.rb}

El código anterior muestra la configuración base de variables para el
despliegue del sistema \texttt{PEAT} usando el DSL definido por \texttt{Capistrano}.
En las líneas 1-9 se definen las variables de configuración para el sistema
\texttt{PEAT} y el servidor Nginx, los valores por defecto están definidos
para un ambiente en producción, lo cual facilita conocer las diferencias
entre el ambiente de producción y los ambientes de pruebas.

En la línea 11 se definen los ambientes soportados, teniendo los cuatro
ambientes descritos anteriormente, cabe señalar que cada ambiente tiene
su propio archivo de configuración en donde se indican los cambios en variables
y/o tareas con respecto a la configuración base.

En las líneas 14-16 se indica el sistema de versiones, la dirección del repositorio
y la rama\footnote{Es una versión etiquetada del código fuente en el control de
  versiones Git.} a usar para obtener el código fuente del sistema.
En \texttt{PEAT} el sistema de versiones es Git, donde la rama por defecto
es \textquote{release/v1.0} que es la rama estable del sistema que
se usa para producción, por medio de la variable de ambiente \texttt{BRANCH} es
posible hacer el despliegue de cualquier versión del sistema.

En las líneas 18-24 se tiene el uso de las declaraciones \texttt{after}
y \texttt{before} que permiten el establecer que una tarea sea realizada
antes o después de una segunda tarea. Cabe señalar que las tareas
\texttt{deploy:setup} y \texttt{deploy:restart} son tareas definidas
por \texttt{Capistrano}, la primera es una tarea para la configuración inicial
de un ambiente\footnote{Esta tarea es solo ejecutada una única vez para preparar
  un nuevo ambiente.} y la segunda tarea es realizada al reiniciar el sistema.

\lstinputlisting[language=Ruby]{code/deploy-tasks.rb}

En el código anterior se tiene las tareas mas importantes para el
despliegue del sistema. En las líneas 3-5 se define la tarea \texttt{upstart}
la cual por medio del comando \texttt{foreman} genera la configuración para agregar
al sistema \texttt{PEAT} como un servicio al sistema Upstart\footnote{Es un demonio
  que maneja el inicio de tareas y servicios durante el arranque del servidor y la
  supervisión de éstos mientras el servidor esta funcionando.} usado en el sistema
operativo Ubuntu 12.04 LTS que es el sistema operativo seleccionado para los
servidores EC2.

En las líneas 6-13 se definen las tareas para iniciar y parar el sistema
\texttt{PEAT}, se hace uso de los comandos \texttt{start} y \texttt{stop}, así como
de la configuración generada para el sistema Upstart en la tarea anterior para
iniciar o parar fácilmente el sistema.

En las líneas 15-19 se define la tarea \texttt{peat\_setup} la cual genera
los archivos de configuración para \texttt{PEAT} (\texttt{peat.yml}) y
Nginx (\texttt{nginx.conf}) y posteriormente los transfiere a los servidores
del ambiente, esta tarea se ejecuta solamente cuando se inicia un nuevo
ambiente.

En las líneas 21-27 se define la tarea \texttt{config\_sync} la cual
realiza las tares de generar los archivos de configuración, transferirlos y
ejecuta el reinicio del sistema para que éste cargue los cambios de configuración.

En las líneas 30-35 se define la tarea \texttt{config} la cual genera
los archivos de configuración para \texttt{PEAT} (\texttt{peat.yml}) y
Nginx (\texttt{nginx.conf}) por medio de \texttt{ConfigCreator}.

\lstinputlisting[language=Ruby]{code/deploy-peatstage.rb}

Como se mencionó anteriormente cada ambiente tiene su propio archivo de
configuración, el código anterior es la configuración para el ambiente
\texttt{peatstage}.

En la línea 1-2 se define el servidor asociado al ambiente. En las líneas
3-10 se redefinen varias variables de configuración para el ambiente siendo uno
de los principales cambios que la rama por defecto es \textquote{master} la
cual es la rama principal de desarrollo en el sistema \texttt{PEAT}.
En las líneas 29-46 se redefinen las tareas \texttt{peat\_setup},
\texttt{config\_sync} y \texttt{config} para que tomen en cuenta la configuración
actualizada por medio del método \texttt{override\_defaults}.

\vspace{2.5mm}

\subsubsection{Despliegue automatizado de ambientes}

La interacción con \texttt{Capistrano} es por medio del comando \texttt{cap}
en la línea de comandos, así para hacer un despliegue al ambiente \texttt{peatprod}
se ejecutaría el siguiente comando.

\begin{verbatim}
cap peatprod deploy
\end{verbatim}

El comando \texttt{cap} toma dos argumentos el primero es algunos de los
ambientes definidos en la variable \texttt{stages} y el segundo es el nombre
de la tarea a realizar.

Como se puede ver por medio de \texttt{Capistrano} se puede realizar tanto
la configuración inicial de los ambientes y el despliegue del sistema \texttt{PEAT}.

El uso de \texttt{Capistrano} para el despliegue tiene las siguientes ventajas:
\begin{itemize}
\item Rápidamente se puede agregar un nuevo ambiente agregando su nombre a la
  variable \texttt{stages} y agregando su archivo de configuración con los
  cambios deseados para este nuevo ambiente.
\item La configuración definida con \texttt{Capistrano} es parte del código fuente
  del sistema y se encuentra en el sistema de versiones por lo que se tiene historia
  de los cambios realizados en la configuración de los ambientes.
\item El sistema es auto contenido, es decir, el código para realizar el despliegue
  del sistema es parte del código fuente del sistema.
\end{itemize}

En el servidor \textit{Jenkins} se implemento una tarea que al tenerse un resultado
positivo de las pruebas del sistema se hace uso del comando \texttt{cap}
para efectuar el despliegue del sistema en los ambientes de prueba.

\subsection{Pruebas de unidad y aceptación}

En el servidor \textit{Jenkins} se definen las etapas del proceso de despliegue,
siendo el inicio del proceso de despliegue cuando se detecta un cambio en el
repositorio del sistema. Al detectarse este cambio se ejecuta la primera etapa
que implica ejecutar las pruebas de unidad.

\subsubsection{Pruebas de unidad}

En la sección 2.3.2 se habló sobre la práctica del desarrollo guiado por
comportamiento (\textit{Behavior Driven Development}, BDD) la cual involucra
escribir las pruebas antes de escribir el código fuente y la refactorización
continúa en el código fuente.

En el desarrollo guiado por comportamiento las pruebas se enfocan en describir el
comportamiento y la interacción de los módulos del sistema. Para el sistema
\texttt{PEAT} se hace uso del marco de trabajo \textit{RSpec} la cual define un
Lenguaje de Dominio Específico (LDE) para implementar las pruebas unitarias.

RSpec da acceso al comando \texttt{rspec} que permite ejecutar y verificar
que el sistema pasa todas las pruebas unitarias del sistema. En la primera
etapa el servidor \textit{Jenkins} obtiene el código fuente del sistema y ejecuta el
comando \texttt{rspec} el cual ejecuta todas las pruebas unitarias del sistema.

\lstinputlisting[language=Ruby]{code/rspec-questions-controller.rb}

El código anterior se tiene un ejemplo de una prueba unitaria para el controlador
\texttt{QuestionsController} (línea 3) que maneja las acciones en relación
a las preguntas asociadas al perfil de un edificio. En las líneas 4-13 se
inicializan los modelos necesarios para llevar a cabo las pruebas de este
controlador.
En las líneas 16-20 se tiene por medio de la declaración \texttt{it} la
prueba unitaria sobre la acción \texttt{next} la cual debe regresar
la siguiente pregunta a contestar del perfil del edificio. En la línea
17 se define la expectativa de que el controlador llama al método \texttt{next
  \_question} del modelo \texttt{PEATEngine} para obtener la siguiente pregunta.

Por medio de \texttt{RSpec} se tienen pruebas de los controladores y vistas
principales del sistema \texttt{PEAT}. Los modelos no son probados directamente
dado que los modelos hacen uso de la clase \texttt{Bezel::Base} de la biblioteca
\texttt{Bezel} la cual también tiene pruebas de unidad por medio de \texttt{RSpec}.

\subsubsection{Pruebas de aceptación}

Para el sistema \texttt{PEAT} se hace uso del marco de trabajo \textit{Cucumber}
el cual permite especificar y ejecutar pruebas de aceptación siguiendo el proceso
del desarrollo guiado por comportamiento.

\begin{lstlisting}
  Característica:
  Permitir a los usuarios el añadir una recomendación a su plan de ahorro
  El usuario tiene una cuenta, una dirección de servicio y un edificio

  Antecedentes:
  Dado que ingreso como el usuario @juan por vez primera
  Y que tengo una cuenta, una dirección de servicio y un edificio

  @javascript
  Escenario: Despliega la lista de recomendaciones
  Cuando visito la pagina de recomendaciones
  Entonces debo ver un botón de añadir al plan en cada recomendación

  @javascript
  Escenario: Cambiar al estado "Completado" a una recomendación
  Dado que visito la pagina de recomendaciones
  Y selecciono el primer botón añadir al plan
  Debo ver un menú de selección
  Cuando selecciono la opción "Completado" en el menú desplegable
  Entonces el botón de añadir al plan debe indicar "Completado"

  @javascript
  Escenario: Añadir una recomendación al plan de ahorro
  Dado que visito la pagina de recomendaciones
  Cuando selecciono "Añadir al plan" en la primera recomendación
  Y selecciono "Guardar" en la ventana.
  Entonces el botón de añadir al plan debe indicar "Agregado"
  Y no debo ver el menú despegable
\end{lstlisting}

El código anterior es una prueba de aceptación sobre el requerimiento de que los
usuarios puedan añadir una recomendación a su plan de ahorro.
En las líneas 1-3 se documenta el objetivo de la característica que se esta
probando, se prueba una sola característica por archivo pero se pueden tener
varios escenarios por característica. En las líneas 5-7 se definen
los antecedentes, es decir, el contexto inicial que comparten todos los
escenarios. Cada escenario empieza con la declaración \texttt{Escenario:}
con una breve descripción del escenario, las líneas posteriores indican
tanto el contexto como la expectativa del escenario. Los escenarios
pueden ser etiquetados en este ejemplo tienen la etiqueta \texttt{@javascript}
lo que indica que estos escenarios hacen uso de un navegador con soporte
a Javascript para ejecutar la prueba.

Cabe señalar el parecido entre el caso de uso \textit{Administrar recomendaciones}
y la prueba de aceptación, esto es una de las ventajas del uso de \textit{Gherkin}
que es el lenguaje DLE definido por \textit{Cucumber} para facilitar la
implementación de pruebas de aceptación a partir de casos de uso.

\textit{Cucumber} da acceso al comando \texttt{cucumber} que permite ejecutar y
verificar que el sistema pasa todas las pruebas de aceptación del sistema.
El servidor \textit{Jenkins} ejecuta el comando \texttt{cucumber} si el resultado
de las pruebas unitarias es exitoso. Si el resultado de las pruebas de aceptación es
positivo entonces se hace un despliegue del sistema a los ambientes de prueba.

\subsection{Ambientes de prueba}

Después de que tanto las pruebas de unidad como las de aceptación dan resultados
positivos el servidor \textit{Jenkins} realiza el despliegue del sistema al ambiente
\textit{peatstage}.

En el ambiente \textit{peatstage} se realizan pruebas de rendimiento por medio
de la herramienta CloudTest, la cual permite realizar pruebas de carga y
rendimiento para sistemas web, además esta herramienta tiene integración con
\textit{Jenkins} por lo que se agregó como una etapa mas en el proceso de despliegue.

Usando CloudTest se tenían tres pruebas de carga, teniendo en cada prueba 500, 1000
y 2000 usuarios concurrentes respectivamente. Si estas pruebas eran satisfactorias
entonces el servidor \textit{Jenkins} hace un despliegue del sistema en los
ambientes \textit{peattest} y \textit{peatqa}.

\subsection{Ciclo de desarrollo}

En \texttt{PEAT} se tiene que el modelo de desarrollo es una combinación tanto del
modelo en cascada y del modelo ágil, teniendo en un inicio la creación de documentos
sobre los requerimientos de los componentes del sistema y su arquitectura general.
A diferencia de un modelo típico de cascada en cuanto se tenían la suficiente
información sobre un componente se iniciaba su implementación haciendo de
metodologías mas asociadas al modelo ágil como las iteraciones y el despliegue
continuo, esto permitía obtener retroalimentación de usuarios finales.

En el sistema \texttt{PEAT} se tenia el siguiente ciclo de desarrollo de
software:

\begin{enumerate}
\item El programador crea una nueva rama de trabajo en el repositorio \texttt{Git}
  para la implementación de una nueva característica del sistema.
\item Conforme se van acumulando los cambios del código en la rama de trabajo
  el servidor \textit{Jenkins} ejecuta solamente las pruebas de unidad y aceptación,
  permitiendo una rápida retroalimentación del estado del sistema al programador.
\item Cuando la implementación se considera terminada y las pruebas son positivas,
  se realiza una revisión del código de la rama de trabajo por el resto del
  equipo.
\item Después de la revisión y dado el visto bueno del equipo se integra el
  código de la rama de trabajo a la rama \textquote{master}, lo cual provoca el
  inicio de todo el proceso de despliegue.
\item El servidor ejecuta las pruebas de unidad y aceptación del sistema en
  caso positivo realiza un despliegue al ambiente \textit{peatstage}, en caso
  contrario el equipo es informado de cuales pruebas fallaron.
\item El servidor ejecuta las pruebas de carga usando el ambiente \textit{peatstage}
  si el resultado es positivo se hace un despliegue a los ambientes \textit{peattest}
  y \textit{peatqa}, en caso contrario se informa al equipo sobre que partes del
  sistema no tiene el rendimiento esperado.
\item En coordinación con PG\&E se despliega la versión en uso en \textit{peattest}
  al ambiente en producción.
\end{enumerate}

Cualquier falla se convierte en una tarea prioritaria para el equipo de desarrollo
para evitar que el lapso de tiempo en que el sistema no puede ser desplegado sea
el mínimo posible.

Haciendo uso del despliegue continuo el equipo entero tiene conocimiento sobre el
estado del sistema aunque el sistema \texttt{PEAT} se encuentre en un estado
inconsistente por fallas, ya sea en las pruebas unitarias, aceptación o de carga,
se tiene información sobre la naturaleza y origen de las fallas. Esta información
permite al equipo arreglar el sistema de una forma eficiente.
