\chapter{Despliegue a producción}

La etapa para desplegar una aplicación es una etapa de creciente importancia ya
que por un lado hace algunos años las aplicaciones web se desarrollaban bajo el
modelo de tres capas: la base de datos, el servidor de la aplicación y el servidor
web; actualmente se tienen mas servicios que deben estar en línea sobre todo para
garantizar un servicio concurrente; caches, equilibradores de carga, servidores
de cola, etcétera. Por lo que automatizar el despliegue de una aplicación en sus
diferentes contextos, desarrollo, producción y pruebas, es de vital importancia
para el desarrollo de software en tiempo y en forma.

\section{Despliegue continuo}

El despliegue continuo (\textit{Continuous Delivery}, CD) es una practica de software
en la cual se implementa software de tal forma de que éste pueda ser desplegado a
producción en cualquier momento\cite{27_martin_fowler_cd}. Su objetivo es crear,
probar y liberar software más rápido y con mayor frecuencia.

Se considera que un sistema implementa CD de forma correcta cuando
\cite{27_martin_fowler_cd}:
\begin{itemize}
\item El sistema puede ser desplegado a producción en cualquier momento del
  ciclo de desarrollo.
\item Se tiene una rápida retroalimentación sobre la capacidad del sistema
  para ser desplegado en producción al realizarse cambios en el sistema.
\item Se puede desplegar en forma fácil cualquier versión del software para
  cualquier entorno.
\end{itemize}

Los principales beneficios de esta practica son:
\begin{itemize}
\item Reducción del riesgo de despliegue: dado que se realizan despliegues
  a producción de forma recurrente los cambios contenidos en cada despliegue
  son menores por lo que hay menores posibilidades de errores y si un error
  se presenta en mas fácil de arreglar.
\item Retroalimentación del usuario: uno de los mayores riesgos en la ingeniería
  del software es el desarrollar un sistema que no es útil para el cliente o usuario.
  Por lo que obtener retroalimentación sobre el desarrollo del sistema lo mas
  rápido posible y de forma frecuente permite descubrir que tan valioso es el sistema
  al usuario.
\end{itemize}

El despliegue continuo es una practica de software en la cual los integrantes
de un proyecto integran su trabajo (código, diseño, etcétera) de forma frecuente
realizando un despliegue al ambiente de producción al menos una vez al día
\cite{26_martin_fowler_ci}. Esta practica busca reducir de forma significativa los
problemas de integración, los cuales se incrementan en sistemas de gran tamaño
que tienen varios módulos que interactúan entre si.

Cada integración es verificada por medio de un constructor automatizado, el cual
hace uso de las pruebas del sistema\footnote{Pruebas de unidad, funcional e
  integración} para detectar errores de integración lo más rápido posible.
