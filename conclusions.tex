\chapter{Conclusiones}

En este documento se expuso el diseño e implementación del sistema \texttt{PEAT} que
permite a sus usuarios el identificar y monitorear su consumo energético. En C3
Energy se tienen varios sistemas en producción haciendo uso de lenguajes de
programación como Java y Javascript y herramientas como Rhino y Ext JS.

En la implementación de \texttt{PEAT} se presentaron nuevas retos como la necesidad
de una interfaz simple y de rápida respuesta a las acciones del usuario y
dar servicio a un numero considerable de usuarios.
Estos requerimientos no se podían solucionar de forma óptima haciendo uso de los
lenguajes y herramientas usadas hasta ese momento. Es por eso que al iniciar
la implementación del sistema \texttt{PEAT} se eligió tener un ambiente de
desarrollo multi-lenguaje en el cual se hace uso de un lenguaje de programación
o herramienta en los contextos en que éstos dan la mayor ventaja. Un ambiente
multi-lenguaje trae consigo sus propios retos como la integración con sistemas ya
existentes y el despliegue exitoso de los sistemas a producción.

Haciendo uso de las capacidades de metaprogramación del lenguaje de programación
Ruby y de la creación de un lenguaje de dominio específico en la biblioteca
\texttt{Bezel} se facilitó la implementación de los modelos necesarios para
el funcionamiento del sistema de forma eficiente y con la flexibilidad necesaria
para lograr la integración del sistema \texttt{PEAT} con los subsistemas ya
existentes, permitiendo al programador definir modelos y sus asociaciones con
pocas lineas de código.

El uso de lenguajes de dominio específico fue el aspecto mas sobresaliente
para el desarrollo exitoso del sistema \texttt{PEAT} por su ayuda en la
integración con los subsistemas ya existentes y en la automatización de la
configuración y despliegue del sistema.

El uso del despliegue continuo y de pruebas con usuarios finales fueron vitales para
que el sistema cumpliera con su objetivo de dar la mayor utilidad posible
a los usuarios PyMES sobre su consumo energético. La retroalimentación continúa
obtenida por los ambientes de prueba permitió detectar deficiencias tanto en el
\texttt{backend} como en la interfaz del usuario, siendo la principal deficiencia
detectada la dificultad de ingreso de la información inicial para crear un perfil
de un edificio. Gracias a esta retroalimentación se pudo obtener una interfaz
mas intuitiva para el usuario sin afectar el tiempo de entrega.

También por medio del despliegue continuo se logró tener un lanzamiento a la etapa
de producción sin problemas, dado el que con anterioridad se realizaron varios
despliegues en ambientes de prueba los cuales eran idénticos al ambiente de
producción.

\pagebreak

El sistema \texttt{PEAT} fue recibido muy bien por los usuarios y el cliente PG\&E
ya que desde hace cuatro años se encuentra en funcionamiento dando servicio a cerca
de 250,000 usuarios PyMES a la fecha\cite{30_pge_annual_report}. Su éxito provocó que
el sistema fuera también adoptado por otras compañías proveedoras de electricidad y
gas como San Diego Gas \& Electric\cite{32_reuters_c3} y Southern California Edison,
que prácticamente abarca el estado de California en los Estados Unidos de
América\cite{31_energy_map}.
