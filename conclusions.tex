\chapter{Conclusiones}

En este documento se expuso el diseño e implementación del sistema \texttt{PEAT} que
permite a sus usuarios el identificar y monitorear su consumo energético. En C3
Energy se tienen varios sistemas en producción haciendo uso de lenguajes de
programación como Java y Javascript y herramientas como Rhino y Ext JS.

En la implementación de \texttt{PEAT} se presentaron nuevas retos como la necesidad
de una interfaz simple y de rápida respuesta a las acciones del usuario y
dar servicio a un numero considerable de usuarios.
Estos requerimientos no se podían solucionar de forma óptima haciendo uso de los
lenguajes y herramientas usadas hasta ese momento. Es por eso que al iniciar
la implementación del sistema \texttt{PEAT} se eligió tener un ambiente de
desarrollo multi-lenguaje en el cual se hace uso de un lenguaje de programación
o herramienta en los contextos en que éstos dan la mayor ventaja. Un ambiente
multi-lenguaje trae consigo sus propios retos como la integración con sistemas ya
existentes y el despliegue exitoso de los sistemas a producción.

En el Capítulo 3 se describió la implementación de la interfaz de usuario y el flujo
de navegación y de interacción del usuario con el sistema permitiendo a éste obtener
la mayor información para tomar decisiones sobre su edificio con la menor información
disponible. En este capítulo también se describió el uso de las capacidades de
metaprogramación de Ruby para facilitar la implementación y uso de la biblioteca
\texttt{Bezel} que permite la integración con los subsistemas ya existentes al
permitir definir en pocas lineas de código los modelos y sus asociaciones.

Otro punto de creciente importancia es el despliegue del sistema, ya que por
un lado hace algunos años las aplicaciones web se desarrollaban bajo el modelo
de tres capas la base de datos, el servidor de la aplicación y el servidor web;
actualmente se tienen mas servicios que deben estar en línea, para garantizar un
servicio concurrente; repositorios de memoria, equilibradores de carga, servidores
de cola, etcétera. Por lo que automatizar el despliegue de una aplicación en sus
diferentes ambientes, como el de desarrollo, pruebas y producción, es de vital
importancia para conocer siempre el estado del sistema y así lograr que el proyecto
sea desarrollado en tiempo y en forma.

En el Capítulo 4 se describió tanto la arquitectura usada en los ambientes de
prueba y producción así como la implementación de un despliegue continuo con el
cual se logró tener un lanzamiento a la etapa de producción sin problemas, dado el
que con anterioridad se realizaron varios despliegues en ambientes de prueba
los cuales eran idénticos al ambiente de producción.

El uso del despliegue continuo y de pruebas con usuarios finales fueron vitales
que el sistema cumpliera con su objetivo de dar la mayor utilidad posible
a los usuarios PyMES sobre su consumo energético. La retroalimentación continua
obtenida por los ambientes de prueba permitió detectar deficiencias tanto en el
\texttt{backend} como en la interfaz del usuario, las cuales pudieron ser arregladas
en tiempo y forma para lograr tener una entrega exitosa.

El sistema \texttt{PEAT} fue recibido muy bien por los usuarios y el cliente PG\&E
ya desde que hace cuatro años se encuentra en funcionamiento dando servicio a cerca
de 250,000 usuarios PyMES a la fecha\cite{30_pge_annual_report}. Su éxito provocó que
el sistema fuera también adoptado por otras compañías proveedoras de electricidad y
gas como San Diego Gas \& Electric\cite{32_reuters_c3} y Southern California Edison,
que prácticamente abarca el estado de California en los Estados Unidos de
América\cite{31_energy_map}.
