\chapter{Conclusiones}

En este documento se expuso el diseño e implementación del sistema \texttt{PEAT} que
permite a sus usuarios el identificar y monitorear su consumo energético. En la
implementación de \texttt{PEAT} se tiene el contexto en el cual se tiene varios
sistemas en producción con un lenguaje de programación y herramientas ya
probadas.

Nuevas necesidades se presentan las cuales no siempre se pueden solucionar en
forma óptima haciendo uso del lenguaje de programación y/o herramientas usadas
anteriormente. Este contexto es el ocurrido al inicio del sistema \texttt{PEAT}
donde se eligió tener un ambiente de desarrolló políglota en el cual se hace uso
de un lenguaje de programación o herramienta en los contextos en que estos dan la
mayor ventaja. Un ambiente políglota trae consigo sus propios retos como la
integración con sistemas ya existentes y el despliegue exitoso de los sistemas a
producción. En el capitulo 3 se describió el uso de las capacidades de
metaprogramación de Ruby para facilitar la implementación de la biblioteca
\texttt{Bezel} que permite la integración con los subsistemas ya existentes.

Otro punto de creciente importancia es el despliegue del sistema, ya que por
un lado hace algunos años las aplicaciones web se desarrollaban bajo el modelo
de tres capas la base de datos, el servidor de la aplicación y el servidor web;
actualmente se tienen mas servicios que deben estar en linea sobre todo para
garantizar un servicio concurrente:; caches, equilibradores de carga, servidores
de cola, etcétera. Por lo que automatizar el despliegue de una aplicación en sus
diferentes ambientes, desarrollo, pruebas y producción, es de vital importancia
para siempre conocer el estado del sistema y así lograr que el proyecto sea
desarrollado en tiempo y en forma. En el capitulo 4 se vio tanto la arquitectura
usada para el despliegue del sistema como la implementación de un proceso
de despliegue continuo con el cual se logro tener un despliegue a producción sin
problemas dado que desde meses antes del despliegue final a producción se
realizaron despliegues en ambientes de prueba que eran idénticos al ambiente de
producción.

El uso del despliegue continuo y de pruebas con usuarios fueron vitales
que el sistema cumpliera con su objetivo de dar la mayor utilidad posible
a los usuarios PyMES sobre su consumo energético. La retroalimentación continua
obtenida por los ambientes de prueba permitió detectar deficiencias tanto en el
\texttt{backend} como en la interfaz del usuario, las cuales pudieron ser arregladas
en tiempo y forma para lograr tener una entrega exitosa.

El sistema \texttt{PEAT} fue recibido muy bien por los usuarios y el cliente PG\&E
tenido ya cuatro años en funcionamiento dando servicio a cerca de 250,000 usuarios
PyMES\cite{30_pge_annual_report}. Su éxito provoco que el sistema fuera
también adoptado por otras compañías proveedoras de electricidad y gas como
San Diego Gas \& Electric y Southern California Edison.
